\documentclass[a4paper,11pt]{ltjsarticle}


% 数式
\usepackage{amsmath,amsfonts}
\usepackage{amssymb}
\usepackage{amsthm}
\usepackage{bm}
\usepackage{physics}
% 画像
\usepackage{graphics}
\usepackage{graphicx}
\usepackage{here} %画像の表示位置調整用
\usepackage{type1cm}
\usepackage{hyperref}
\usepackage{isotope}
\usepackage[style=phys,articletitle=false,biblabel=brackets,chaptertitle=false,pageranges=false]{biblatex}
\addbibresource{green.bib}
%A4: 21.0 x 29.7cm
\newtheorem{theorem}{定理}[section]   % sectionごとに番号リセット
\newtheorem{lemma}[theorem]{補題}      % theoremと同じカウンタを共有
\newtheorem{proposition}[theorem]{命題}
\newtheorem{corollary}[theorem]{系}

\begin{document}

\title{test}
\author{長尾 昂青}
\date{\today}
\maketitle

\tableofcontents

\newpage
\section{Green's Functions, T- and S-Matrices}

\cite{canto2013scattering}を参照すること.

\subsection{Lippmann-Schwinger equations}
波数ベクトル$\bm{k}$,エネルギー$E_{\bm{k}} := \hbar^2 k^2 / 2\mu$の自由粒子のシュレーディンガー方程式は,
\begin{align}\label{eq:free_p_shcrodinger}
  (E_k - H_0) \ket{\phi(\bm{k})} = 0
\end{align}
であり,$r>\bar{R}$で消えるポテンシャル$V$をもつシュレーディンガー方程式は,
\begin{align}\label{eq:int_p_schrodinger}
  (E_k - H)\ket{\psi(\bm{k})} = 0
\end{align}
と書ける.ここで,$H_0 = K$は運動エネルギー演算子であり,$H = H_0 + V$である.

$H = H^\dagger, H_0 = H_0^\dagger$であることから,$\ket{\phi(\bm{k})}$と$\ket{\psi(\bm{k})}$は,以下の関係で正規化される.
\begin{align}\label{eq:normalization_condition}
  \braket{\phi(\bm{k}')}{\phi(\bm{k})} &= \delta (\bm{k}- \bm{k}') \\
  \braket{\psi(\bm{k}')}{\psi(\bm{k})} &= \delta (\bm{k}-\bm{k}') \\
  \braket{\psi_m}{\psi(\bm{k})} &= 0 \\
  \braket{\psi_m}{\psi_n} &= \delta_{mn}
\end{align}
$m$,$n$はハミルトニアン$H$の負エネルギー固有状態を表している.
$\ket{\psi(\bm{k})}$,$\ket{\phi(\bm{k})}$
における完全性の式は,
\begin{gather}\label{eq:complete_set}
  \int \ket{\phi(\bm{k})} \dd[3]{\bm{k}} \bra{\phi(\bm{k})} = \bm{1} \\
  \int \ket{\psi(\bm{k})} \dd[3]{\bm{k}} \bra{\psi(\bm{k})} + \sum_n \ket{\psi_n} \bra{\psi_n} = \bm{1} 
\end{gather}

freeとfullのグリーン関数はそれぞれ,
\begin{align}\label{eq:def_green_fnc}
  G_0(E) = \frac{1}{E - H_0} \\
  G(E) = \frac{1}{E- H}
\end{align}
と定義される.
$G$と$G_0$の間の関係を求めるために,
\begin{align*}
  A^{-1} = B^{-1} + B^{-1}(B-A)A^{-1}
\end{align*}
という恒等式を用い,$A = E-H = E - (H_0 + V)$,$B = E - H_0$
と置き換えることで,
\begin{align}\label{eq:G_equal_G_0_equation}
  G(E) = G_0(E) + G_0(E)V G(E)
\end{align}
となり,$A = E - H_0$,$B = E-(H_0 +V)$と置き換えることで,
\begin{align}\label{eq:G_0_equal_G_equation}
  G_0(E) = G(E) + G(E)(-V)G_0(E) \notag \\
  G(E) = G_0(E) + G(E) V G_0(E)  
\end{align}
となる.

$\ket{\psi(\bm{k})}$についてみてみよう.
いま,式\ref{eq:int_p_schrodinger}において,
\begin{align}\label{eq:int_p_shcro_rev}
  (E_k-H_0) \ket{\psi(\bm{k})} = V \ket{\psi(\bm{k})}
\end{align}
と書き換えて,
$\ket{\psi(\bm{k})} = \ket{\phi(\bm{k})} + \ket{\psi^\mathrm{sc} (\bm{k})}$と分解し,式\ref{eq:int_p_shcro_rev}
に代入し式\ref{eq:free_p_shcrodinger}を用いると,
\begin{align}\label{eq:int_p_shcro_rev_2}
  (E_k - H_0) \ket{\psi^\mathrm{sc}(\bm{k})} = V \ket{\psi(\bm{k})}
\end{align}
を得る.式\ref{eq:int_p_shcro_rev_2}に左から$G_0(E_k)$を掛けて,
$\ket{\psi^\mathrm{sc}(\bm{k})} = \ket{\psi (\bm{k})} - \ket{\phi(\bm{k})}$
を代入して,
\begin{align}\label{eq:lippman_schwinger_eq_1}
  \ket{\psi(\bm{k})} = \ket{\phi(\bm{k})} + G_0(E_k) V \ket{\psi(\bm{k})}
\end{align}
を得る.
一方で,式\ref{eq:int_p_shcro_rev_2}の右辺に,$\ket{\psi(\bm{k})} = \ket{\phi(\bm{k})} + \ket{\psi^\mathrm{sc} (\bm{k})}$
という関係式を用いることで,
\begin{align}\label{eq:inti_p_shcro_rev_3}
  (E_k - H_0 - V) \ket{\psi^\mathrm{sc}(\bm{k})} = V \ket{\phi(\bm{k})}
\end{align}
となり,左から$G(E_k)$を掛けて,
\begin{align*}
  \ket{\psi^\mathrm{sc}(\bm{k})} = G(E_k) V \ket{\phi(\bm{k})}
\end{align*}
を得る.よって,$\ket{\psi(\bm{k})} = \ket{\phi(\bm{k})} + \ket{\psi^\mathrm{sc} (\bm{k})}$
を用いると,
\begin{align}\label{eq:lippman_scwinger_eq_2}
  \ket{\psi(\bm{k})} = \ket{\phi(\bm{k})} + G(E_k) V \ket{\phi(\bm{k})} = (1 + G(E_k) V) \ket{\phi(\bm{k})}
\end{align}
となる.式\ref{eq:lippman_schwinger_eq_1},\ref{eq:lippman_scwinger_eq_2}がリップマン・シュウィンガー方程式である.

\subsubsection{The free particle Green's function}





















\newpage
% \bibliographystyle{unsrt}
% \bibliography{main}
\printbibliography% biblatex
\end{document}