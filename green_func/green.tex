\documentclass[a4paper,11pt]{ltjsarticle}


% 数式
\usepackage{amsmath,amsfonts}
\usepackage{amssymb}
\usepackage{amsthm}
\usepackage{bm}
\usepackage{physics}
% 画像
\usepackage{graphics}
\usepackage{graphicx}
\usepackage{here} %画像の表示位置調整用
\usepackage{type1cm}
\usepackage{hyperref}
\usepackage{isotope}
\usepackage[style=phys,articletitle=false,biblabel=brackets,chaptertitle=false,pageranges=false]{biblatex}
\addbibresource{green.bib}
%A4: 21.0 x 29.7cm
\newtheorem{theorem}{定理}[section]   % sectionごとに番号リセット
\newtheorem{lemma}[theorem]{補題}      % theoremと同じカウンタを共有
\newtheorem{proposition}[theorem]{命題}
\newtheorem{corollary}[theorem]{系}

\begin{document}

\title{test}
\author{長尾 昂青}
\date{\today}
\maketitle

\tableofcontents

\newpage
\section{Green's Functions, T- and S-Matrices}

\cite{canto2013scattering}を参照すること.

\subsection{Lippmann-Schwinger equations}
波数ベクトル$\bm{k}$,エネルギー$E_{\bm{k}} := \hbar^2 k^2 / 2\mu$の自由粒子のシュレーディンガー方程式は,
\begin{align}\label{eq:free_p_shcrodinger}
  (E_k - H_0) \ket{\phi(\bm{k})} = 0
\end{align}
であり,$r>\bar{R}$で消えるポテンシャル$V$をもつシュレーディンガー方程式は,
\begin{align}\label{eq:int_p_schrodinger}
  (E_k - H)\ket{\psi(\bm{k})} = 0
\end{align}
と書ける.ここで,$H_0 = K$は運動エネルギー演算子であり,$H = H_0 + V$である.

$H = H^\dagger, H_0 = H_0^\dagger$であることから,$\ket{\phi(\bm{k})}$と$\ket{\psi(\bm{k})}$は,以下の関係で正規化される.
\begin{align}\label{eq:normalization_condition}
  \braket{\phi(\bm{k}')}{\phi(\bm{k})} &= \delta (\bm{k}- \bm{k}') \\
  \braket{\psi(\bm{k}')}{\psi(\bm{k})} &= \delta (\bm{k}-\bm{k}') \\
  \braket{\psi_m}{\psi(\bm{k})} &= 0 \\
  \braket{\psi_m}{\psi_n} &= \delta_{mn}
\end{align}
$m$,$n$はハミルトニアン$H$の負エネルギー固有状態を表している.
$\ket{\psi(\bm{k})}$,$\ket{\phi(\bm{k})}$
における完全性の式は,
\begin{gather}\label{eq:complete_set}
  \int \ket{\phi(\bm{k})} \dd[3]{\bm{k}} \bra{\phi(\bm{k})} = \bm{1} \\
  \int \ket{\psi(\bm{k})} \dd[3]{\bm{k}} \bra{\psi(\bm{k})} + \sum_n \ket{\psi_n} \bra{\psi_n} = \bm{1} 
\end{gather}

freeとfullのグリーン関数はそれぞれ,
\begin{align}\label{eq:def_green_fnc}
  G_0(E) = \frac{1}{E - H_0} \\
  G(E) = \frac{1}{E- H}
\end{align}
と定義される.
$G$と$G_0$の間の関係を求めるために,
\begin{align*}
  A^{-1} = B^{-1} + B^{-1}(B-A)A^{-1}
\end{align*}
という恒等式を用い,$A = E-H = E - (H_0 + V)$,$B = E - H_0$
と置き換えることで,
\begin{align}\label{eq:G_equal_G_0_equation}
  G(E) = G_0(E) + G_0(E)V G(E)
\end{align}
となり,$A = E - H_0$,$B = E-(H_0 +V)$と置き換えることで,
\begin{align}\label{eq:G_0_equal_G_equation}
  G_0(E) = G(E) + G(E)(-V)G_0(E) \notag \\
  G(E) = G_0(E) + G(E) V G_0(E)  
\end{align}
となる.

$\ket{\psi(\bm{k})}$についてみてみよう.
いま,式\ref{eq:int_p_schrodinger}において,
\begin{align}\label{eq:int_p_shcro_rev}
  (E_k-H_0) \ket{\psi(\bm{k})} = V \ket{\psi(\bm{k})}
\end{align}
と書き換えて,
$\ket{\psi(\bm{k})} = \ket{\phi(\bm{k})} + \ket{\psi^\mathrm{sc} (\bm{k})}$と分解し,式\ref{eq:int_p_shcro_rev}
に代入し式\ref{eq:free_p_shcrodinger}を用いると,
\begin{align}\label{eq:int_p_shcro_rev_2}
  (E_k - H_0) \ket{\psi^\mathrm{sc}(\bm{k})} = V \ket{\psi(\bm{k})}
\end{align}
を得る.式\ref{eq:int_p_shcro_rev_2}に左から$G_0(E_k)$を掛けて,
$\ket{\psi^\mathrm{sc}(\bm{k})} = \ket{\psi (\bm{k})} - \ket{\phi(\bm{k})}$
を代入して,
\begin{align}\label{eq:lippman_schwinger_eq_1}
  \ket{\psi(\bm{k})} = \ket{\phi(\bm{k})} + G_0(E_k) V \ket{\psi(\bm{k})}
\end{align}
を得る.
一方で,式\ref{eq:int_p_shcro_rev_2}の右辺に,$\ket{\psi(\bm{k})} = \ket{\phi(\bm{k})} + \ket{\psi^\mathrm{sc} (\bm{k})}$
という関係式を用いることで,
\begin{align}\label{eq:inti_p_shcro_rev_3}
  (E_k - H_0 - V) \ket{\psi^\mathrm{sc}(\bm{k})} = V \ket{\phi(\bm{k})}
\end{align}
となり,左から$G(E_k)$を掛けて,
\begin{align*}
  \ket{\psi^\mathrm{sc}(\bm{k})} = G(E_k) V \ket{\phi(\bm{k})}
\end{align*}
を得る.よって,$\ket{\psi(\bm{k})} = \ket{\phi(\bm{k})} + \ket{\psi^\mathrm{sc} (\bm{k})}$
を用いると,
\begin{align}\label{eq:lippman_scwinger_eq_2}
  \ket{\psi(\bm{k})} = \ket{\phi(\bm{k})} + G(E_k) V \ket{\phi(\bm{k})} = (1 + G(E_k) V) \ket{\phi(\bm{k})}
\end{align}
となる.式\ref{eq:lippman_schwinger_eq_1},\ref{eq:lippman_scwinger_eq_2}がリップマン・シュウィンガー方程式である.

\subsubsection{The M\o ller wave operators}
散乱状態と平面波の間の関係を結ぶM\o ller wave operators,$\Omega^{(\pm)}$を導入すると便利であろう.
\begin{align}\label{eq:def_Moller_wave_op}
  \Omega^{(\pm)} \ket{\phi(\bm{k})} = \ket{\psi^{(\pm)}(\bm{k})}
\end{align}
これは,リップマンシュウィンガー方程式から,
\begin{align*}
  \Omega^{(\pm)}\ket{\phi(\bm{k})} = \qty[1 + G_0^{(\pm)}(E_k) V \Omega^{(\pm)}] \ket{\phi(\bm{k})}
\end{align*}
という関係式があるため,M\o ller wave operatorsは,エネルギー$E_k$依存性をもち,
\begin{align}\label{eq:Moller_wop_expression}
  \Omega^{(\pm)} (E_k)= 1 + G_0^{(\pm)}(E_k) V \Omega^{(\pm)} 
\end{align}
と表されることがわかる.
また,
\begin{align}
  \Omega^{(\pm)} (E_k) = 1 + G^{(\pm)}(E_k) V
\end{align}
という関係もわかる.また,$\Omega^{(\pm)}$は
\begin{align}
  \Omega^{(\pm)} = \int \dd[3]{\bm{k}} \ket{\psi^{(\pm )} (\bm{k})} \bra{\phi(\bm{k})}
\end{align}
と表されることもわかる.この表式から,
\begin{align}
  \Omega^{(\pm)\dagger} \Omega^{(\pm)} &= \bm{1} \\
  \Omega^{(\pm)} \Omega^{(\pm)\dagger} &= 1 - \sum_n \ket{\psi_n} \bra{\psi_n}
\end{align}
が満たされることもわかる.

いくつかの便利な恒等式を導こう.式\ref{eq:Moller_wop_expression}から,
\begin{align}
  \Omega^{(\pm)\dagger} = 1 + \Omega^{(\pm)\dagger} V G_0^{(\pm) \dagger} = 1 + \Omega^{(\pm)\dagger} V G_0^{(\mp)}
\end{align}
となるため,
\begin{align}
  \Omega^{(\pm)\dagger} = \frac{1}{1-VG_0^{(\mp)}} \\
  \Omega^{(\pm)} = \frac{1}{1-G_0^{(\pm)}V}.
\end{align}



\subsection{The transition and the scattering operators}

散乱振幅は,散乱状態と平面波のポテンシャルの行列要素に比例している.
\begin{align}\label{eq:scattering_amplitude}
  f(\theta) = -2\pi^2 \qty(\frac{2\mu}{\hbar^2}) \mel{\phi(\bm{q})}{V}{\psi^{(\pm)}(\bm{k})}
\end{align}
これらは,同じ基底による表現になっていないため,
前節で導入したM\o ller演算子を用いて,
\begin{align}\label{eq:scattering_amplitude_moller}
  f(\theta) = -2\pi^2\qty(\frac{2\mu}{\hbar^2}) \mel{\phi(\bm{k}')}{V \Omega{(+)}}{\phi(\bm{k})}
\end{align}
と書き換える.遷移演算子は,
\begin{align}\label{eq:def_transition_operator}
  T = V \Omega^{(+)}
\end{align}
で定義され,自由状態における$T$の行列要素は,
\begin{align}\label{eq:mel_T_free}
  T_{\bm{k}', \bm{k}} \equiv \mel{\phi(\bm{k}')}{T}{\phi(\bm{k})}
\end{align}
と書かれ,$T$行列と呼ばれる.$\bm{k}' = \bm{k}$において運動量$\hbar \bm{k}$から$\hbar \bm{k}'$
へのポテンシャル$V$の下での遷移確率を与える.
式\ref{eq:mel_T_free}を式\ref{eq:scattering_amplitude}に代入することで,
\begin{align}\label{eq:scattering_amplitude_tmatrix}
  f(\theta) = -2\pi^2 \qty(\frac{2\mu}{\hbar^2}) T_{\bm{k}', \bm{k}}
\end{align}
を得る.
それゆえ,$T$行列は散乱振幅を与え,断面積を得ることができる.

式\ref{eq:mel_T_free}は$G$行列のパワーシリーズで書くことができ,
リップマンシュウィンガー方程式から,
\begin{align}\label{eq:tmatrix_power_series}
  T_{\bm{k}', \bm{k}} = \mel{\phi(\bm{k}')}{V\sum_{n=0}^{\infty} \qty(G_0^{(+)}(E_k)V)^n}{\phi(\bm{k})}
\end{align}
となり,これを\emph{ボルン近似}と呼ぶ.$n=0$のみをとったものを一次ボルン近似,
あるいは単にボルン近似と呼ぶ.

式\ref{eq:Moller_wop_expression}の+に対して,左から$V$を作用させると,
\begin{align}\label{eq:LS_for_t}
  T = V + V G_0^{(+)}(E_k)T
\end{align}
を得る.また,
\begin{align}\label{eq:LS_for_t_g}
  T = V + VG^{(+)}(E_k)V
\end{align}
もわかる.

遷移演算子$T$は,$\Omega^{-}$についても与えることができ,
\begin{align}\label{eq:omega_minus_t}
  V \Omega^{(-)}(E) = V + V G^{(-)}(E) V
\end{align}
となる.この式でエルミート共役をとり,$V^\dagger = V$と$\qty[G^{(-)(E)}] ^\dagger = G^{(+)}(E) $
を用いることで,
\begin{align}\label{eq:omega_minus_t_dagger}
  \qty[\Omega^{(-)}(E)]^\dagger V = V + V G^{(+)}(E) V
\end{align}
となる.式\ref{eq:omega_minus_t_dagger}と式\ref{eq:LS_for_t_g}を比較することで,
\begin{align}\label{eq:t_for_omega_pm}
  T = V \Omega^{(+)} = \Omega^{(-)\dagger} V
\end{align}
を得る.また,散乱振幅についても$\Omega^{(-)}$を用いて,
\begin{align}
  f(\theta) = - 2\pi^2 \frac{2\mu}{\hbar^2} \mel{\psi^{(-)}(\bm{k}')}{V}{\phi(\bm{k})}
\end{align}
と書けることがわかる.
\subsubsection{The Optical Theorem}
$T$行列のリップマンシュウィンガー方程式から,光学定理を導出しよう.
\begin{align}\label{eq:completeset}
  \int \ket{\psi^{(+)}(\bm{q})} \dd[3]{\bm{q}} \bra{\psi^{(+)}} + \sum_n \ket{n}\bra{n} = 1
\end{align}
という完全系を用いて,$G^{(+)}(E)$のスペクトル展開を行う.
ここで,$\ket{n}$はエネルギー$E_n = - B_n$の束縛状態を表している.
式\ref{eq:LS_for_t_g}に代入し,平面波状態で挟むことで,
\begin{align}\label{eq:spectral_exp_for_t}
  T_{\bm{k}', \bm{k}}(E) = V_{\bm{k}', \bm{k}} + \sum_n \frac{V_{\bm{k}', n} V_{n, \bm{k}}}{E + B_n} + \int \dd[3]{\bm{q}} \frac{\mel{\phi(\bm{k}')}{V}{\psi^{(+)}(\bm{q})}\mel{\psi^{(+)}(\bm{q})}{V}{\phi(\bm{k})}}{E-E_q + i\epsilon}
\end{align}
となる.ここで,$\mel{\phi(\bm{k}')}{V}{\psi^{(+)}(\bm{q})} = T_{\bm{k}', \bm{q}}(E_q)$,$\mel{\psi^{(+)}(\bm{q})}{V}{\phi(\bm{k})} = T_{\bm{k}, \bm{q}}^*(E_q)$と置き換えることで,
\begin{align}\label{eq:spectral_exp_for_t_2}
  T_{\bm{k}', \bm{k}}(E) = V_{\bm{k}', \bm{k}} + \sum_n \frac{V_{\bm{k}', n} V_{n, \bm{k}}}{E + B_n} + \int \dd[3]{\bm{q}} \frac{T_{\bm{k}', \bm{q}}(E_q)T_{\bm{k}, \bm{q}}^*(E_q)}{E-E_q + i\epsilon}
\end{align}
と書かれる.これは,Law's equationと呼ばれる.

$\bm{k}' = \bm{k}$,$E = E_k$とおいて,式\ref{eq:spectral_exp_for_t_2}の虚部をみることで,
\begin{align}\label{eq:imaginary_part_of_T}
  T_{\bm{k}, \bm{k}}(E_k) - T_{\bm{k}, \bm{k}}^*(E_k) = -2\pi i \int \dd[3]{\bm{q}}T_{\bm{k}, \bm{q}}(E_q) T_{\bm{k}, \bm{q}}^*(E_q)\delta(E-E_q)
\end{align}
積分測度を
\begin{align*}
  \dd[3]{\bm{q}} = q^2 \dd{q} \dd{\Omega_{\bm{q}}} = \frac{\mu q}{\hbar^2} \dd{E_q} \dd{\Omega_{\bm{q}}}
\end{align*}
と置き換えることで,
\begin{align}\label{eq:imaginary_part_of_t_2}
  \mathrm{Im} \qty{T_{\bm{k}, \bm{k}}(E_k)} = - \pi \frac{\mu k}{\hbar^2} \int \dd{\Omega_{\bm{q}}} \abs{T_{\bm{k},\bm{q}}(E_k)}^2
\end{align}
となり,式\ref{eq:scattering_amplitude_tmatrix}を用いることで,
\begin{align}\label{eq:optical_theorem}
  \frac{4\pi}{k}\mathrm{Im}\qty{f(0)} = \int \dd{\Omega} \abs{f(\theta)}^2 = \sigma_\text{el}
\end{align}
を得る.
\subsubsection{The S-matrix}












\newpage
% \bibliographystyle{unsrt}
% \bibliography{main}
\printbibliography% biblatex
\end{document}