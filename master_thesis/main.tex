\documentclass[a4paper,11pt]{ltjsarticle}


% 数式
\usepackage{amsmath,amsfonts}
\usepackage{amssymb}
\usepackage{bm}
\usepackage{physics}
% 画像
\usepackage{graphics}
\usepackage{graphicx}
\usepackage{here} %画像の表示位置調整用
\usepackage{type1cm}
\usepackage{hyperref}
%A4: 21.0 x 29.7cm


\begin{document}

\title{test}
\author{長尾 昂青}
\date{\today}
\maketitle

\tableofcontents

\newpage
\section{はじめに}

$^{12}$Cの核融合

\begin{align}\label{eq:fusion}
  ^{12} \text{C} + ^{12} \text{C} \rightarrow ^{24} \text{Mg}
\end{align}
式\ref{eq:fusion}は巨大恒星の進化の終盤において,重要な役割を担う.
現状では,この過程における反応率は,
実験で得られた断面積を低エネルギー領域へと外挿することに
よってのみ,見積もられている.
しかしながら,この外挿は$^{12}$C + $^{12}$C核融合反応の低エネルギー,重心系での運動エネルギー
$E_\text{c.m.} \lesssim 7$MeV,における共鳴構造により難しくなっている.

さらに,片方の原子核に中性子を一つ加えた形である,$^{12}\mathrm{C} + ^{13}\mathrm{C}$

\end{document}