\documentclass[a4paper,11pt]{ltjsarticle}


% 数式
\usepackage{amsmath,amsfonts}
\usepackage{amssymb}
\usepackage{bm}
\usepackage{physics}
% 画像
\usepackage{graphics}
\usepackage{graphicx}
\usepackage{here} %画像の表示位置調整用
\usepackage{type1cm}
\usepackage{hyperref}
\usepackage{isotope}
\usepackage[style=phys,articletitle=false,biblabel=brackets,chaptertitle=false,pageranges=false]{biblatex}
\addbibresource{main.bib}
%A4: 21.0 x 29.7cm


\begin{document}

\title{test}
\author{長尾 昂青}
\date{\today}
\maketitle

\tableofcontents

\newpage
\section{はじめに}

$^{12}$Cの核融合

\begin{align}\label{eq:fusion}
  \isotope[12]{C} + \isotope[12]{C} \rightarrow  \isotope[24]{Mg}
\end{align}
式\ref{eq:fusion}は巨大恒星の進化の終盤において,重要な役割を担う.
現状では,この過程における反応率は,
実験で得られた断面積を低エネルギー領域へと外挿することに
よってのみ,見積もられている.
しかしながら,この外挿は$\isotope[12]{C} + \isotope[12]{C}$核融合反応の低エネルギー,重心系での運動エネルギー
$E_\text{c.m.} \lesssim 7$MeV,における共鳴構造により難しくなっている.

さらに,片方の原子核に中性子を一つ加えた形である,$\isotope[12]{C} + \isotope[13]{C}$核融合反応との違いも注目を集めている.
これら二つの系の断面積の振る舞いを比べたら一目瞭然であるが,

\section{Potential}

本研究では,原子核間の有効ポテンシャルをいくつか用いる.
\subsection{M3Y + repulsive double folding potential}
一つ目のポテンシャルとして,Esbensenら\cite{Esbensen2011}がC+C核融合反応の解析に用いた,M3Y + repulsive Double Folding potentialを用いる.
このポテンシャルは,低エネルギー重イオン核融合反応におけるhindranceを説明するために導入されたポテンシャル\cite{PhysRevC.75.034606}であり,
$\isotope[60]{Ni} + \isotope[89]{Y}$の核融合反応において最初に観測された\cite{PhysRevLett.89.052701}.

M3Y double folding potentialは,
\begin{align}
  U_n (\bm{r}) = \int \dd{\bm{r}_1} \dd{\bm{r}_2} \rho_1(\bm{r}_1) \rho_2(\bm{r}_2) v_\mathrm{M3Y} (\bm{r} + \bm{r}_2 - \bm{r}_1)
\end{align}
のように定義される.
ここで,$v_\mathrm{M3Y}(\bm{r})$は,Reidポテンシャルから導かれる有効核子間相互作用である.


\section{S行列の求め方と,ポール探索}


\subsection{ハミルトニアン}
$\isotope{C} + \isotope{C}$チャネルと,複合核($\isotope{Mg}$)チャネルを記述するハミルトニアンは,
行列形式で,
\begin{align}\label{eq:total_hamiltonian}
  \bm{H} = 
  \begin{pmatrix}
    \bm{H}_\text{C+C} & \bm{V} \\
    \bm{V}^T & \bm{H}_\text{CN}
  \end{pmatrix}
\end{align}
のように書かれる.
ここで,$\bm{H}_\text{C+C}$はC+Cチャネルのハミルトニアン,$\bm{H}_\text{CN}$は複合核チャネルのハミルトニアン,$\bm{V}$はカップリング行列である.

$\bm{H}_\text{C+C}$は,メッシュ間隔$\Delta r$であり,サイトが$r_i = i \Delta r$,($i = 1, \dots , N_\text{C+C}$)
をもつ,動径方向の相対運動シュレーディンガー方程式,
\begin{align}\label{eq:entrance_hamiltonian}
  \bm{H}_{\text{C+C}, ij} = \qty[2t + V(r_i)]\delta_{ij} - t\delta_{i, j+1} - t\delta_{i,j-1}
\end{align}

を解く.


\subsection{S行列の計算}

\subsection{S行列のポール探索}
共鳴は,$S$行列のポールとして定義され,共鳴エネルギーを持つとき,
系は外向波境界条件が課される.
つまり,
\begin{align}
  \frac{u(N_\text{C+C} + 1) }{ u(N_\text{C+C})}= \frac{H_l^+(k (N_\text{C+C} + 1)) }{ H_l^+(k N_\text{C+C})}
\end{align}
と,$u(0) = 0$を境界条件として固有値問題を解く.
\begin{align}
  \bm{M}(k) \vec{u} = \qty(\bm{H} - t )
\end{align}





















\newpage
% \bibliographystyle{unsrt}
% \bibliography{main}
\printbibliography% biblatex
\end{document}