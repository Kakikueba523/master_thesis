\documentclass[a4paper,11pt]{ltjsarticle}


% 数式
\usepackage{amsmath,amsfonts}
\usepackage{amssymb}
\usepackage{bm}
\usepackage{physics}
% 画像
\usepackage{graphics}
\usepackage{graphicx}
\usepackage{here} %画像の表示位置調整用
\usepackage{type1cm}
\usepackage{hyperref}
\usepackage{isotope}
\usepackage[style=phys,articletitle=false,biblabel=brackets,chaptertitle=false,pageranges=false]{biblatex}
\addbibresource{main.bib}
%A4: 21.0 x 29.7cm

\numberwithin{figure}{section}
\numberwithin{equation}{section}
\numberwithin{table}{section}



\begin{document}

\title{炭素核融合反応の微視的原子核模型による記述}
\author{長尾 昂青}
\date{\today}
\maketitle

\tableofcontents

\newpage
\section{はじめに}

$\isotope[12]{C} + \isotope[12]{C}$核融合反応は,巨大恒星の進化の過程や宇宙における
爆発現象に深く関わっている.
\begin{align}
  \isotope[12]{C} + \isotope[12]{C} \rightarrow  \isotope[24]{Mg}^* &\rightarrow \isotope[23]{Na} + p \notag \\
                                                                  &\rightarrow \isotope[20]{Ne} + \alpha \notag
\end{align}
が主反応となっている.
現状では,この過程における反応率は,
実験で得られた断面積を低エネルギー領域へと外挿することに
よってのみ,見積もられている.
しかしながら,この外挿は$\isotope[12]{C} + \isotope[12]{C}$核融合反応の低エネルギー,重心系での運動エネルギー
$E_\text{c.m.} \lesssim 7$MeV,における共鳴構造により難しくなっている.

さらに,片方の原子核に中性子を一つ加えた形である,$\isotope[12]{C} + \isotope[13]{C}$核融合反応との違いも注目を集めている.
これら二つの系の断面積の振る舞いを比べたら一目瞭然であるが,

\section{散乱理論}
本研究において,重要なパートを担う散乱理論についてこの節で議論する.
クーロン場がない場合から始め,ある場合について述べる.
また,同種粒子同士の散乱問題についても,相互作用がスピンに依らない場合に限り,議論する.

\subsection{原子核間ポテンシャル}
原子核間ポテンシャルは,相対距離$r$の関数として,
\begin{align}\label{eq:general_ion_potential}
  V(r) = V_N(r) + V_C(r)
\end{align}
と書かれる.ここで,$V_N(r)$核力に由来する原子核ポテンシャルであり,$V_C(r)$がクーロンポテンシャルである.
入射粒子と標的粒子の電荷を$Z_1,Z_2$とすると,クーロンポテンシャルは,
\begin{align}\label{eq:naive_coulomb}
  V_C(r) = \frac{Z_1 Z_2 e^2}{r}
\end{align}
と書かれる.

\begin{figure}[htpb]
  \centering
  \includegraphics[width=8cm]{figure/potential_example.pdf}
  \caption{原子核間ポテンシャルの相対角運動量が0の場合の例.赤の点線がクーロンポテンシャルであり,青の破線が核力に由来するポテンシャルであり,全ポテンシャルが
  その和を表している.}
  \label{fig:potential_example}
\end{figure}

図~\ref{fig:potential_example}に式~(\ref{eq:general_ion_potential})の例をプロットしている.
点線がクーロンポテンシャルであり,破線が核力であり,その和が実線でプロットされている.
短距離力である核力と,長距離力であるクーロンポテンシャルの差分により,ポテンシャル障壁が
生じていることがわかる.このポテンシャル障壁はクーロン障壁と呼ばれ,核融合反応が生じるためには
この障壁を越えなければならない.

\subsection{クーロン場がないときの散乱理論}

% 短距離ポテンシャル$V_0(r)$があるときの,シュレーディンガー方程式は,
% \begin{align}\label{eq:shrodinger_eq_wo_coulomb}
%   \qty[-\frac{\hbar^2}{2\mu}\nabla^2 + V_0(r) - E] \psi(\bm{r}) = 0
% \end{align}
% のように書かれる.ポテンシャルがない場合,波数ベクトル$\bm{k}$,$\abs{\bm{k}} = \sqrt{2\mu E/ \hbar}$を用いて,式~(\ref{eq:shrodinger_eq_wo_coulomb})の解は,
% $\psi(\bm{r}) = \exp(i \bm{k} \cdot \bm{r})$と書かれる.

短距離ポテンシャル$V_0(r)$の下で,シュレーディンガー方程式は,波数ベクトル$\bm{k}$,
大きさ$k = \abs{\bm{k}} = \sqrt{2\mu E}/\hbar$を用いて,
\begin{align}\label{eq:stationary_schrodinger_eq}
  \qty(\nabla^2 + k^2) \psi(\bm{r}) = U(r) \psi(\bm{r})
\end{align}
と書ける.ここで,$\mu$は系の換算質量であり,$U(r) = \frac{2\mu}{\hbar^2}V_0(r)$である.
式~(\ref{eq:stationary_schrodinger_eq})の解は,平面波と散乱波の二つに分解できて,
\begin{align}\label{eq:desired_wave_function}
  \psi^{(+)}(\bm{k}; \bm{r}) = \phi(\bm{k};\bm{r}) + \psi^{\text{sc}}(\bm{k}; \bm{r})
\end{align}
と書かれる.ここで,
\begin{align}\label{eq:plane_wave}
  \phi(\bm{k};\bm{r}) = A e^{i \bm{k} \cdot \bm{r}}
\end{align}
および,$\psi^{(+)}(\bm{k};\bm{r})$は散乱源により生じた外向き波である.よって,$r \rightarrow \infty$において,
\begin{align}\label{eq:scatt_asympt_form}
  \psi^{\text{sc}}(\bm{k};\bm{r}) \rightarrow A f(\Omega) \frac{e^{ikr}}{r}
\end{align}
のような漸近的な振る舞いをする.
ここで,$f(\Omega) \equiv f(\theta, \varphi)$は散乱振幅であり,断面積の計算において重要な役割を担う.
式~(\ref{eq:plane_wave}),~(\ref{eq:scatt_asympt_form})より,
\begin{align}\label{eq:outgoing_asympt_form}
  \psi^{\text{(+)}}(\bm{k};\bm{r}) \rightarrow A\qty(e^{i\bm{k}\cdot \bm{r}} + f(\Omega)\frac{e^{ikr}}{r})
\end{align}
なる漸近形が得られる.

\subsubsection{部分波展開}
別のアプローチとして,部分波展開と呼ばれる方法がある.これは,ポテンシャルが球対称な場合かつ
低エネルギーの解析において極めて有効な方法である.
ハミルトニアンは運動エネルギー演算子$K \equiv -\frac{\hbar^2}{2\mu}\nabla^2$と,ポテンシャル$V$を用いて,
\begin{align}\label{eq:hamiltonian_op}
  H = K + V
\end{align}
と書こう.このときシュレーディンガー方程式は,
\begin{align}\label{eq:shcrodinger_eq_op}
  H \psi = E \psi 
\end{align}
となる.ここで,$E$は衝突エネルギーである.

まず,運動エネルギー演算子$K \equiv -\frac{\hbar^2}{2\mu}\nabla^2$を球座標$(r, \theta, \varphi)$で書き換える.
\begin{align}\label{eq:kinetic_op_in_spherical}
  K = -\frac{\hbar^2}{2\mu}\qty[\frac{1}{r^2}\pdv{r}\qty(r^2\pdv{r}) + \frac{1}{r^2\sin^2 \theta}\pdv{\theta}\qty(\sin \theta \pdv{\theta}) + \frac{1}{r^2\sin^2\theta} \pdv[2]{}{\varphi}]
\end{align}
軌道角運動量演算子$L^2$が,
\begin{align}\label{eq:angular_momentum_op}
  L^2 = -\frac{\hbar^2}{\sin^2\theta}\qty[\pdv{\theta}\qty(\sin\theta \pdv{\theta}) + \pdv[2]{}{\varphi}]
\end{align}
で書かれることから,運動エネルギー演算子は,
\begin{align}\label{eq:kinetic_op_angular_op}
  K = - \frac{\hbar^2}{2\mu} \qty[\frac{1}{r^2}\pdv{r}\qty(r^2\pdv{r})-\frac{L^2}{\hbar^2r^2}]
\end{align}
と書き直すことができる.ポテンシャルが球対称な場合には,式~(\ref{eq:kinetic_op_angular_op})から,ハミルトニアン$H$と
$L^2$および$L_z$が可換であることがわかる.
よって,式~(\ref{eq:shcrodinger_eq_op})の解は,演算子$H,L^2, L_z$の同時固有状態で書くことができ,
\begin{align}\label{eq:shrodinger_eq_eigenf}
  \psi_{klm}(\bm{r}) = R_l(k, r) Y_{l m} (\hat{\bm{r}})
\end{align}
となる.ここで,$\hat{\bm{r}}$は$\theta, \varphi$の略記である.
また,$Y_{lm}$は球面調和関数であり,$R_l$は動径波動関数である.散乱波動関数をこの基底で展開して,
$Y_{lm}$と内積をとることで,
動径方向のシュレーディンガー方程式
\begin{align}\label{eq:radial_eq}
  -\frac{\hbar^2}{2\mu}\qty[\frac{1}{r^2}\dv{r}\qty(r^2\dv{r})-\frac{l(l+1)}{r^2}]R_l(k, r) + V(r) R_l(k, r) = E R_l(k, r)
\end{align}
を得る.さらに,
\begin{align}\label{eq:mod_radiad_wf}
  R_l(k, r) = \frac{u_l(k, r)}{kr}
\end{align}
のように動径波動関数を変換することで,式~(\ref{eq:radial_eq})は,
\begin{align}\label{eq:mod_radial_eq}
  -\frac{\hbar^2}{2\mu}\qty[\dv[2]{}{r} - \frac{l(l+1)}{r^2}]u_l(k, r) + V(r)u_l(k, r) = E u_l(k, r)
\end{align}
となる.これは,ポテンシャルが
\begin{align}
  V_l(r) = V(r) + V_l^{\text{Cf}}(r)
\end{align}
である,一次元シュレーディンガー方程式となっており,
$V_l^{\text{Cf}}$は遠心力ポテンシャルとよばれ,
\begin{align}\label{eq:centrifugal_potential}
  V_l^{\text{Cf}}(r) = \frac{\hbar^2}{2\mu}\frac{l(l+1)}{r^2}
\end{align}
で定義される.




\section{模型}

本研究では,原子核間の有効ポテンシャルをいくつか用いる.
\subsection{M3Y + repulsive double folding potential}
一つ目のポテンシャルとして,Esbensenら\cite{Esbensen2011}がC+C核融合反応の解析に用いた,M3Y + repulsive Double Folding potentialを用いる.
このポテンシャルは,低エネルギー重イオン核融合反応におけるhindranceを説明するために導入されたポテンシャル\cite{PhysRevC.75.034606}であり,
$\isotope[60]{Ni} + \isotope[89]{Y}$の核融合反応において最初に観測された\cite{PhysRevLett.89.052701}.

M3Y double folding potentialは,
\begin{align}
  U_n (\bm{r}) = \int \dd{\bm{r}_1} \dd{\bm{r}_2} \rho_1(\bm{r}_1) \rho_2(\bm{r}_2) v_\mathrm{M3Y} (\bm{r} + \bm{r}_2 - \bm{r}_1)
\end{align}
のように定義される.
ここで,$v_\mathrm{M3Y}(\bm{r})$は,Reidポテンシャルから導かれる有効核子間相互作用である.

\subsection{Woods-Saxon Potential}



\section{S行列の求め方と,ポール探索}


\subsection{ハミルトニアン}
本研究で用いるハミルトニアンは,文献\cite{PhysRevC.98.014604}で用いられたものと同様のものを用いる.
文献\cite{PhysRevC.98.014604}では,$n + \isotope[194]{Pt}$反応の解析をおこなっており,そこでは$s$波
のみの解析に限られていたが,本研究ではそれを荷電粒子同士の反応,およびすべての部分波$l$で用いられるように拡張した.

具体的には,
$\isotope{C} + \isotope{C}$チャネルおよび複合核($\isotope{Mg}$)チャネルを一つのハミルトニアンで,
陽に取り扱う.系全体のハミルトニアンは,
\begin{align}\label{eq:total_hamiltonian}
  \bm{H}^{l,J^\pi} = 
  \begin{pmatrix}
    \bm{H}^l_\text{C+C} & \bm{V} \\
    \bm{V}^T & \bm{H}^{J^\pi}_\text{CN}
  \end{pmatrix}
\end{align}
のように書かれる.
ここで,$l$は$\isotope{C}+\isotope{C}$チャネルの相対角運動量,$J^\pi$は$\isotope{Mg}$のスピンおよびパリティ,$\bm{H}_\text{C+C}$はC+Cチャネルのハミルトニアン,$\bm{H}_\text{CN}$は複合核チャネルのハミルトニアン,$\bm{V}$はカップリング行列である.

\subsection{入口チャネル}
  $\bm{H}_\text{C+C}$は,
炭素原子核$\isotope[12]{C}$,$\isotope[13]{C}$を球形と近似し,二つの原子核間の相対運動を記述する:
\begin{align}\label{eq:entrance_hamiltonian}
  \bm{H}^l_{\text{C+C}, ij} = \qty[2t + V_l(r_i)]\delta_{ij} - t\delta_{i, j+1} - t\delta_{i,j-1}.
\end{align}
ここで,$i,j = (1, 2, \dots, N_\text{C+C})$,$N_\text{C+C}$は$\isotope{C}+\isotope{C}$チャネルのメッシュのサイト数
であり,$t = \hbar^2/ 2\mu \Delta r^2$,$\mu$は換算質量である.
また,$\Delta r$はメッシュ間隔であり,$r_i = i \Delta r$である.
なお,ここで$\isotope[12]{C}$,$\isotope[13]{C}$の励起状態は考えず,質点として取り扱っている.
$V_l(r_i)$は遠心力ポテンシャルを含む,原子核間ポテンシャルであり,原子核間ポテンシャルとしては,いくつかのケースを
試す.

\subsection{複合核チャネル}
  $\bm{H}^{J^\pi}_\text{CN}$は複合核状態を記述するハミルトニアンであり,
\begin{align}\label{eq:compound_hamiltonian}
  \bm{H}^{J^\pi}_{\text{CN}, \mu \nu} = \qty[\mathcal{E}^{J^\pi}_\mu - i \Gamma^{J^\pi}(\mathcal{E}^{J^\pi}_\mu)/2]\delta_{\mu\nu}
\end{align}
と書かれる($\mu = 1, 2, \cdots , N_\text{CN}$).ここで,$\mathcal{E}^{J^\pi}_\mu$はスピン$J^\pi$を持つ複合核の第$\mu$励起エネルギーでそれに対応する崩壊幅$\Gamma^J(\mathcal{E}_\mu)$を
虚部に加えている.
複合核の励起スペクトルとして,本研究では殻模型を用いて計算を行った.
具体的にはKSHELL\cite{Shimizu2016}というコードを用いて,$\isotope[24]{Mg}$,$\isotope[25]{Mg}$の励起スペクトルを
計算した.
$\isotope[16]{O}$を不活性コアとし,$sd - pf$軌道をバレンス軌道として,励起スペクトルを計算した.
用いた核子間相互作用はsdpf-mu\cite{PhysRevC.86.051301}と呼ばれる相互作用である.

\subsection{カップリング行列}
  $\bm{V}$は$\isotope{C} + \isotope{C}$チャネルと$\isotope{Mg}$チャネルを結合する行列であり,
本研究では
\begin{align}
  \bm{V}_{i,\mu} = v_0 (\Delta r)^{-1/2} \delta_{i,i_e}
\end{align}
という形を用いる.
ここで,$\Delta r$はメッシュ間隔であり,$i_e$は相互作用点である.
$i_e$の選び方はさまざまあるが,今回は$\isotope{C} + \isotope{C}$チャネルのポテンシャル最小点
でとる.また,$v_0$はカップリングパラメータであり,本研究では,このパラメータを変えながら議論する.
ただし,簡単のために,$v_0$は$l$および$J^\pi$にはよらないパラメータとする.



\subsection{S行列の計算}
核融合断面積は,$\isotope{C} + \isotope{C}$チャネルの境界条件と,\ref{eq:total_hamiltonian}の固有ベクトルから得られる.
$\isotope{C} + \isotope{C}$チャネルの波動関数に境界条件として,
\begin{align}
  u(0) &= 0, \label{eq:boundary_origin}\\
  u(r) &\rightarrow A(k) \qty[H_l^- (kr) - S^{J^\pi}_l H_l^+(kr)] \label{eq:boundary_approxfrom}
\end{align}
を課す.ここで,$S^{J^\pi}_l$は弾性チャネルの$S$行列である.また,$H^\pm_l(kr)$はクーロン・ハンケル関数であり,正則なクーロン関数$F_l(kr)$と
非正則なクーロン関数$G_l(kr)$を用いて,
\begin{align}\label{eq:coulomb_hankel_function}
  H^\pm_l (kr) = G_l(kr) \pm i F_l(kr)
\end{align}
と定義される.
次に,\ref{eq:total_hamiltonian}の固有ベクトルについて考える.\ref{eq:total_hamiltonian}は
$N_\text{C+C} + N_\text{CN}$次元のハミルトニアンだから,固有ベクトルは$N_\text{C+C} + N_\text{CN}$
成分を持ったベクトルである.固有ベクトルを$\vec{u}$とすると,
\begin{align}\label{eq:eigenvector_of_total_H}
  \vec{u} = \begin{pmatrix}
    (N_\text{C+C}\text{成分}) \\
    (N_\text{CN}\text{成分}) 
  \end{pmatrix}
\end{align}
のようになり,上$N_\text{C+C}$成分は$\isotope{C} + \isotope{C}$チャネルの波動関数に対応しており,
下$N_\text{CN}$成分は複合核チャネルの固有ベクトルに対応する.

\ref{eq:boundary_approxfrom}から,メッシュ点$N_\text{C+C}$,$N_\text{C+C}+1$について,
\begin{align}\label{eq:ratio_from_boundary}
  \frac{u(N_\text{C+C})}{u(N_\text{C+C}+1)} = \frac{H_l^-(k\Delta r N_\text{C+C}) - S^{J^\pi}_l H_l^+(k\Delta r N_\text{C+C})}{H_l^-(k\Delta r (N_\text{C+C}+1)) - S^{J^\pi}_l H_l^+ (k\Delta r (N_\text{C+C} + 1))}
\end{align}
を得る.

次に,シュレーディンガー方程式について考える.$\isotope{C}+\isotope{C}$チャネルにおいて
$-t u(N_\text{C+C} + 1)$という,チャネルの外側であり,かつ0でない項を考えなければならない.
よって,エネルギー$E = \hbar^2 k^2 / 2\mu$の散乱解をもつシュレーディンガー方程式は,
\begin{align}\label{eq:total_hamiltonian_eigen_problem}
  (E - \bm{H}^{l, J^\pi}) \vec{u} = \vec{h}
\end{align}
となる.ここで,$h(i) = -tu(N_\text{C+C}+1) \delta_{i, N_\text{C+C}}$である.\ref{eq:total_hamiltonian_eigen_problem}に左から,
$G^{l, J^\pi}(E) = (E - \bm{H}^{l, J^\pi})^{-1}$を掛けて,第$N_\text{C+C}$成分を比較することで,
\begin{align}\label{eq:ratio_from_eigenvector}
  \frac{u(N_\text{C+C})}{u(N_\text{C+C}+1)} = - t G^{l, J^\pi}(E)_{N_\text{C+C}, N_\text{C+C}}
\end{align}
\ref{eq:ratio_from_boundary},\ref{eq:ratio_from_eigenvector}から,$S^{J^\pi}_l$について整理すると,
\begin{align}\label{eq:s_matrix}
  S^{J^\pi}_l = \frac{H_l^- (k\Delta r N_\text{C+C}) + t G^{l, J^\pi}(E)_{N_\text{C+C}, N_\text{C+C}} H^-_l(k\Delta r (N_\text{C+C}+1))}{H_l^+(k \Delta r N_\text{C+C})+ t G^{l, J^\pi}(E)_{N_\text{C+C}, N_\text{C+C}} H_l^+(k\Delta r (N_\text{C+C} + 1))}
\end{align}
を得る.
リアクションチャネルを核融合チャネルとすることで,核融合断面積を$S^{J^\pi}_l$を用いて計算することができる.
ただし,$\isotope[12]{C} + \isotope[12]{C}$反応では,$\isotope[12]{C}$の基底状態が$0^+$のボース粒子であることから,
\begin{align}\label{eq:fus_cs_12_12}
  \sigma_{\text{fus}, \isotope[12]{C} + \isotope[12]{C}} = \frac{\pi}{k^2} \sum_l (2l + 1) \qty(1-\abs{S^{J^\pi=l^+}_l}^2) \qty(1 + (-1)^l)
\end{align}
となる.

また,$\isotope[12]{C} + \isotope[13]{C}$反応では,$\isotope[13]{C}$の基底状態が,$\frac{1}{2}^-$であることから,
\begin{align}\label{eq:fus_cs_12_13}
  \sigma_{\text{fus}, \isotope[12]{C} + \isotope[13]{C}} = \frac{\pi}{k^2} \sum_l \sum_{J^\pi = (l - 1/2)^{(-1)^{l+1}}}^{(l + 1/2)^{(-1)^{l+1}}} \frac{2 J + 1}{2} \qty(1 - \abs{S^{J^\pi}_l}^2)
\end{align}
となる.ここで,$(2J+1)/2$はスピン統計因子であり,始状態の状態数2で割ってある.


\ref{eq:s_matrix}と,\ref{eq:fus_cs_12_12}および\ref{eq:fus_cs_12_13}から,核融合断面積を計算するが,
$S$行列は1にとても近い量であるため,直接計算すると精度が足りない.
よって,$1-\abs{S^{J^\pi}_l}^2$を直接計算する.そのために,解析的な式を以下で与えておく.
\ref{eq:coulomb_hankel_function}を用いて,
\begin{align}
  1 - S^{J^\pi}_l =2i \frac{F_l (k\Delta r N_\text{C+C}) + t G^{l, J^\pi}(E)_{N_\text{C+C}, N_\text{C+C}} F_l(k\Delta r (N_\text{C+C}+1))}{H_l^+(k \Delta r N_\text{C+C})+ t G^{l, J^\pi}(E)_{N_\text{C+C}, N_\text{C+C}} H_l^+(k\Delta r (N_\text{C+C} + 1))}
\end{align}
$1 - S^{J^\pi}_l = \delta^{J^\pi}_l$と$\delta^{J^\pi}_l$を定義すると,
\begin{align}\label{eq:1_minus_s_squared}
  1-\abs{S^{J^\pi}_l}^2 
  &= 1 - \abs{1 - \delta^{J^\pi}_l}^2 \notag \\
  &= 1 - \qty(1 -  \delta^{J^\pi}_l -  (\delta^{J^\pi}_l)^*  + \abs{ \delta^{J^\pi}_l}^2 ) \notag \\
  &= 2 \mathrm{Re}\qty( \delta^{J^\pi}_l) - \abs{ \delta^{J^\pi}_l}^2
\end{align}
となり,この式を用いて,\ref{eq:fus_cs_12_12}および\ref{eq:fus_cs_12_13}を計算する.





















\newpage
% \bibliographystyle{unsrt}
% \bibliography{main}
\printbibliography% biblatex
\end{document}