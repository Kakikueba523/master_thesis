\chapter*{要旨}
\isotope[12]{C}+\isotope[12]{C}核融合反応は,大質量星の進化やX線スーパーバースト,
Ia型超新星爆発などの天体現象において重要な役割を担うと考えられている.
しかしながら,それらの反応のエネルギー領域においては反応率は極めて小さく,断面積の
直接測定は非常に困難である.
それに加えて,核融合断面積に生じる多数の共鳴構造により実験データの低エネルギー側への
外挿は大きな不定性が生じる.

一方で,片方の原子核に中性子を一つ加えた系である\isotope[12]{C}+\isotope[13]{C}核融合反応の断面積は
天体核反応領域においても顕著な共鳴構造はなく,滑らかなエネルギー依存性をみせる.
これらの振る舞いの差は,複合核状態である\isotope[24]{Mg}と\isotope[25]{Mg}の違い
に起因するものだとC.L.Jiangらによって指摘された.
それに加えて,\isotope[12]{C}+\isotope[12]{C}核融合断面積のいくつかの共鳴ピークが,
\isotope[12]{C}+\isotope[13]{C}核融合断面積に一致し,
\isotope[12]{C}+\isotope[13]{C}核融合断面積が\isotope[12]{C}+\isotope[12]{C}
核融合断面積の上限を与えるという
振る舞いを見せるということも知られている.
しかしながら
現状では,複合核状態を陽に取り扱いながら同じ枠組みで\isotope[12]{C}+\isotope[12]{C}
核融合反応と\isotope[12]{C}+\isotope[13]{C}核融合反応を取り扱った研究は行われていない.

本研究の目的は,実験での直接測定が困難な低エネルギー側での計算が可能な模型を構築すること,ならびに\isotope[12]{C}+\isotope[12]{C}核融合反応と\isotope[12]{C}+\isotope[13]{C}核融合反応の断面積を同じ枠組みで計算し,実験で確認されている振る舞いの違いを再現することである.
そのために,本研究では複合核状態を陽に取り扱う反応模型を構築し,\isotope[12]{C}+\isotope[12]{C}核融合反応と
\isotope[12]{C}+\isotope[13]{C}核融合反応を同じ枠組みで計算することを行った.
% 具体的には,複合核状態を核子自由度に基づいた殻模型計算により
% 励起スペクトルを計算することにより用意し,その崩壊幅を統計模型で見積もる.
% そして,
% 計算した複合核状態と散乱状態である\isotope{C}+\isotope{C}の相対運動とを
% 同時に記述することのできる模型を構築し,
% 核融合断面積の計算を行った.
そして,\isotope[12]{C}+\isotope[12]{C}核融合反応断面積では多数の共鳴構造が現れ,
\isotope[12]{C}+\isotope[13]{C}核融合反応では顕著な共鳴構造はなく滑らかなエネルギー依存性を見せるような,二つの系における実験で観測された振る舞いの違いを再現することができた.
また,
実験データに合わせやすい\isotope[12]{C}+\isotope[13]{C}核融合反応から計算をはじめ,
求めた値が\isotope[12]{C}+\isotope[12]{C}核融合断面積の上限になっているという制限をかけることにより,
\isotope[12]{C}+\isotope[12]{C}系に対して追加のフィットを行わずに断面積を評価する手順を与えることに成功した.
そこで,全体の振る舞いを詳細に再現するには至らないものの,\isotope[12]{C}+\isotope[12]{C}核融合反応では,実験で測定されているエネルギー領域においては同じ程度の大きさの共鳴構造を再現することができ,
実験で直接測定されていない低エネルギー側では,実験で得られているエネルギー領域に比べて
\isotope[12]{C}+\isotope[12]{C}核融合反応の共鳴構造がさらに大きなものになっていることを見出した.

