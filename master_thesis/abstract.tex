\chapter*{要旨}
\isotope[12]{C}+\isotope[12]{C}核融合反応は,大質量星の進化やX線スーパーバースト,
Ia型超新星爆発において重要な役割を担うと考えられている.
しかしながら,それらの反応のエネルギー領域においては反応率は極めて小さく,断面積の
測定は非常に困難である.
それに加えて,核融合断面積に生じる共鳴構造により実験データの低エネルギー側への
外挿は大きな不定性が生じる.

一方で,片方の原子核に中性子を一つ加えた系である\isotope[12]{C}+\isotope[13]{C}核融合反応の断面積は
天体核反応領域においても共鳴構造はなく,滑らかなエネルギー依存性をみせる.
これらの振る舞いの差は,複合核状態である\isotope[24]{Mg}と\isotope[25]{Mg}の違い
によるものだとC.L.Jiangらによって指摘された.
それに加えて,\isotope[12]{C}+\isotope[12]{C}核融合断面積の共鳴ピークが,
\isotope[12]{C}+\isotope[13]{C}核融合断面積に一致するという振る舞いも見せるということも知られている.

現状では,複合核状態を陽に取り扱いながら同じ枠組みで\isotope[12]{C}+\isotope[12]{C}
核融合反応と\isotope[12]{C}+\isotope[13]{C}核融合反応を取り扱った研究は行われていない.
そこで,本研究では複合核状態を陽に取り扱う模型を構築し,\isotope[12]{C}+\isotope[12]{C}核融合反応と,
\isotope[12]{C}+\isotope[13]{C}核融合反応を同じ枠組みで計算することを行った.
具体的には,実験データに合わせやすい\isotope[12]{C}+\isotope[13]{C}核融合反応から計算をはじめ,
求めた値が\isotope[12]{C}+\isotope[12]{C}核融合断面積の上限になっているという制限をかけることにより,
\isotope[12]{C}+\isotope[12]{C}核融合断面積を低エネルギー領域まで計算した.
さらに複数の原子核間ポテンシャルを用いることにより,本模型における
原子核間ポテンシャルの依存性を議論した.