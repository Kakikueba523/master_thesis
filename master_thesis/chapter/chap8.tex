\chapter{まとめ}\label{chap:conclusion}
\section*{まとめ}
本研究では,天体核反応エネルギー領域における\isotope[12]{C}+\isotope[12]{C}核融合反応
の断面積を計算した.類似系である\isotope[12]{C}+\isotope[13]{C}核融合反応の断面積の振る舞いの
違いから,中間状態である複合核状態に注目し,同一模型で計算することによりパラメータの
不定性を減らすことに成功した.

\isotope[12]{C}+\isotope[12]{C}核融合反応は,大質量星の進化やX線スーパーバースト
などの爆発的天体現象において重要な役割を果たす.
しかしながら,これらの反応は地上での実験が困難な極めて低エネルギーで起こるため,
断面積の測定は困難である.
それに加えて,\isotope[12]{C}+\isotope[12]{C}核融合反応は低エネルギーでは
断面積に共鳴構造が現れ得られた実験データを低エネルギー側へと外挿するには
大きな不定性が伴う.
よって,理論的なアプローチが必要不可欠である.

また,類似系である\isotope[12]{C}+\isotope[13]{C}核融合反応の断面積のエネルギー依存性
との違いも注目を集めている.片方の原子核に中性子を一つ加えた反応系であるこの系では,
低エネルギーにおいても核融合断面積は滑らかなエネルギー依存性を見せ,共鳴構造が現れない.
さらに,\isotope[12]{C}+\isotope[12]{C}核融合反応における$S$-factorの共鳴ピークが
\isotope[12]{C}+\isotope[13]{C}核融合反応における$S$-factorに一致するという
実験事実も指摘されている.

核融合反応において,入射核と標的核との間のクーロン障壁のトンネリングだけでなく,
複合核形成を陽に取り扱った研究はあまり行われていない.
\isotope[12]{C}+\isotope[12]{C}核融合反応において,低エネルギーでの共鳴構造が
複合核の\isotope[24]{Mg}における孤立共鳴によるものだと考えられている.
よって,複合核状態である\isotope[24]{Mg}の性質,
つまりは準位間隔や崩壊幅を陽に取り扱う必要がある.
そこで,本研究では複合核状態を微視的原子核模型に基づいて計算し,
それを用いて複合核状態を陽に取り扱うことのできる反応模型を構築した.
それに加えて,類似系である\isotope[12]{C}+\isotope[13]{C}核融合反応も
同じ模型で取り扱い,核融合断面積の振る舞いの差を再現することができた.

\isotope[12]{C}+\isotope[12]{C}核融合反応における$S$-factorの共鳴ピークが
\isotope[12]{C}+\isotope[13]{C}核融合反応における$S$-factorに一致するという
実験事実に基づいて,二つの系における複合核状態への遷移強度のパラメータ
の比を決めるようなアプローチをとることにより,
断面積の振る舞いが複雑な\isotope[12]{C}+\isotope[12]{C}ではなく,
断面積の振る舞いが単純な\isotope[12]{C}+\isotope[13]{C}反応にパラメータを
合わせることで,\isotope[12]{C}+\isotope[12]{C}核融合反応にパラメータを合わせることなく
断面積を計算することができた.さらに,\isotope[12]{C}+\isotope[12]{C}核融合反応について,
実験で得られていない低エネルギー領域においては,実験で得られている
エネルギー領域よりも大きな共鳴構造を持つことがわかった.

\section*{今後の展望}

本研究では,\isotope[12]{C}+\isotope[12]{C}核融合反応における$S$-factorの共鳴ピークが
\isotope[12]{C}+\isotope[13]{C}核融合反応における$S$-factorに一致するという
実験事実を前提として断面積の計算を行ったが,
この一致がなぜ起こるのかは極めて重要な問いである.
複合核状態への遷移強度をパラメータとしてではなく,
微視的な原子核模型による計算に基づいて決めることができれば,この問いについて
明確な答えが出せるであろう.

また,本研究では\isotope{C}+\isotope{C}チャネルを基底状態同士の
反応のみと制限したが,クーロン障壁のトンネリングでは多数のチャネルへの
遷移が生じているはずである.
したがって,
炭素原子核の内部励起を模型に取り入れることにより,
より現実に即した計算へと近づくことができるであろう.