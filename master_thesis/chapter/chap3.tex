\chapter{散乱理論}
本研究において,重要なパートを担う散乱理論についてこの節で議論する.
クーロン場がない場合から始め,ある場合について述べる.
また,同種粒子同士の散乱問題についても,相互作用がスピンに依らない場合に限り,議論する.
なお,本章における議論は\cite{canto2013scattering},\cite{1970586434905966474}を参考にした.

\section{原子核間ポテンシャル}
原子核間ポテンシャルは,相対距離$r$の関数として,
\begin{align}\label{eq:general_ion_potential}
  V(r) = V_N(r) + V_C(r)
\end{align}
と書かれる.ここで,$V_N(r)$核力に由来する原子核ポテンシャルであり,$V_C(r)$がクーロンポテンシャルである.
入射粒子と標的粒子の電荷を$Z_1,Z_2$とすると,クーロンポテンシャルは,
\begin{align}\label{eq:naive_coulomb}
  V_C(r) = \frac{Z_1 Z_2 e^2}{r}
\end{align}
と書かれる.

\begin{figure}[thpb]
  \centering
  \includegraphics[width=8cm]{figure/chap3/potential_example.pdf}
  \caption{原子核間ポテンシャルの相対角運動量が0の場合の例.
  赤の点線がクーロンポテンシャルであり,青の破線が核力に由来するポテンシャルであり,
  黒の実線がその和,全ポテンシャルを表している.}
  \label{fig:potential_example}
\end{figure}

図~\ref{fig:potential_example}に式~(\ref{eq:general_ion_potential})の例をプロットしている.
点線がクーロンポテンシャルであり,破線が核力であり,その和が実線でプロットされている.
短距離力である核力と,長距離力であるクーロンポテンシャルの差分により,ポテンシャル障壁が
生じていることがわかる.このポテンシャル障壁はクーロン障壁と呼ばれ,核融合反応が生じるためには
この障壁を越えなければならない.

\section{クーロン場がないときの散乱理論}

% 短距離ポテンシャル$V_0(r)$があるときの,シュレーディンガー方程式は,
% \begin{align}\label{eq:shrodinger_eq_wo_coulomb}
%   \qty[-\frac{\hbar^2}{2\mu}\nabla^2 + V_0(r) - E] \psi(\bm{r}) = 0
% \end{align}
% のように書かれる.ポテンシャルがない場合,波数ベクトル$\bm{k}$,$\abs{\bm{k}} = \sqrt{2\mu E/ \hbar}$を用いて,式~(\ref{eq:shrodinger_eq_wo_coulomb})の解は,
% $\psi(\bm{r}) = \exp(i \bm{k} \cdot \bm{r})$と書かれる.

短距離ポテンシャル$V_0(r)$の下で,シュレーディンガー方程式は,波数ベクトル$\bm{k}$,
大きさ$k = \abs{\bm{k}} = \sqrt{2\mu E}/\hbar$を用いて,
\begin{align}\label{eq:stationary_schrodinger_eq}
  \qty(\nabla^2 + k^2) \psi(\bm{r}) = U(r) \psi(\bm{r})
\end{align}
と書ける.ここで,$\mu$は系の換算質量であり,$U(r) = \frac{2\mu}{\hbar^2}V_0(r)$である.
式~(\ref{eq:stationary_schrodinger_eq})の解は,平面波と散乱波の二つに分解できて,
\begin{align}\label{eq:desired_wave_function}
  \psi^{(+)}(\bm{k}; \bm{r}) = \phi(\bm{k};\bm{r}) + \psi^{\text{sc}}(\bm{k}; \bm{r})
\end{align}
と書かれる.ここで,
\begin{align}\label{eq:plane_wave}
  \phi(\bm{k};\bm{r}) = A e^{i \bm{k} \cdot \bm{r}}
\end{align}
および,$\psi^{(+)}(\bm{k};\bm{r})$は散乱源により生じた外向き波である.よって,$r \rightarrow \infty$において,
\begin{align}\label{eq:scatt_asympt_form}
  \psi^{\text{sc}}(\bm{k};\bm{r}) \rightarrow A f(\Omega) \frac{e^{ikr}}{r}
\end{align}
のような漸近的な振る舞いをする.
ここで,$f(\Omega) \equiv f(\theta, \varphi)$は散乱振幅であり,断面積の計算において重要な役割を担う.
式~(\ref{eq:plane_wave}),~(\ref{eq:scatt_asympt_form})より,
\begin{align}\label{eq:outgoing_asympt_form}
  \psi^{\text{(+)}}(\bm{k};\bm{r}) \rightarrow A\qty(e^{i\bm{k}\cdot \bm{r}} + f(\Omega)\frac{e^{ikr}}{r})
\end{align}
なる漸近形が得られる.
次に,この散乱振幅を用いて,微分散乱断面積を求めよう.
\begin{figure}[t]
\centering
\begin{tikzpicture}[
    >=Latex,
    beam/.style={thick,->},
    scat/.style={thick,->},
    detector/.style={thick,rectangle,draw,minimum width=1.5cm,minimum height=0.7cm},
    every node/.style={font=\small}
]

% ターゲット位置
\coordinate (T) at (0,0);

% ===== ユーザー調整用パラメータ =====
\def\thetaA{25}   % 散乱方向1
\def\thetaB{-55}   % 散乱方向2
\def\thetaC{100}  % 散乱方向3
\def\thetaD{-120}
\def\dtheta{3}    % ΔΩ の半開口角
\def\r{3}

% ===== 極薄ターゲット =====
\draw[thick] (-0.02,-0.2) rectangle (0.02,0.2);
\node[below] at (-0.2,-0.25) {ターゲット};

% ===== ビーム =====
% 入射ビーム
\draw[beam] (-4,0) -- (-0.02,0) node[midway,above] {入射ビーム};

% 透過ビーム
\draw[beam] (0.02,0) -- (4,0) node[midway,below] {透過ビーム};

% ===== 散乱ビーム1 =====


% ===== 散乱ビーム2 =====
\draw[scat] (T) -- ({2.6*cos(\thetaB)},{2.6*sin(\thetaB)}) ;
\draw[scat] (T) -- ({2.6*cos(\thetaC)},{2.6*sin(\thetaC)}) 
  node[midway,above right] {散乱粒子};
\draw[scat] (T) -- ({2.6*cos(\thetaD)},{2.6*sin(\thetaD)}) ;

% ===== ΔΩ =====
\path (T) -- ({\r*cos(\thetaA-\dtheta)},{\r*sin(\thetaA-\dtheta)}) coordinate (L);
\path (T) -- ({\r*cos(\thetaA+\dtheta)},{\r*sin(\thetaA+\dtheta)}) coordinate (R);

\fill[black!15]
  (T)--(L) arc (\thetaA-\dtheta:\thetaA+\dtheta:\r) -- cycle;
\draw[thick]
  (T)--(L) arc (\thetaA-\dtheta:\thetaA+\dtheta:\r) -- (T);

\node[anchor=east] at ({3.1*cos(\thetaA)},{3.1*sin(\thetaA)+0.2}) {$\Delta\Omega$};

% ===== 検出器 =====
\node[detector,anchor=west] at ({3.3*cos(\thetaA)},{3.3*sin(\thetaA)}) {検出器};
\draw[scat] (T) -- ({3*cos(\thetaA)},{3*sin(\thetaA)});


% ===== ビーム軸 =====
\draw[thin,dashed] (-4,-0.8) -- (4,-0.8);
\node[below] at (-3.3,-0.8) {ビーム軸};

\end{tikzpicture}
\caption{散乱実験の模式図。}
\label{fig:scattering_multi}
\end{figure}

微分散乱断面積は入射ビームのフラックスと,立体角$\Omega$に散乱された粒子のフラックスの比で与えられる.
散乱波によるフラックス$\bm{j}^{\text{sc}}$は,
\begin{align}
  \bm{j}^{\text{sc}} = \frac{\hbar}{2i\mu}\qty[(\psi^{\text{sc}}(\bm{k}; \bm{r}))^*\nabla\psi^{\text{sc}}(\bm{k}; \bm{r}) - \psi^{\text{sc}}(\bm{k};\bm{r})\nabla (\psi^{\text{sc}}(\bm{k}; \bm{r}))^*]
\end{align}
で与えられ,単位時間あたりに検出器の表面を通過する粒子数は,
\begin{align}
  \bm{\Delta S} = \Delta S \hat{\bm{r}} = \Delta \Omega r^2 \hat{\bm{r}}
\end{align}
という面積ベクトルを用いて,
\begin{align}\label{eq:count_rate_by_flux}
  N(\Omega, \Delta\Omega) = \bm{j}^{\text{sc}} \cdot \bm{\Delta S} = \Delta \Omega r^2 \hat{\bm{r}} \cdot \bm{j}^{\text{sc}}
\end{align}
で与えられる.演算子ナブラは球座標表示で,
\begin{align}
  \nabla = \hat{\bm{r}}\pdv{r} + \hat{\bm{\theta}} \frac{1}{r} \pdv{\theta} + \hat{\bm{\phi}} \frac{1}{r\sin\theta}\pdv{\phi}
\end{align}
となるため,式~(\ref{eq:count_rate_by_flux})は$r \rightarrow \infty$において,
\begin{align}
  N(\Omega, \Delta\Omega) \simeq \frac{\hbar}{2i\mu} \abs{f(\Omega)}^2 \abs{A}^2 \Delta\Omega r^2 
  \qty[\frac{e^{-ikr}}{r}\dv{r}\qty(\frac{e^{ikr}}{r}) - \frac{e^{ikr}}{r} \dv{r} \qty(\frac{e^{-ikr}}{r})]
\end{align}
となる.$r$での微分は,$e^{ikr}$にあたるものだけが残るため,
\begin{align}
  N(\Omega, \Delta\Omega) = v \abs{A}^2 \abs{f(\Omega)}^2
\end{align}
を得る.ここで,$v = \hbar k / \mu$は,運動量$\hbar k$を持つ質量$\mu$の古典粒子が持つ速さに対応する量である.
次に,入射ビームのフラックスを求める.入射ビームは平面波であったため,単位時間あたりに散乱源に入射する粒子数は,
\begin{align}
  J = \hat{\bm{z}} \cdot \bm{j}^{\text{in}} = \frac{\hbar}{2i\mu} \abs{A}^2 \qty[ e^{-ikz}\dv{z} \qty(e^{ikz}) - e^{ikz} \dv{z} \qty(e^{-ikz})] = v \abs{A}^2
\end{align}
と書かれる.
これらの比をとることで,
\begin{align}\label{eq:diff_cross_section_by_scattering_amplitude}
  \dv[]{\sigma}{\Omega} = \abs{f(\Omega)}^2
\end{align}
という関係式が得られる.また,全断面積は
\begin{align}
  \sigma = \int \dd{\Omega} \dv{\sigma}{\Omega}
\end{align}
で与えられる.

\section{部分波展開}
別のアプローチとして,部分波展開と呼ばれる方法がある.これは,ポテンシャルが球対称な場合かつ
低エネルギーの解析において極めて有効な方法である.
ハミルトニアンは運動エネルギー演算子$K \equiv -\frac{\hbar^2}{2\mu}\nabla^2$と,ポテンシャル$V$を用いて,
\begin{align}\label{eq:hamiltonian_op}
  H = K + V
\end{align}
と書こう.このときシュレーディンガー方程式は,
\begin{align}\label{eq:shcrodinger_eq_op}
  H \psi = E \psi 
\end{align}
となる.ここで,$E$は衝突エネルギーである.

まず,運動エネルギー演算子$K \equiv -\frac{\hbar^2}{2\mu}\nabla^2$を球座標$(r, \theta, \varphi)$で書き換える.
\begin{align}\label{eq:kinetic_op_in_spherical}
  K = -\frac{\hbar^2}{2\mu}\qty[\frac{1}{r^2}\pdv{r}\qty(r^2\pdv{r}) + \frac{1}{r^2\sin^2 \theta}\pdv{\theta}\qty(\sin \theta \pdv{\theta}) + \frac{1}{r^2\sin^2\theta} \pdv[2]{}{\varphi}]
\end{align}
軌道角運動量演算子$L^2$が,
\begin{align}\label{eq:angular_momentum_op}
  L^2 = -\frac{\hbar^2}{\sin^2\theta}\qty[\pdv{\theta}\qty(\sin\theta \pdv{\theta}) + \pdv[2]{}{\varphi}]
\end{align}
で書かれることから,運動エネルギー演算子は,
\begin{align}\label{eq:kinetic_op_angular_op}
  K = - \frac{\hbar^2}{2\mu} \qty[\frac{1}{r^2}\pdv{r}\qty(r^2\pdv{r})-\frac{L^2}{\hbar^2r^2}]
\end{align}
と書き直すことができる.ポテンシャルが球対称な場合には,式~(\ref{eq:kinetic_op_angular_op})から,ハミルトニアン$H$と
$L^2$および$L_z$が可換であることがわかる.
よって,式~(\ref{eq:shcrodinger_eq_op})の解は,演算子$H,L^2, L_z$の同時固有状態で書くことができ,
\begin{align}\label{eq:shrodinger_eq_eigenf}
  \psi_{klm}(\bm{r}) = R_l(k, r) Y_{l m} (\hat{\bm{r}})
\end{align}
となる.ここで,$\hat{\bm{r}}$は~($\theta, \varphi$)の略記である.
また,$Y_{lm}$は球面調和関数であり,$R_l$は動径波動関数である.散乱波動関数をこの基底で展開して,
$Y_{lm}$と内積をとることで,
動径方向のシュレーディンガー方程式
\begin{align}\label{eq:radial_eq}
  -\frac{\hbar^2}{2\mu}\qty[\frac{1}{r^2}\dv{r}\qty(r^2\dv{r})-\frac{l(l+1)}{r^2}]R_l(k, r) + V(r) R_l(k, r) = E R_l(k, r)
\end{align}
を得る.さらに,
\begin{align}\label{eq:mod_radiad_wf}
  R_l(k, r) = \frac{u_l(k, r)}{kr}
\end{align}
のように動径波動関数を変換することで,式~(\ref{eq:radial_eq})は,
\begin{align}\label{eq:mod_radial_eq}
  -\frac{\hbar^2}{2\mu}\qty[\dv[2]{}{r} - \frac{l(l+1)}{r^2}]u_l(k, r) + V(r)u_l(k, r) = E u_l(k, r)
\end{align}
となる.これは,ポテンシャルが
\begin{align}
  V_l(r) = V(r) + V_l^{\text{Cf}}(r)
\end{align}
である,一次元シュレーディンガー方程式となっており,
$V_l^{\text{Cf}}$は遠心力ポテンシャルとよばれ,
\begin{align}\label{eq:centrifugal_potential}
  V_l^{\text{Cf}}(r) = \frac{\hbar^2}{2\mu}\frac{l(l+1)}{r^2}
\end{align}
で定義される.

$\rho = k r$という無次元量を導入すると便利である.
このとき,式~(\ref{eq:mod_radial_eq})は,
\begin{align}\label{eq:mod_radial_eq_by_rho}
  \qty[\dv[2]{}{\rho} - \frac{l(l+1)}{\rho^2} + 1 ]u_l(\rho) = U(\rho) u_l(\rho)
\end{align}
となる.ここで,
\begin{align}
  U(\rho) = \frac{V(\rho/k)}{E}
\end{align}
とした.また,式~(\ref{eq:radial_eq})は,
\begin{align}\label{eq:radial_eq_by_rho}
  \qty[\dv[2]{}{\rho} + \frac{2}{\rho}\dv{\rho} + \qty(1 - \frac{l(l+1)}{\rho^2})]R_l(\rho) = U(\rho) R_l (\rho)
\end{align}
となる.

$V_0(r) = 0$の場合について考える.このとき,式~(\ref{eq:radial_eq_by_rho})および~(\ref{eq:mod_radial_eq_by_rho})の右辺は0となる.
式~(\ref{eq:radial_eq_by_rho})は球ベッセル方程式に帰着される.その解は,独立な2解$j_l(\rho)$と$n_l(\rho)$の
線形結合で表される.$j_l(\rho)$は原点で正則な球ベッセル関数であり,$n_l(\rho)$は原点で非正則な
球ノイマン関数である.詳しくは付録~(\ref{apx:bessel_eq})で述べている.それぞれの~$\rho \rightarrow \infty$での漸近形は,
\begin{align}
  j_l(\rho \rightarrow \infty) \sim \frac{\sin(\rho-l\pi/2)}{\rho} \label{eq:asympt_form_spherical_bessel_inf} \\
  n_l(\rho \rightarrow \infty) \sim \frac{\cos(\rho-l\pi/2)}{\rho} \label{eq:asympt_form_spherical_neumann_inf}
\end{align}
である.複素数の解として,
\begin{align}\label{eq:def_of_hankel_func}
  h_l^{(\pm)}(\rho) = n_l(\rho) \pm i j_l(\rho)
\end{align}
で定義される.ハンケル関数も球ベッセル方程式の解であり,それらの漸近的な振る舞いは,
外向き波と内向き波になる.つまり,
\begin{align}\label{eq:asympt_form_of_hankel_func}
  h_l^{(\pm)}(\rho) \sim \frac{e^{\pm i(\rho-l\pi/2)}}{\rho}
\end{align}
となる.

また,
\begin{align}\label{eq:ricatti_bessel_and_neumann}
  \hat{\jmath}_l(\rho) = \rho j_l(\rho) ; \quad \hat{n}_l(\rho) = \rho n_l(\rho)
\end{align}
で定義される,リッカチ・ベッセル関数~$\hat{\jmath}_l$とリッカチ・ノイマン関数~$\hat{n}_l$もよく用いられる.
これらは,$\rho \rightarrow \infty$での漸近形は,
\begin{align}
  \hat{\jmath}_l(\rho \rightarrow \infty) \sim \sin(\rho-l\pi/2) \label{eq:asympt_form_ricatti_bessel_inf} \\
  \hat{n}_l(\rho \rightarrow \infty) \sim \cos(\rho-l\pi/2) \label{eq:asympt_form_ricatti_neumann_inf}
\end{align}
であり,ハンケル関数についても同様に,
\begin{align}\label{eq:def_of_ricatti_hankel}
  \hat{h}_l^{(\pm)}(\rho) = \rho h_l^{(\pm)}(\rho) = \hat{n}_l(\rho) \pm i \hat{\jmath}_l(\rho) 
\end{align}
でリッカチ・ハンケル関数が定義され,その漸近的な振る舞いは,
\begin{align}\label{eq:asympt_form_ricatti_hankel}
  \hat{h}_l^{(\pm)}(\rho \rightarrow \infty) \sim e^{\pm i(\rho-l\pi/2)}
\end{align}
となる.

\section{位相差}

$r = R$以降では,0と近似できるポテンシャル$V(r)$について考える.
ここで,$r < R$の領域を領域Ⅰ,$r > R$の領域を領域Ⅱとする.
領域Ⅱでの動径波動関数$u_l(k, r)$は,ポテンシャルの影響がないため,
\begin{align}\label{eq:raidal_wf_in_region_2}
  u_l^{I\hspace{-1.2pt}I} (k, r) = \alpha_l \qty[\hat{\jmath}_l (kr)+\beta_l \hat{n}_l(kr)]
\end{align}
と書ける.$\alpha_l$は規格化定数であり,ポテンシャル$V(r)$の影響は,$\beta_l$に含まれる.
境界,すなわち$r=R$での対数微分を計算すると,
\begin{align}\label{eq:logarithmic_derivative_in_region_2}
  \mathcal{L}^{I\hspace{-1.2pt}I} = R \qty[\dv{u_l^{I\hspace{-1.2pt}I}(k, r)}{r}]_{r=R}/u_l^{I\hspace{-1.2pt}I}(k, R) = k R\qty[\frac{\hat{\jmath}_l^\prime (kR)+\beta_l \hat{n}^{\prime}_l(kR)}{\hat{\jmath}_l (kR)+\beta_l \hat{n}_l(kR)}]
\end{align}
となる.ここで,関数の引数の微分をプライムを用いて表している.
領域Ⅰでの動径波動関数について,境界$r=R$での対数微分を$\mathcal{L}^{I}$が得られたとすると,
その値を用いて,
\begin{align}
  \mathcal{L}^I = \mathcal{L}^{I\hspace{-1.2pt}I}
\end{align}
という接続条件から,$\beta_l$を得ることができる
\begin{align}
  \beta_l = - \frac{kR \hat{\jmath}_l^\prime(kR) - \mathcal{L}^I\hat{\jmath}_l(kR)}{kR \hat{n}_l^\prime(kR) - \mathcal{L}^{I} \hat{n}_l(kR)}.
\end{align}
ここで,$\beta_l = \tan\delta_l$と新たなパラメータ$\delta_l$を定義し,
式~(\ref{eq:asympt_form_ricatti_bessel_inf}),~(\ref{eq:asympt_form_ricatti_neumann_inf})の
漸近形を用いることで,この系の動径波動関数の漸近的な振る舞いは,
\begin{align}
  u_l(k, r \rightarrow \infty) &= \alpha'_l \qty[\sin(kr -l\pi/2) \cos\delta_l + \sin\delta_l \cos(kr-l\pi/2)] \notag \\
  &= \alpha'_l \qty[\sin(kr-l\pi/2 + \delta_l)] \label{eq:phase_shift_bessel}
\end{align}
と書くことができる.
ここで,$\alpha'_l = \alpha/ \cos\delta_l$は規格化定数である.
ポテンシャル$V$がなければ,$\delta_l = 0$となり,式~(\ref{eq:phase_shift_bessel})は,リッカチ・ベッセル関数
と同じ$kr - l\pi/2$という位相を持つ.よって,$\delta_l$は位相差と呼ばれポテンシャル$V$が生じさせる.

また,領域Ⅱでの波動関数を
\begin{align}
  u_l^{I\hspace{-1.2pt}I}(k, r) = \gamma_l \qty[\hat{h}_l^{(-)}(kr) - S_l \hat{h}_l^{(+)}(kr)]
\end{align}
とすることもあり,~$S_l = e^{2i\delta_l}$とすれば,漸近形は,式~(\ref{eq:asympt_form_ricatti_hankel})から,
\begin{align}
  u_l(k, r \rightarrow \infty) &= \frac{2 \gamma_l}{i} e^{i\delta_l}\sin(kr-l\pi/2 + \delta_l) \notag \\
  &= \gamma_l \qty[e^{-i(kr-l\pi/2)} -S_l e^{i(ir - l\pi/2)}] \label{eq:phase_shift_hankel}
\end{align}
となる.
\section{散乱振幅と断面積}
まず,平面波の部分波展開をする.ビームの方向を$z$軸ととると,
$e^{i\bm{k}\cdot \bm{r}} = e^{ikr\cos\theta}$となり,これを部分波展開すると,
ルジャンドル多項式$P_l$と,球ベッセル関数を用いて,
\begin{align}\label{eq:plane_wave_exp_by_partial_wave}
  e^{ikr\cos\theta} = \sum_l (2l+1)P_l(\cos\theta)i^l j_l(kr)
\end{align}
と書かれる.

さらに,散乱波解,式~(\ref{eq:outgoing_asympt_form})も同様に部分波展開する.
まず,式~(\ref{eq:shrodinger_eq_eigenf})を用いて,固有関数$\psi_{klm}$で展開する.
ただし,$z$軸をビーム方向に取ったことから,$m = 0$とすることができるので,
球面調和関数をルジャンドル多項式で置き換えることができ,
\begin{align}\label{eq:outgoing_base_expansion}
  \psi_{\bm{k}}^{(+)}(\bm{r}) = \sum_l C_l P_l(\cos\theta) \frac{u_l(k, r)}{kr}
\end{align}
と展開することができる.$r \rightarrow \infty$で,式~(\ref{eq:phase_shift_hankel})を用いることで,
\begin{align}\label{eq:outoing_partial_wave_expansion}
  \psi_{\bm{k}}^{(+)}(\bm{r}) \rightarrow \sum_{l=0}^{\infty} \gamma_l C_l P_l(\cos\theta) \frac{e^{-i(kr-l\pi/2)} -S_l e^{i(ir - l\pi/2)}}{kr} = \frac{e^{-ikr}}{r} X_{-} + \frac{e^{ikr}}{r} X_{+}
\end{align}
となる.ここで,
\begin{align}\label{eq:x_minus}
  X_- =\sum_l P_l(\cos\theta) \qty[\frac{\gamma_l C_l}{k} e^{il\pi/2}]
\end{align}
と
\begin{align}\label{eq:x_plus}
  X_+ = \sum_l P_l(\cos\theta) \qty[-\frac{\gamma_l C_l}{k}e^{-il\pi/2} e^{2i\delta_l}]
\end{align}
散乱振幅を位相差で書くために,式~(\ref{eq:outoing_partial_wave_expansion})を式~(\ref{eq:outgoing_asympt_form})と
比較しなければならない.
ここで,式~(\ref{eq:plane_wave_exp_by_partial_wave})を外向き波と内向き波に分解しよう.
\begin{align}
  \phi_{\bm{k}}(\bm{r}) = A \sum_l (2l+1)P_l(\cos\theta) i^l \frac{\hat{\jmath}_l(kr)}{kr}
\end{align}
$\hat{\jmath}_l(kr)$の漸近形を用いて,$\sin(kr-l\pi/2)$を指数関数で書くことにより,
\begin{align}\label{eq:plane_wave_approx_form}
  e^{ikr\cos\theta} = \frac{1}{2i} \sum_{l=0}^{\infty} (2l+1)i^l P_l(\cos\theta) \qty[\frac{e^{i(kr-l\pi/2) } - e^{-i(kr-l\pi/2)}}{kr}]
\end{align}
という漸近形が得られる.
第一項が,外向き波を表しており,第二項が内向き波に対応している.
また,式~(\ref{eq:plane_wave_approx_form})を
\begin{align}
  e^{ikr\cos\theta} \rightarrow \frac{e^{-ikr}}{r} \qty{\sum_l P_l(\cos\theta)\qty[-\frac{2l+1}{2ik}i^l e^{il\pi/2}]} + \frac{e^{ikr}}{r}\qty{\sum_l P_l(\cos\theta)\qty[\frac{2l+1}{2ik}i^le^{-il\pi/2}]}
\end{align}
と書き換えて,この式を式~(\ref{eq:outgoing_asympt_form})に代入することで,
\begin{align}\label{eq:outgoing_incoming_decomposition}
  \psi_{\bm{k}}^{(+)} \rightarrow \frac{e^{-ikr}}{r} \bar{X}_- + \frac{e^{ikr}}{r} \bar{X}_+
\end{align}
となる.ここで,
\begin{align}
  \bar{X}_- = \sum_l P_l(\cos\theta)\qty[-A\frac{2l+1}{2ik}i^l e^{il\pi/2}]
\end{align}
および,
\begin{align}
  \bar{X}_+ = A f(\theta) + A \sum_l P_l(\cos\theta) \qty[\frac{2l+1}{2ik}i^l e^{-il\pi/2}]
\end{align}
とした.ここで,
$e^{ikr}/r$と$e^{-ikr}/r$は線形独立であるために,
\begin{align}
  \bar{X}_- = X_- \quad \mathrm{and} \quad \bar{X}_+ = X_+
\end{align}
が成り立つ.係数を比較することで,
\begin{align}\label{eq:gamma_l_c_l}
  \gamma_l C_l = -\frac{A}{2i} (2l+1)i^l
\end{align}
を得る.また,これを用いることで,
\begin{align}
  f(\theta) 
  &= \frac{1}{2ik}\sum_l P_l(\cos\theta) (2l+1)\qty[S_l -1] \\
  &= \frac{1}{k} \sum_l P_l(\cos\theta) (2l+1) e^{i\delta_l} \sin\delta_l \label{eq:scattering_amplitude_partial_expansion}
\end{align}
を得る.ただし,$e^{-il\pi/2} = (-i)^l$を用いた.

弾性散乱断面積は
\begin{align}
  \dv{\sigma}{\Omega} = \abs{f(\theta)}^2
\end{align}
から得られるので,ルジャンドル多項式の直交性
\begin{align}
  \int_{-1}^{1} \dd{(\cos\theta)} P_l(\cos\theta)P_l'(\cos\theta) = \frac{2}{2l+1}\delta_{l,l'}
\end{align}
を用いることで,
\begin{align}
  \sigma_{\text{el}} 
  &= \frac{\pi}{k^2}\sum_l (2l+1)\abs{S_l-1}^2 \\
  &= \frac{4\pi}{k^2} \sum_l (2l + 1) \sin^2\delta_l \label{eq:elastic_cross_section_partial_expansion}
\end{align}
を得る.
式~(\ref{eq:elastic_cross_section_partial_expansion})から,
\begin{align}
  \mathrm{Im}\qty{f(\theta = 0)} = \frac{1}{k}\sum_l (2l+1) \sin^2 \delta_l
\end{align}
であるために,式~(\ref{eq:elastic_cross_section_partial_expansion})と比較して,
\begin{align}
  \sigma_{\text{el}} = \frac{4\pi}{k}\mathrm{Im}\qty{f(\theta = 0)}
\end{align}
という関係式が得られる.これは光学定理と呼ばれる関係式であり,入射粒子のフラックスの$\theta = 0$での減少分が,
断面積に反映されていることを表している.

\section{反応断面積}
ポテンシャルに虚部がある場合について考えよう.
このとき,ハミルトニアンは非エルミートになるために,フラックス
\begin{align*}
  \bm{j} = \frac{\hbar}{2i\mu} \qty[\psi^*(\bm{r})\nabla \psi(\bm{r}) - \psi(\bm{r})\nabla \psi^*]
\end{align*}
について,$\nabla \cdot \bm{j} = 0$が成り立たない.ハミルトニアンを
\begin{align}\label{eq:hamiltonian_with_imaginary_part}
  H = -\frac{\hbar^2}{2\mu}\nabla^2 + V(r) - iW(r)
\end{align}
とする.ここで,$W(r)$は短距離で正の値をとる関数とする.
このとき,シュレーディンガー方程式およびその複素共役は,
\begin{align}
  \qty[-\frac{\hbar^2}{2\mu}\nabla^2 + V(r) - iW(r)]\psi(\bm{r}) &= E \psi(\bm{r}) \label{eq:schrodinge_eq_with_imaginary} \\
  \qty[-\frac{\hbar^2}{2\mu}\nabla^2 + V(r) + iW(r)]\psi^*(\bm{r}) &= E \psi^*(\bm{r}) \label{eq:schrodinge_eq_with_imaginary_conjugate} 
\end{align}
$\psi^*(\bm{r})\times $式~(\ref{eq:schrodinge_eq_with_imaginary})$- \psi(\bm{r}) \times$式~(\ref{eq:schrodinge_eq_with_imaginary_conjugate})から, 
\begin{align}
  \frac{\hbar}{2i\mu}\qty[\psi^*(\bm{r})\nabla^2 \psi(\bm{r}) - \psi(\bm{r})\nabla^2\psi^*(\bm{r})] = -\frac{2}{\hbar} W(\bm{r}) \abs{\psi(\bm{r})}^2
\end{align}
左辺を変形し,
\begin{align}
  \nabla \cdot \bm{j} = -\frac{2}{\hbar}  W(\bm{r}) \abs{\psi(\bm{r})}^2
\end{align}
となる.両辺を$\bm{r}$について積分することで,
\begin{align}
  \int \dd{r} \nabla \cdot \bm{j} = \int \dd{\bm{S}} \cdot \bm{j} = - N_a = -\frac{2}{\hbar} \int \dd{r} W(\bm{r}) \abs{\psi(\bm{r})}^2
\end{align}
を得る.ここで,$N_a$は単位時間当たりの吸収される粒子数であり,弾性チャネルからほかのチャネルへ
遷移する粒子数である.
反応断面積$\sigma_a$は,単位時間当たりの反応粒子数$N_a$を入射粒子のフラックス$J = \abs{A}^2 v$で割ることで得られる.ここで,$A$は波動関数の漸近形の規格化定数である.
よって,
\begin{align}
  \sigma_a = \frac{2}{\hbar \abs{A}^2v} \int \dd{\bm{r}} W(\bm{r}) \abs{\psi(\bm{r})}^2
\end{align}
あるいは,$v$を$E,k$を用いて表すことで,
\begin{align}
  \sigma_a = \frac{k}{\abs{A}^2E} \ev{W}{\psi}
\end{align}
を得る.
$\psi$の境界条件を外向き波にすることにより,
\begin{align}\label{eq:absorption_cross_section_of_ev}
  \sigma_a = \frac{k}{\abs{A}^2E_k}\ev{W}{\psi_{\bm{k}}^{(+)}}
\end{align}
と書くことができる.

ポテンシャルの虚部があることにより,漸近形にも変化が生じる.
実部しかない場合は,$S_l$についてその位相を見るだけでよかったが,その絶対値も変化する.
弾性散乱の微分散乱断面積は,吸収がない場合も同じく,
\begin{align}
  \dv{\sigma_{el}}{\Omega} = \abs{f(\theta)}^2 = \frac{1}{4k^2} \abs{\sum_{l=0}^\infty (2l+1)(1-S_l)P_l(\cos\theta)}
\end{align}
と書け,弾性散乱断面積は,
\begin{align}
  \sigma_{el} = \frac{\pi}{k^2} \sum_{l=0}^\infty (2l+1) \abs{1-S_l}^2
\end{align}
となる.

つぎに,反応断面積を評価しよう.
式~(\ref{eq:absorption_cross_section_of_ev})から,
式~(\ref{eq:outgoing_base_expansion})と同じ展開を用いて,
\begin{align}
  \sigma_a = \frac{k}{E_k}\int \dd{r} \sum_{l=0}^\infty 4\pi(2l+1) W(r) \frac{\abs{u_l(k, r)}}{k^2} 
\end{align}
ここで,透過係数$\mathcal{T}_l$を,
\begin{align}
  \mathcal{T}_l = \frac{4k}{E_k}\int \dd{r} W(r) \abs{u_l(k, r)}^2
\end{align}
とすることで,反応断面積は,
\begin{align}
  \sigma_a = \frac{\pi}{k^2}\sum_{l=0}^{\infty} (2l+1) \mathcal{T}_l
\end{align}
となる.透過係数$\mathcal{T}_l$は$S$行列を用いて,
\begin{align}\label{eq:transmission_by_smat}
  \mathcal{T}_l = 1 - \abs{S_l}^2
\end{align}
となる.これを示すために,
\begin{align}
  \sigma_a = \frac{N_a}{J} = - \frac{1}{\abs{A}^2 v} \int \dd{\bm{S}} \cdot \bm{j} = -\frac{r^2}{\abs{A}^2 v} \int \dd{\Omega} j_r 
\end{align}
を評価しよう.
ここで,$j_r$はフラックス$\bm{j}$の動径成分であり,
\begin{align}
  j_r &= \frac{\hbar}{2i \mu} \qty[\qty(\psi_{\bm{k}}^{(+)}(\bm{r}))^* \pdv{r} \qty(\psi_{\bm{k}}^{(+)}(\bm{r})) - \qty(\psi_{\bm{k}}^{(+)}(\bm{r}))\pdv{r} \qty(\psi_{\bm{k}}^{(+)}(\bm{r}))^*] \notag\\
   &= \frac{\hbar}{\mu} \mathrm{Im}\qty[\qty(\psi_{\bm{k}}^{(+)}(\bm{r}))^*\pdv{r} \qty(\psi_{\bm{k}}(\bm{r})^{(+)})]
\end{align}
となる.
ここで,式~(\ref{eq:outoing_partial_wave_expansion})を参照すると,
残るのは,微分演算子が$e^{\pm i k r}$にあたる項だけであり,
$X_+$,$X_-$についてのクロスタームは実になることから,
\begin{align}
  j_r = \frac{v}{r^2} \qty(\abs{X_+}^2 - \abs{X_+}^2)
\end{align}
ここで,式~(\ref{eq:x_minus}),式~(\ref{eq:x_plus})と式~(\ref{eq:gamma_l_c_l})から,
\begin{align}
  X_- &=-\sum_{l=0}^{\infty} A \frac{2l+1}{2ik} (-1)^l P_l(\cos\theta) \\
  X_+ &= \sum_{l=0}^{\infty} \frac{2l+1}{2ik}(-1)^l S_l P_l (\cos\theta)
\end{align}
であることから,反応断面積は,
\begin{align}
  \sigma_a = \frac{\pi}{k^2} \sum_{l=0}^{\infty} (2l+1) \qty[1 - \abs{S_l}^2]
\end{align}
となり,式~(\ref{eq:transmission_by_smat})は示された.

実ポテンシャルのみがある場合は,$\abs{S_l}^2 = 1$となり,反応断面積は常に0となる.
また,反応断面積の最大値は$\abs{S_l}^2 = 0$となる場合に対応する.

複素ポテンシャルがある場合の光学定理について考えよう.
まず,散乱振幅$f(\theta)$について,$\abs{S_l} < 1$
の場合,
\begin{align}
  \mathrm{Im} \qty[f(\theta = 0)] = \frac{1}{2k} \sum_{l=0}^{\infty} (2l+1)P_l(1)\qty[1- \mathrm{Re}\qty(S_l)]
\end{align}
となるため,
\begin{align}
  \frac{4\pi}{k} \mathrm{Im} \qty[f(\theta = 0)] = \frac{2\pi}{k^2} \sum_{l=0}^{\infty} (2l+1)\qty[1-\mathrm{Re}\qty(S_l)]
\end{align}
となる.ここで,反応断面積と弾性散乱断面積の和を考えると,
\begin{align}
  \sigma_a + \sigma_{el} = \frac{2\pi}{k^2} \sum_{l = 0}^{\infty}(2l+1)\qty[1 - \mathrm{Re}\qty(S_l)]
\end{align}
を得る.よって,
\begin{align}
  \sigma_a + \sigma_{el} = \frac{4\pi}{k}\mathrm{Im}[f(\theta = 0)]
\end{align}
となる.
\section{複素ポテンシャルの起源}
前節では,複素ポテンシャルを形式的に導入して議論したが,
複素ポテンシャルの理論的定式化については議論しなかった.
そこで,本節では複素ポテンシャルが理論的にはどのように導入されるか
を述べる.
原子核物理学においては古くから複素ポテンシャルを用いて
原子核反応の計算が行われてきた.
ここでは,Feshbachの方法に従って複素ポテンシャルの理論的な導出を考える.
核融合反応過程では,始状態の原子核が複合核と呼ばれる中間状態を経て
$p,n,\alpha$のような粒子を放出しそれが終状態となる.
それぞれのチャネルを$P$,$Q$,$R$と呼ぶことにする.一般に$R$は$\alpha$崩壊チャネル,$p$崩壊チャネルなど,
複数のチャネルがあるが,簡単のために一つのチャンネルのみを考えることにする.
まず,ハミルトニアンを
\begin{align}\label{eq:hamiltonian_internal_ex}
  H = 
  \begin{pmatrix}
    H_{PP} & H_{PQ} & \\
    H_{QP} & H_{QQ} & H_{QR} \\
     & H_{RQ} & H_{RR}
  \end{pmatrix}
\end{align}
とする.ここで,
\begin{align}\label{eq:feshbach_shcrodinger}
  (E - H)\Psi = 0
\end{align}
というシュレーディンガー方程式について考える.
ただし,
\begin{align}
  \Psi = 
  \begin{pmatrix}
    \ket{\psi_P} \\
    \ket{\psi_Q} \\
    \ket{\psi_R}
  \end{pmatrix}
\end{align}
とした.ここで,
式~(\ref{eq:feshbach_shcrodinger})の第三成分から,
\begin{align}
  \ket{\psi_R} = \frac{1}{E-H_{RR}}H_{RQ}\ket{\psi_Q}
\end{align}
となる.また,この式と式~(\ref{eq:feshbach_shcrodinger})の第二成分から,
\begin{align}
  -H_{QP} \ket{\psi_P} + \qty(E - H_{QQ} - H_{QR} \frac{1}{E-H_{RR}}H_{RQ})\ket{\psi_Q} = 0
\end{align}
よって,式~(\ref{eq:feshbach_shcrodinger})を$P+Q$空間に射影することで,
\begin{align}
  \qty(E - H_{P+Q})\Psi_{P+Q} = 0 
\end{align}
を得る.
ただし,
\begin{align}\label{eq:effective_PQ_hamiltonian}
  H_{P+Q} =
  \begin{pmatrix}
    H_{PP} & H_{PQ} \\
    H_{QP} & H'_{QQ}
  \end{pmatrix},\quad
  \Psi_{P+Q} = 
  \begin{pmatrix}
    \ket{\psi_P} \\
    \ket{\psi_Q}
  \end{pmatrix}
\end{align}
とした.また,
\begin{align}\label{eq:effective_h_qq}
  H'_{QQ} = H_{QQ} - H_{QR}\frac{1}{E-H_{RR}}H_{RQ}
\end{align}
である.
ここで,式~(\ref{eq:effective_h_qq})について,$H_{RR}$の固有状態
\begin{align}
  H_{RR}\ket{\phi(\varepsilon)} = \varepsilon \ket{\phi(\varepsilon)}
\end{align}
で展開すると,
\begin{align}
  H'_{QQ} = H_{QQ} - H_{QR} \int \dd{\varepsilon} \frac{\ket{\phi(\varepsilon)}\bra{\phi(\varepsilon)}}{E - \varepsilon + i0^+} H_{RQ}
\end{align}
となる.ここで,
\begin{align}
  \frac{1}{E - E' + i 0^+} = \mathcal{P} \frac{1}{E-E'} - i\pi \delta(E-E')
\end{align}
という関係式を用いると,
\begin{align}
  H'_{QQ} = H_{QQ} - \Delta + i \Gamma /2
\end{align}
となる.ただし,
\begin{align}
  \Delta &= H_{QR} \mathcal{P} \int \dd{\varepsilon} \frac{\ket{\phi(\varepsilon)}\bar{\ket{\phi(\varepsilon)}}}{E - \varepsilon} H_{RQ} \\
  \Gamma &= 2 \pi H_{QR} \ket{\phi(E)}\bra{\phi(E)} H_{RQ}
\end{align}
である.
さらに,式~(\ref{eq:effective_PQ_hamiltonian})を$P$空間へと射影すると,
\begin{align}
  \qty(E - H_{PP} - H_{PQ}\frac{1}{E-H'_{QQ}}H_{QP})\ket{\phi_{P}} = 0
\end{align}
となり,有効ハミルトニアンとして,
\begin{align}
  H'_{PP} = H_{PP} + H_{PQ} \frac{1}{E-H'_{qq}}H_{QP}
\end{align}
が得られる.~$H'_{QQ}$が一般に虚部をもつことから,
有効ハミルトニアンにも虚部が現れる.これが虚数ポテンシャルの
理論的定式化であり,プラクティカルには
虚数ポテンシャルを実験データを再現するように
半経験的な式を用いることが多い.


\section{クーロン場があるときの散乱理論}

波動関数の漸近形が,式~(\ref{eq:outgoing_asympt_form})で記述されるためには,ポテンシャル$V(r)$が$1/r$よりも
$r\rightarrow\infty$で,0に近づく必要がある.
そうでないとき,入射波動関数がポテンシャルによって歪められる.
これは,荷電粒子同士の反応におけるクーロン力がその例である.
入射粒子の電荷を$q_P$,標的粒子の電荷を$q_T$とすると,二粒子間のポテンシャルは,
\begin{align}\label{eq:coulomb_pot_between_two_charged_particles}
  V_C(r) = \frac{q_P q_T}{r}
\end{align}
で書かれる.

ゾンマーフェルトパラメータと呼ばれる量を導入すると便利である:
\begin{align}\label{eq:def_of_sommerfeld_parameter}
  \eta = \frac{q_P q_T}{\hbar v}.
\end{align}
また,正面衝突における最近接距離の半分
\begin{align}
  a = \frac{q_P q_T}{2E}
\end{align}
は,ゾンマーフェルトパラメータと,
\begin{align}
  \eta = k a
\end{align}
という関係にある.

クーロン散乱の波動関数は,シュレーディンガー方程式
\begin{align}\label{eq:coulomb_scattering_schrodinger}
  - \frac{\hbar^2}{2 \mu} \nabla^2 \phi_C(\bm{k}; \bm{r}) 
  + \frac{q_P q_T}{r} \phi_C(\bm{k}; \bm{r}) = E\phi_C(\bm{k}; \bm{r})
\end{align}
を満たす.
これは,
\begin{align}\label{eq:coulomb_s_schro_rev}
  \qty[\nabla^2 + k^2- \frac{2\eta k}{r}] \phi_C(\bm{k}; \bm{r}) = 0
\end{align}
変形することができる.

クーロン力は長距離力なので,波動関数は平面波には漸近しない.この方程式を解くために,
入射ビームの方向を$z$軸にとり,
\begin{align}\label{eq:coulomb_s_initial_ansatz}
  \phi_C(\bm{k};\bm{r}) = C e^{ikz}g(r-z)
\end{align}
と置く.クーロン力を0とする($\eta \rightarrow 0$)と,$ g \rightarrow 1$となる.
極座標$(r, \theta, \varphi)$を放物線座標$(\xi, \zeta, \varphi)$へと変換する.
ここで,
\begin{align}\label{eq:def_of_parabolic_coordinates}
  \xi = r-z = r(1-\cos \theta), \quad \zeta = r + z = r(1+ \cos \theta)
\end{align}
である.

ここで,$\xi$が定数の局面は,原点を共通の焦点としての$z$の方向に開いた放物線を$z$軸回りに
回転してできる回転放物面の集まりである.
また,$\zeta$が定数の局面も同様に,$z$の負の方向に開いた回転放物面の集まりである.
これは,式~(\ref{eq:parabolic_coordinates})の形から明らかである.

波動関数を式~(\ref{eq:coulomb_s_initial_ansatz})のように置いたが,これは散乱波を含む波動関数の
漸近的な振る舞いが
\begin{align}\label{eq:asymptotic_form}
  \psi(\bm{k};\bm{r}) \rightarrow e^{i \bm{k} \cdot \bm{r}} + f(\theta) \frac{e^{ikr}}{r}
\end{align}
であり,$e^{-ikr}$の形を含んでいない.このことから$e^{ikz}$という位相因子を取り出したときに,$g(r+z)$ではなくて$g(r-z)$となることが予想されるのである.

放物線座標において,$g$の従う方程式は,
\begin{align}\label{eq:coulomb_sc_dif_eq}
  \xi g''(\xi) + (1 - ik\xi) g'(\xi) - \eta k g(\xi) = 0
\end{align}
である.詳しい導出は付録\ref{app:coulomb_diff_eq}で行っている.
この微分方程式は合流型超幾何微分方程式であり,一般に,
\begin{align}\label{eq:cofluent_hyp_geom_dif}
  z \dv[2]{F}{z} + (b - z) \dv{F}{z} - aF = 0
\end{align}
の解を,$F(a,b,z)$と書く.

よって,式~(\ref{eq:coulomb_sc_dif_eq})において,$s = ik\xi$と変数変換をし,$f(s) = g(\xi)$と置き換えることで,
\begin{align*}
  s f''(s) + (1-s) f'(s) - (- i \eta) f(s)  = 0
\end{align*}
となる.よって式~(\ref{eq:coulomb_s_schro_rev})の解は,
\begin{align}\label{eq:coulomb_sc_solution}
  \phi_C(\bm{k};\bm{r}) = C e^{ikz} F(-i\eta, 1, ik(r-z))
\end{align}
である.


% \section{合流型超幾何関数}


% 合流型超幾何関数は,式~(\ref{eq:cofluent_hyp_geom_dif}の解であり,
% ガンマ関数を用いることで,
% \begin{align}\label{eq:expansion_cofluent_hyp}
%   F(a,b,z) = \sum_{n=0}^{\infty} \frac{\Gamma(a+n)\Gamma(b)}{\Gamma(a)\Gamma(b+n)} \frac{x^n}{n!}
% \end{align}
% と表される.

% 実際に,式~(\ref{eq:cofluent_hyp_geom_dif}から,級数解を求めてみる.
% \begin{theorem}[合流型超幾何関数]
%   \begin{align}\tag{\ref{eq:cofluent_hyp_geom_dif}}
%     z \dv[2]{F}{z} + (b - z) \dv{F}{z} - aF = 0
%   \end{align}
%   の解は,
%   \begin{align}\tag{\ref{eq:expansion_cofluent_hyp}}
%     F(a,b,z) = \sum_{n=0}^{\infty} \frac{\Gamma(a+n)\Gamma(b)}{\Gamma(a)\Gamma(b+n)} \frac{x^n}{n!}
%   \end{align}
%   と書かれる.
% \end{theorem}
% \begin{proof}
%   式~(\ref{eq:cofluent_hyp_geom_dif}について,$F$を
%   \begin{align}\label{eq:cofluent_initial_guess}
%     F(z) = \sum_{n = 0}^{\infty} c_n \frac{z^n}{n!}
%   \end{align}
%   と置く.式~(\ref{eq:cofluent_initial_guess}を式~(\ref{eq:cofluent_hyp_geom_dif}に代入して,漸化式を求める.
%   まず,各項について求めると,
%   \begin{align*}
%     z \dv[2]{F}{z} &= \sum_{n = 2}^{\infty} c_n \frac{z^{n-1}}{(n-2)!} = \sum_{n=0}^{\infty} n c_{n+1} \frac{z^n}{n!} \\
%     (b - z) \dv{F}{z} &= \sum_{n=1}^{\infty} c_n (b-z) \frac{z^{n-1}}{(n-1)!} = \sum_{n=0}^{\infty} (b c_{n+1} - n c_n) \frac{z^n}{n!} \\
%     a F(z) &= \sum_{n=0}^{\infty} a c_n \frac{z^n}{n!} .
%   \end{align*}
%   よって,
%   \begin{align}\label{eq:reccurance_cofluent_coeff}
%     \sum_{n=0}^{\infty} \qty[\qty(n+b)c_{n+1} - (n + a) c_n] \frac{z^n}{n!} = 0
%   \end{align}
%   となる.式~(\ref{eq:reccurance_cofluent_coeff}が任意の$z$で成り立つことから,
%   \begin{align}\label{eq:coeff_ratio}
%     \frac{c_{n+1}}{c_n} = \frac{n+a}{n+b}
%   \end{align}
%   となるため,
%   \begin{align}\label{eq:coeff_det}
%     c_n = c_0 \prod_{k=0}^{n-1} \qty(\frac{k+a}{k+b}) = c_0 \frac{\Gamma(n + a)}{\Gamma(a)} \frac{\Gamma(b)}{\Gamma(n + b)}
%   \end{align}
%   を得る.
%   ここで,$\Gamma(n+a) = (n-1+a) \Gamma(n-1 + a) = \cdots = (n-1 + a)(n-2 + a) \cdots (1 + a) a\Gamma(a)$を用いた.

%   式,\ref{eq:coeff_det}を式~(\ref{eq:cofluent_initial_guess}に代入することで,
%   \begin{align}\label{eq:solution_cofluent_diffeq}
%     F(z) = c_0 \sum_{n=0}^{\infty}  \frac{\Gamma(a+n)\Gamma(b)}{\Gamma(a)\Gamma(b+n)} \frac{z^n}{n!}.
%   \end{align}
%   となり,式~(\ref{eq:solution_cofluent_diffeq}の係数を1としたものが,式~(\ref{eq:expansion_cofluent_hyp}となっている.
% \end{proof}




