\newpage
\section{はじめに}
原子核は陽子と中性子から成る有限量子多体系であり,
その内部には多彩な物理現象が織りなす壮大な世界が広がっている.
その中でも核融合反応は,我々の世界に輝きと構造をもたらし,
その豊かさを形作る根源的な現象の一つである.
しかしながら,その反応過程では多くの自由度が相互に結びつき,
量子多体系としての複雑さが前面に現れるため,理解は容易ではない.
特に,引力である核力ポテンシャルと斥力であるクーロンポテンシャルの
競合により生じるクーロン障壁近傍のエネルギー領域では,
核融合反応は,量子多体系におけるトンネル現象を通じてのみ生じ得るため,
その記述は極めて複雑である.
核融合反応の理論的記述においては,光学模型や結合チャネル法などを用いて研究が
行われてきた.これらは多くの実験データを再現する一方で,反応過程における
中間状態,すなわち複合核状態の構造を陽に取り込めているわけではない.

このような吸収後の過程が重要になる現象は,いくつかある.
一つは,重イオンの中性子捕獲反応である.この反応において,$(n, \mathrm{fission})$
断面積に複合核の励起スペクトルの情報がクリアに反映される\cite{MIGNECO1968603}.
これは,複合核の励起スペクトルにおいて平均準位間隔$D$と平均崩壊幅$\Gamma$の間の関係が,
$D>\Gamma$となるような励起エネルギーを持つ複合核を形成することに起因する.
しかし,重イオン核融合反応においてはクーロン障壁の高さ$V_B$と核融合反応の$Q$値が,
複合核の励起エネルギーを極めて高くし,このような関係式が成り立たない.
したがって,重い原子核同士の核融合反応において,複合核状態の構造は陽に取り入れることは必要なく,
クーロン障壁のトンネリングをもってして核融合反応が起こるという記述が良く成り立つのである.
しかしながら,複合核の構造が極めて重要な役割を担う重イオン核融合反応として,
\isotope[12]{C} + \isotope[12]{C}核融合反応が挙げられる\cite{PhysRevLett.110.072701}.

$\isotope[12]{C} + \isotope[12]{C}$核融合反応は,宇宙物理学において,巨大恒星の進化の過程にある
炭素燃焼過程,Ia型超新星爆発,X線スーパーバーストに深く関わっていると考えられているため
注目を集めている.この反応は,
\begin{align}
  \isotope[12]{C} + \isotope[12]{C} \rightarrow  \isotope[24]{Mg}^* &\rightarrow \isotope[23]{Na} + p \notag \\
                                                                  &\rightarrow \isotope[20]{Ne} + \alpha \notag
\end{align}
が主反応となっている.
現状では,この過程における反応率は,
実験で得られた断面積を低エネルギー領域へと外挿することに
よってのみ,見積もられている.
しかしながら,この外挿は$\isotope[12]{C} + \isotope[12]{C}$核融合反応の低エネルギー,重心系での運動エネルギー
$E_\text{c.m.} \lesssim 7$MeV,における共鳴構造により難しくなっている.

さらに,片方の原子核に中性子を一つ加えた形である,$\isotope[12]{C} + \isotope[13]{C}$核融合反応との違いも注目を集めている.
この系では核融合断面積に共鳴構造は見られなく,滑らかなエネルギー依存性を見せている.
それに加えて,$\isotope[12]{C} + \isotope[12]{C}$の共鳴ピークが,$\isotope[12]{C} + \isotope[13]{C}$
に一致するという振る舞いをみせることも指摘されている\cite{PhysRevC.85.014607}.

これらの振る舞いの差は,中間状態である\isotope{Mg}同位体の構造による違いからくるものである.
\isotope[12]{C} + \isotope[12]{C}反応では,\isotope[12]{C} + \isotope[13]{C}反応に比べて,
$Q$値,すなわち\isotope{Mg}との質量差が小さいこと.
\isotope[12]{C}+\isotope[12]{C}反応により生じる\isotope[24]{Mg}は偶偶核である一方,
\isotope[12]{C} + \isotope{13}{C}反応により生じる\isotope[25]{Mg}は偶奇核であるため,
同じ励起エネルギーにおいて,\isotope[24]{Mg}よりも\isotope[25]{Mg}の状態密度が大きくなる.
最後に,\isotope[12]{C} + \isotope[12]{C}反応においては,\isotope[12]{C}は基底状態が$0^{+}$
のボース粒子であるため,中間状態としてできる\isotope[24]{Mg}は偶数スピンかつ正のパリティ
を持つものに限られる.
これらの影響から,低エネルギー\isotope[12]{C}+\isotope[12]{C}核融合反応では
孤立共鳴となり共鳴構造が現れ,\isotope[12]{C}+\isotope[13]{C}核融合反応では,
重なり共鳴となり滑らかなエネルギー依存性をみせるのである.
孤立共鳴の場合は,透過係数が$P_J = 1 - \exp(-2\pi\Gamma_J/D_J)$という
Moldauer因子\cite{PhysRev.157.907}により平均的に減衰する.
ここで,$J$を複合核のスピンとし$\Gamma_J$平均崩壊幅,$D_J$は平均準位間隔とする.

しかしながら,複合核状態に焦点をあてて,かつ\isotope[12]{C} + \isotope[12]{C}
だけではなく,\isotope[12]{C}+\isotope[13]{C}核融合反応についても
同様に微視的原子核模型に基づいた統一的な取り扱いをした研究はまだ行われていない.
そこで本研究では,この二つの系を同じ枠組みで取り扱い,断面積の振る舞いの違いを再現し,
実験で得られていない低エネルギー領域においても計算可能なモデルを構築することが
目的である.
