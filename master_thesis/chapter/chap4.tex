\section{殻模型計算}

本研究では,複合核(\isotope{Mg})の励起状態をKSHELLという計算コード\cite{SHIMIZU2019372}により求める.
そこで,本章で殻模型計算について簡単にまとめる.
\subsection{Large-Scale Shell-Model}
閉殻原子核の殻模型による記述の成功に基づいて,large-scale shell-model(LSSM)と呼ばれる
手法が閉殻を超える原子核の記述に適用されるようになった.LSSMでは,原子核を不活性コアと,
活性軌道上を動く活性粒子で構成されると考える.
この活性粒子は,通常バレンス粒子,軌道はバレンス軌道として考えられる.原子核波動関数
は,スレーター行列式の重ね合わせで表現され,そのスレーター行列式は活性粒子の軌道上の占有として表現される.

\subsection{$M$-scheme}

殻模型計算では,波動関数はバレンス軌道において活性粒子をさまざまな占有によって表現される
配位の重ね合わせとして表現される.つまり,波動関数は膨大な数のスレーター行列式の
線形結合としてあらわされる.もっとも簡単な多体スレーター行列式の表現は,$M$-scheme
と呼ばれるものであり,
\begin{align}\label{eq:m_scheme_basis}
  \ket{M_i} = c_{a_{i, 1}}^{\dagger} c_{a_{i, 2}}^{\dagger} \cdots c_{a_{i, A}}^{\dagger} \ket{-}
\end{align}
とあらわされる.ここで,$A$と$\ket{-}$はそれぞれ活性核子の数と不活性コアを表している.
スレーター行列式$\ket{M_i}$は,1番目の活性粒子が$a_{i, 1}$を2番目の活性粒子が$a_{i, 2}$
をそして$A$番目の活性粒子が$a_{i, A}$という一粒子状態を占有していることを表している.
これらを並べて書いた$(a_{i,1}, a_{i,2}, \dots, a_{i,A})$は配位と呼ばれ,コンピュータで表現するのに便利である.

模型空間がすべて$M$-scheme基底によって分けられるので,殻模型の波動関数はそれらの線形結合
\begin{align}\label{eq:wf_expansion_m_scheme}
  \ket{\Psi} = \sum_{i=1}^{D_M}v_i \ket{M_i}
\end{align}
で書ける.
ここで,$M$-schemeの基底の数$D_M$は$M$-shceme次元と呼ばれる.係数$v_i$はシュレーディンガー方程式
を解くことによって得られる.$M$-scheme基底では固有値問題
\begin{align}\label{eq:schrodinger_eq_by_m_scheme}
  \sum_{j=1}^{D_M} H_{ij} v_j = E v_i
\end{align}
に帰着することができる.ここで,$H_{ij} = \mel{M_i}{H}{M_j}$はハミルトニアンの行列要素であり,
実対称行列である.

通常の殻模型計算ではハミルトニアンは1体演算子と2体演算子の二つに分けることができ,
\begin{align}\label{eq:tme_and_ome_shellmodel_h}
  H = H^{(1)} + H^{(2)} = \sum_{ac} h_{ac}^{(1)}c_a^\dagger c_c + \sum_{a<b, c<d} h_{abcd}^{(2)} c_a^\dagger c_b^\dagger c_d c_c
\end{align}
のように書かれる.ここで,$c_a^\dagger$は一粒子軌道$a$の生成演算子である.
多体ハミルトニアン行列$H_{ij}$はハミルトニアンの対称性から,ブロック対角行列となる.
$M$-scheme基底において用いられる対称性は二つあり,$z$軸回りの回転対称性と,パリティ
対称性である.回転とパリティ反転の演算子を$J_z$と$\Pi$とし,それに対応する固有値を
$M$と$\pi$とすると,$M$と$\pi$で分類されるブロック行列に分けられる.
よって,それぞれのブロック行列のみを取り扱えばよい.偶数核の場合,$M=0$のみをもつスレーター
行列式で分けられる部分空間のみ考えれば,重複なしですべての$J$状態を含むことができる.
奇数核の場合,$M = 1/2$の部分空間ですべての殻模型状態を得ることができる.
$M$-scheme次元は,そのようなブロック行列の最大次元のことを指す.

\subsection{相互作用について}
殻模型計算においては,模型空間を決めその中での相互作用(式~(\ref{eq:tme_and_ome_shellmodel_h})の係数)
を決めることにより殻模型ハミルトニアンを作る.