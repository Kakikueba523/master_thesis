\subsection{\isotope[12]{C} + \isotope[12]{C}核融合反応および,\isotope[12]{C} + \isotope[13]{C}核融合反応}

本研究では,\isotope[12]{C} + \isotope[12]{C}核融合反応だけではなく,類似系である\isotope[12]{C} + \isotope[13]{C}核融合反応も同時に
解析を行う.それは,先行研究\cite{PhysRevLett.110.072701}をリスペクトしたものである.
そこで,ここではその研究の概要を紹介しよう.

\subsection{\isotope[12]{C} + \isotope[12]{C}核融合反応}
古くは,\isotope[12]{C} + \isotope[12]{C}核融合反応の共鳴の起源は,
\begin{align}\label{eq:fusion_traditional_analisis}
  \sigma_\text{fus} = \sigma_\text{bkg} + \sigma_\text{BW}
\end{align}
のように,共鳴の寄与~($\sigma_\text{BW}$)と,非共鳴の寄与~($\sigma_\text{bkg}$)とを別に考えられていた.
しかしながら,これらの分解は人為的なものであり,証明されているわけではない.
そこで,著者らは類似系である\isotope[12]{C} + \isotope[13]{C}核融合反応
と\isotope[13]{C} + \isotope[13]{C}核融合反応との比較を通じて,
\isotope[12]{C} + \isotope[12]{C}核融合反応にのみ,$E_\text{c.m.} < 7 \text{MeV}$で
断面積の「欠損」が生じている.と解釈したものである.そして,この「欠損」の起源は,
\isotope[24]{Mg}の状態が少ないことによるものだと説明した.
また,これらの類似系と\isotope{Mg}の情報を取り入れ,統一的に取り扱った研究というのも存在しない.

彼らが行った研究を以下にまとめよう.
2011年にEsbensenら\cite{Esbensen2011}は,3つの反応系,\isotope[13]{C} + \isotope[13]{C},\isotope[12]{C} + \isotope[13]{C},\isotope[12]{C} + \isotope[12]{C}においてM3Yポテンシャルと,
原子核の非圧縮率を反映した斥力ポテンシャルとを用いて結合チャネル法で,
3つの系で同じポテンシャル,同じ結合パラメータを用いて核融合断面積を計算した.
\isotope[13]{C} + \isotope[13]{C}および\isotope[13]{C} + \isotope[12]{C}では
実験データを再現することができたが,\isotope[12]{C} + \isotope[12]{C}では$E_{\text{c.m.}} \lesssim 7 \mathrm{MeV}$において実験値の
上限を記述することしかできなかった.
\begin{figure}[t]
  \centering
  \includegraphics[width=8cm]{figure/chap2/cljiang_fig1.png}
  \caption{文献\cite{PhysRevLett.110.072701}より引用.黒線が結合チャネル法による計算.}
  \label{fig:cljiang_cc_calc_comp}
\end{figure}
そして,文献\cite{PhysRevLett.110.072701}では,
この欠損の振る舞いに対して\isotope[24]{Mg}の複合核の情報を反映し,平均的に結合チャネル法で計算された
断面積よりも減少することを示した.
そこで彼らは,\isotope[12]{C} + \isotope[12]{C}系だけに注目するのではなく,類似系にも焦点を置いて統一的な
方法で議論することが必要であると主張した.