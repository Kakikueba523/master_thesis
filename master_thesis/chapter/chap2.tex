\chapter{\isotope{C} + \isotope{C}核融合反応について}

本研究では,\isotope[12]{C} + \isotope[12]{C}核融合反応だけではなく,類似系である\isotope[12]{C} + \isotope[13]{C}核融合反応も同時に
解析を行う.それは,先行研究\cite{PhysRevLett.110.072701}をリスペクトしたものである.
そこで,ここではその研究の概要を紹介しよう.

\section{\isotope[12]{C} + \isotope[12]{C}核融合反応}
古くは,\isotope[12]{C} + \isotope[12]{C}核融合反応の共鳴の起源は,
\begin{align}\label{eq:fusion_traditional_analisis}
  \sigma_\text{fus} = \sigma_\text{bkg} + \sigma_\text{BW}
\end{align}
のように,共鳴の寄与~($\sigma_\text{BW}$)と,非共鳴の寄与~($\sigma_\text{bkg}$)とを別に考えられていた.
しかしながら,これらの分解は人為的なものであり,証明されているわけではない.
そこで,著者らは類似系である\isotope[12]{C} + \isotope[13]{C}核融合反応
と\isotope[13]{C} + \isotope[13]{C}核融合反応との比較を通じて,
\isotope[12]{C} + \isotope[12]{C}核融合反応にのみ,$E_\text{c.m.} < 7 \text{MeV}$で
断面積の「欠損」が生じている.と解釈したものである.そして,この「欠損」の起源は,
\isotope[24]{Mg}の状態が少ないことによるものだと説明した.
また,これらの類似系と\isotope{Mg}の情報を取り入れ,統一的に取り扱った研究というのも存在しない.

彼らが行った研究を以下にまとめよう.
2011年にEsbensenら\cite{Esbensen2011}は,3つの反応系,\isotope[13]{C} + \isotope[13]{C},\isotope[12]{C} + \isotope[13]{C},\isotope[12]{C} + \isotope[12]{C}においてM3Yポテンシャルと,
原子核の非圧縮率を反映した斥力ポテンシャルとを用いて結合チャネル法で,
3つの系で同じポテンシャル,同じ結合パラメータを用いて核融合断面積を計算した.
\isotope[13]{C} + \isotope[13]{C}および\isotope[13]{C} + \isotope[12]{C}では
実験データを再現することができたが,\isotope[12]{C} + \isotope[12]{C}では$E_{\text{c.m.}} \lesssim 7 \mathrm{MeV}$において実験値の
上限を記述することしかできなかった.
\begin{figure}[t]
  \centering
  \includegraphics[width=0.9\textwidth]{figure/chap2/cljiang_fig1.png}
  \caption{文献\cite{PhysRevLett.110.072701}より引用.黒線が結合チャネル法による計算.}
  \label{fig:cljiang_cc_calc_comp}
\end{figure}
そして,文献\cite{PhysRevLett.110.072701}では,
この欠損の振る舞いに対して\isotope[24]{Mg}の複合核の情報を反映し,平均的に結合チャネル法で計算された
断面積よりも減少することを示した.
そこで彼らは,\isotope[12]{C} + \isotope[12]{C}系だけに注目するのではなく,類似系にも焦点を置いて統一的な
方法で議論することが必要であると主張した.

\begin{figure}[thbp]
\centering

\begin{subfigure}{\linewidth}
\centering
% ===== 12C + 12C =====
\begin{tikzpicture}[x=1cm,y=1cm,>=Latex]

% entrance reference (arbitrary)
\def\EIn{0}

% exit relative energies (schematic; replace by actual MeV if desired)
\def\EoutAlpha{-2.3}
\def\EoutProton{-1.1}
\def\EoutNeutron{1.3}

% energy axis
\draw[->] (-1.2,-2.5) -- (-1.2,3.2) node[left] {$E$};

% title
\node[anchor=south] at (0.7,0.0) {$^{12}\mathrm{C}+^{12}\mathrm{C}$};

% entrance (left column)
\node[anchor=east] at (0,\EIn) {入口};
\draw[thick] (0,\EIn) -- (1.4,\EIn);
\node[anchor=south] at (0.7,2.2) {始状態};

% exits (right column)
\node[anchor=south] at (7.5,2.2) {終状態};

\draw[thick] (5.0,\EoutAlpha) -- (6.4,\EoutAlpha);
\node[anchor=north] at (5.7,\EoutAlpha) {$\alpha + {}^{20}\mathrm{Ne}$};
\node[anchor=west] at (6.5, \EoutAlpha) {-4.6 MeV};

\draw[thick] (6.8,\EoutProton) -- (8.2,\EoutProton);
\node[anchor=north] at (7.5,\EoutProton) {$p + {}^{23}\mathrm{Na}$};
\node[anchor=west] at (8.3, \EoutProton) {-2.2 MeV};

\draw[thick] (8.6,\EoutNeutron) -- (10.0,\EoutNeutron);
\node[anchor=north] at (9.3,\EoutNeutron) {$n + {}^{23}\mathrm{Mg}$};
\node[anchor=west] at (10.1, \EoutNeutron) {2.6 MeV};

% arrows (reaction flow)
% \draw[->] (1.4,\EIn) -- (5.0,\EoutAlpha);
% \draw[->] (1.4,\EIn) -- (5.0,\EoutProton);
% \draw[->] (1.4,\EIn) -- (5.0,\EoutNeutron);

\end{tikzpicture}
\caption{$^{12}\mathrm{C}+^{12}\mathrm{C}$}
\end{subfigure}

\vspace{1.5em}

\begin{subfigure}{\linewidth}
\centering
% ===== 12C + 13C =====
\begin{tikzpicture}[x=1cm,y=1cm,>=Latex]

% entrance reference
\def\EIn{0}

% exit relative energies (schematic, relative to entrance)
% Q_alpha ≃ +6.4 MeV, Q_p ≃ +4.3 MeV, Q_n ≃ +9.0 MeV
\def\EoutAlpha{-3.2}
\def\EoutProton{-2.1}
\def\EoutNeutron{-4.5}

% energy axis
\draw[->] (-1.2,-5.0) -- (-1.2,2.5) node[left] {$E$};

% title
\node[anchor=south] at (0.7,0.0) {$^{12}\mathrm{C}+^{13}\mathrm{C}$};

% entrance (left column)
\node[anchor=east] at (0,\EIn) {入口};
\draw[thick] (0,\EIn) -- (1.4,\EIn);
\node[anchor=south] at (0.7,2.0) {始状態};

% exits (right column)
\node[anchor=south] at (7.5,2.0) {終状態};

\draw[thick] (5.0,\EoutAlpha) -- (6.4,\EoutAlpha);
\node[anchor=north] at (5.7,\EoutAlpha) {$\alpha + {}^{21}\mathrm{Ne}$};
\node[anchor=west] at (6.5,\EoutAlpha) {-6.4 MeV};

\draw[thick] (6.8,\EoutProton) -- (8.2,\EoutProton);
\node[anchor=north] at (7.5,\EoutProton) {$p + {}^{24}\mathrm{Na}$};
\node[anchor=west] at (8.3,\EoutProton) {-4.3 MeV};

\draw[thick] (8.6,\EoutNeutron) -- (10.0,\EoutNeutron);
\node[anchor=north] at (9.3,\EoutNeutron) {$n + {}^{24}\mathrm{Mg}$};
\node[anchor=west] at (10.1,\EoutNeutron) {-9.0 MeV};

\end{tikzpicture}
\caption{$^{12}\mathrm{C}+^{13}\mathrm{C}$}
\end{subfigure}

\caption{
炭素同位体融合反応における入口および終状態のエネルギー図。
上図が\isotope[12]{C}+\isotope[12]{C}核融合反応,下図が
\isotope[12]{C}+\isotope[13]{C}核融合反応.
}
\label{fig:energy_fig_c_plus_c_fusion}
\end{figure}

ここで,炭素核融合反応系の反応経路について詳しく確認しよう.
一般に,
\begin{align}
  a + A \rightarrow b + B
\end{align}
であらわされる核反応の$Q$値は,
\begin{align}
  Q = \qty( m_a + m_A - m_b - m_B)c^2
\end{align}
で定義される.ここで,$m_a,m_A,m_b,m_B$は$a,A,b,B$の質量であり,$c$は光速である.
終状態における相対運動エネルギー$E_{\mathrm{final}}$は,始状態での運動エネルギー$E_{\mathrm{c.m.}}$を用いて,
\begin{align}
  E_{\mathrm{final}} = E_{\mathrm{c.m.}} + Q
\end{align}
で与えられるために,右辺が0より小さいときそのチャネルは閉じている,つまりその反応は起こらない.

\isotope[12]{C}+\isotope[12]{C}核融合反応では,
終状態として,主に以下のチャネルが寄与する.
\begin{align*}
  \isotope[12]{C} + \isotope[12]{C} \rightarrow \isotope[20]{Ne} + \alpha \\
  \isotope[12]{C} + \isotope[12]{C} \rightarrow \isotope[23]{Na} + p \\
  \isotope[12]{C} + \isotope[12]{C} \rightarrow \isotope[23]{Mg} + n
\end{align*}
それぞれの$Q$値は,
\begin{align*}
  Q_{\alpha} = 4.6 \mathrm{MeV} \\
  Q_p = 2.2 \mathrm{MeV} \\
  Q_n = -2.6 \mathrm{MeV}
\end{align*}
である.特に中性子放出チャネルは$Q$値が負であり,低エネルギー領域においては
ほか二つのチャネルに比べて強く抑制される.

\isotope[12]{C} + \isotope[13]{C}核融合反応では,
終状態として,主に以下のチャネルが寄与する.
\begin{align*}
    \isotope[12]{C} + \isotope[13]{C} \rightarrow \isotope[21]{Ne} + \alpha \\
  \isotope[12]{C} + \isotope[13]{C} \rightarrow \isotope[24]{Na} + p \\
  \isotope[12]{C} + \isotope[13]{C} \rightarrow \isotope[24]{Mg} + n
\end{align*}
それぞれの$Q$値は,
\begin{align*}
  Q_{\alpha} = 6.4 \mathrm{MeV} \\
  Q_p = 4.3 \mathrm{MeV} \\
  Q_n = 9.0 \mathrm{MeV}
\end{align*}
である.これらの値を見れば明らかなようにすべて正の値をとり,
低エネルギーの反応においてもすべてのチャネルが開いている.


各反応系におけるエネルギー図を図\ref{fig:energy_fig_c_plus_c_fusion}に書いた.
$E_{\mathrm{c.m.}}> 2.6$MeVでないと\isotope[12]{C}+\isotope[12]{C}核融合反応において,中性子
チャネルが開かないことがわかる.

最後に,宇宙物理学において興味のあるエネルギー領域について述べる.
宇宙における原子核反応は熱運動による運動エネルギーにより生じるため,低エネルギー反応であることが多い.
さらに,荷電粒子同士の反応であるため,その反応率は地上では再現することが極めて難しくなるほど
低くなってしまう.
クーロン障壁のトンネリング確率と,熱運動でのエネルギー分布を乗じた量はガモフピークと呼ばれ,
このエネルギー範囲における断面積が選択的に効く.温度が0.6GKのときのガモフピークを図\ref{fig:gamow_peak}
にプロットした.
\begin{figure}[t]
  \centering
  \includegraphics[width=0.9\textwidth]{figure/chap2/gamow_peak.png}
  \caption{
    $T=0.6$GKにおけるガモフピーク.赤の実線がガモフピークであり,紫のdash dotted線であらわされる
    温度$T = 0.6$GKにおける粒子のエネルギー分布であるマクスウェル分布と,クーロン透過確率を表す
    青の破線との積で表されている.
  }
  \label{fig:gamow_peak}
\end{figure}


\section{宇宙物理学における\isotope[12]{C}+\isotope[12]{C}核融合反応}

宇宙物理学において,\isotope[12]{C}+\isotope[12]{C}核融合反応が注目を集めているのは,
以下の3つの反応である.
1)Ia型超新星爆発\cite{Bravo2011}
2)巨大恒星の進化の過程\cite{10.1111/j.1365-2966.2012.20193.x}
3)X線スーパーバースト\cite{Cooper_2009}
それぞれの反応は温度が1.8GK,0.6GK,0.4GK程度と見積もられているため,それぞれのガモフピークを計算すると,
図\ref{fig:gamow_peak_three_systems}のようになる.
\begin{figure}[tb]
  \centering
  \includegraphics[width=0.9\textwidth]{figure/chap2/gamow_peak_theree.png}
  \caption{各温度におけるガモフピーク.赤が0.4GK,青が0.6GK,緑が1.8GKである.}
  \label{fig:gamow_peak_three_systems}  
\end{figure}
\subsection{Ia型超新星爆発}

古来より,時折空に新しい星が現れ,明るさを増して最大光度に達したのち,その後光は
輝きを失い目に見えなくなることが観測されてきた.このような星は新星と呼ばれた.
新星のなかには明るさの変化が極めて大きいものがあり,それらは超新星と呼ばれる.
超新星爆発にはいくつかの種類がある.
I型超新星爆発は連星系にある炭素・酸素白色矮星に物質が降着することで引き起こされる,
熱運動による原子核反応が引き起こす爆発現象である.
I型超新星爆発は,エネルギースペクトルに水素の吸収線がみられないことにより区別され,
水素の吸収線がみられるものがⅡ型超新星爆発と呼ばれる.
さらに,初期および極大光度付近のスペクトルに強いケイ素の線が存在するものは,
Ia型超新星爆発に分類される
\cite{annurev:/content/journals/10.1146/annurev.astro.38.1.191}
.
\isotope[12]{C}+\isotope[12]{C}核融合断面積の低エネルギー,
特に$E_{\text{c.m.}} < 2$ MeVという実験データのない領域での共鳴構造の詳細によって,
点火から超新星への遷移において,中心密度などへの強い影響があることが指摘された\cite{Bravo2011}.
\subsection{巨大恒星の進化の過程}
巨大恒星の進化の過程,特に質量が$M \ge 8 - 10 M_\odot $の恒星において,ヘリウム燃焼過程が終わると星の中心核には\isotope{C}と\isotope{O}
が多く含まれるようになる.
このとき,中心温度は$T \sim 0.5$GK程度である.
熱運動により二つの\isotope[12]{C}原子核が\isotope[24]{Mg}へと核融合反応を起こす.
このガモフピークは1MeV程度であり,共鳴構造によって炭素燃焼過程の寿命や中心温度が変化することが指摘された\cite{10.1111/j.1365-2966.2012.20193.x}.
\subsection{X線スーパーバースト}
中性子星に伴星から物質が降着し引き起こされる爆発現象として,
X線バーストが挙げられる.これは分のオーダーで明るさを失う.
しかし,ときに通常のX線バーストより3桁長い,数時間に及ぶ長時間のバーストを
示すことがあり,これはX線スーパーバーストと呼ばれる.
X線スーパーバーストは一般にアウタークラストと呼ばれる領域で,
\isotope[12]{C} + \isotope[12]{C}核融合反応をトリガー反応として生じると考えられている.
しかしながら,X線スーパーバーストの条件は低エネルギーでの\isotope[12]{C} + \isotope[12]{C}
核融合断面積に強く依存する\cite{Cooper_2009}.

\section{\isotope[12]{C}+\isotope[13]{C}核融合反応との比較}

\isotope[12]{C}+\isotope[12]{C}核融合反応は,前節で述べたような宇宙物理学における興味だけではなく,
その共鳴構造により原子核物理学としても古くから注目を集めている.特に,類似系である
\isotope[12]{C}+\isotope[13]{C}核融合反応との比較においてその振る舞いの差は劇的なものである.
\begin{figure}[tb]
  \centering
  \includegraphics[width=0.9\textwidth]{figure/chap2/sfac_ref_compared.png}
  \caption{
    \isotope[12]{C}+\isotope[12]{C}核融合反応と\isotope[12]{C}+\isotope[13]{C}核融合反応
    の実験データの比較.
  }
  \label{fig:exp_data_comp}
\end{figure}
図\ref{fig:exp_data_comp}に\isotope[12]{C}+\isotope[12]{C}核融合反応の$S$-factorと,\isotope[12]{C}+\isotope[13]{C}核融合反応の$S$-factorの比較
がされている.赤点は\isotope[12]{C}+\isotope[13]{C}核融合反応であり,エネルギーに対して
滑らかな依存性を持っている一方で,青点で表されている\isotope[12]{C}+\isotope[12]{C}核融合反応は,
多数の共鳴構造を持つことがわかる.
それに加えて,\isotope[12]{C}+\isotope[12]{C}核融合反応の共鳴ピークが,\isotope[12]{C}+\isotope[
  13  
]{C}核融合反応の$S$-factorの値に一致することがわかる\cite{PhysRevC.85.014607}.

\isotope[12]{C}+\isotope[12]{C}核融合反応について微視的原子核模型に基づいて
断面積を計算する試みはいくつもあるものの,
類似系である\isotope[12]{C}+\isotope[13]{C}核融合反応と同時に,
同じ枠組みで微視的原子核模型に基づいた計算は行われていない.
そこで,本研究ではこの二つの系\isotope[12]{C}+\isotope[12]{C}および,
\isotope[12]{C}+\isotope[13]{C}を
同じ枠組みでかつ微視的原子核模型に基づいた模型によって計算することを目標とする.