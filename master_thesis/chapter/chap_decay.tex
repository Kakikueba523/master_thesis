\chapter{崩壊幅の評価}\label{chap:decay_width}

複合核の崩壊幅を計算する手法はその多くが統計的な見積りによるものが大きい.
この章では最も簡単な例である,遷移状態理論について説明する.
ここでの議論は\cite{frobrich2023theory}の10.4を参照した.

\section{フェルミの黄金律を用いた評価}
複合核の崩壊について考える.エネルギー$E_C$,スピン$J$を持つ複合核状態$C$
がエネルギー$(\mathcal{E}_b,\mathcal{E}_b + \dd{\mathcal{E}}_b)$,スピン$I_y$
を持った軽い粒子$y$を放出し,残留核$B$としてスピン$I_B$が残る系を考える.
このとき,フェルミの黄金律から反応率が
\begin{align}
  \dd{R}^J_{\qty{C;J;E_C}\rightarrow \qty{b;I_y I_B;\mathcal{E}_b}} = \frac{2\pi}{\hbar} P^J_{\qty{C;J;E_C}\rightarrow \qty{b;I_y I_B;\mathcal{E}_b}} \notag \\
  \times \rho_B(E^*_C -B_b -\mathcal{E}_b; I_B) (2I_B+1)(2I_y+1)\frac{V4\pi\mu_{\beta}p_b}{(2\pi \hbar)^3}\dd{\mathcal{E}_b} \label{eq:diff_reaction_rate_for_ev}
\end{align}
であらわされる.ここで,$P_{\qty{C;J;E_C}\rightarrow \qty{b;I_y I_B;\mathcal{E}_b}}$は
複合核状態$\qty{C;J;E_C}$から蒸発チャネル$\qty{b;I_y I_b;\mathcal{E}_b}$への遷移行列の二乗である.
また,残留核$B$の準位密度$\rho_B$および,相対運動の状態密度
\begin{align}
  \rho(\mathcal{E}_b) = \frac{V4\pi p_b^2}{(2\pi\hbar)^3}\dv{p_b}{\mathcal{E}_b} = \frac{V4\pi \mu_\beta p_b}{(2\pi\hbar)^3}
\end{align}
を掛けてある.また,$B_b$は蒸発粒子$y$の結合エネルギーである.

時間反転対称性により,$T$行列について,$a$から$b$への遷移とその逆の反応について
同じ値をとるために,
\begin{align}
  P_{\qty{C;J;E_C}\rightarrow \qty{b;I_y I_B;\mathcal{E}_b}} = P_{ \qty{b;I_y I_B;\mathcal{E}_b}\rightarrow\qty{C;J;E_C}}
\end{align}
が成り立つ.$P_{ \qty{b;I_y I_B;\mathcal{E}_b}\rightarrow\qty{C;J;E_C}}$は
$y + B \rightarrow C$という核融合反応の遷移行列要素である.
核融合反応の反応率は,断面積$\sigma^J_F(b;I_y I_b; \mathcal{E}_b)$を用いて
\begin{align}
  R^J_{\qty{b;I_y I_B; \mathcal{E}_B} \rightarrow F} = \frac{v_b}{V} \sigma^J_F(b;I_y I_B;\mathcal{E}_b)
\end{align}
と書くことができ,フェルミの黄金律から,
\begin{align}
  R_{\qty{b;I_y I_B;\mathcal{E}_b}\rightarrow F} = \frac{2\pi}{\hbar} P_{ \qty{b;I_y I_B;\mathcal{E}_b}\rightarrow\qty{C;J;E_C}} (2J+1)\rho_C(E^*_C;J)
\end{align}
であるために,式~(\ref{eq:diff_reaction_rate_for_ev})は,
\begin{align}
  \dd{R}^J_{\qty{C;J;E_C}\rightarrow \qty{b;I_y I_B;\mathcal{E}_b}} = \frac{(2I_y + 1)(2I_B + 1)}{(2J+1)\rho(E^*_J; J)} \notag \\
  \times \frac{4\pi p_b^2}{(2\pi \hbar)^3}\rho_B(E^*_C -B_b - \mathcal{E}_b; I_B) \sigma^J_F(b;I_y I_B; \mathcal{E}_b)\dd{\mathcal{E}_b}
\end{align}
と変形できる.残留核のスピン$I_B$で和をとり,相対運動エネルギー$\mathcal{E}_b$で積分することで,
反応率を計算することができ,$p_b^2 = 2 \mu_\beta \mathcal{E}_b$を用いて,
\begin{align}
  R^J_{\qty{C;J;E_C}\rightarrow \qty{b;I_y I_B;\mathcal{E}_b}} = \frac{(2I_y + 1)}{(2J+1)\rho(E^*_J; J)} \frac{\mu_\beta}{\pi^2 \hbar^3} \sum_{I_B} (2I_B + 1)\notag \\
  \times \int_{0}^{E^*_C - B_b} \dd{\mathcal{E}_b }\rho_B(E^*_C -B_b - \mathcal{E}_b; I_B) \sigma^J_F(b;I_y I_B; \mathcal{E}_b) \mathcal{E}_b
\end{align}
を得る.ここで,核融合断面積$\sigma^J_F (b;I_y I_B ;\mathcal{E}_b)$について,
\begin{align}
  \sigma^J_F(b;I_y I_B; \mathcal{E}_b) = \frac{\hbar^2\pi}{2\mu_\beta \mathcal{E}_b} \sum_{S_b = \abs{I_y - I_B}}^{I_y + I_B} \sum_{l_b = \abs{J - S_b}}^{J+S_b} \frac{2J+1}{(2I_B + 1)(2I_y + 1)} T^J_{l_b S_b}
\end{align}
と書くことができるため,反応率と崩壊幅の関係式$R = \hbar \Gamma$を用いることで,崩壊幅は,
\begin{align}
  \Gamma^J_{\qty{C;J;E_C}\rightarrow \qty{b;I_y I_B;\mathcal{E}_b}} = \frac{1}{2\pi\rho(E^*_C;J)} \sum_{I_B} 
  \sum_{S_b = \abs{I_y - I_B}}^{I_y + I_B} \sum_{l_b = \abs{J - S_b}}^{J+S_b} \int_{0}^{E^*_C - B_b} \dd{\mathcal{E}_b }\rho_B(E^*_C -B_b - \mathcal{E}_b; I_B) T^J_{l_b S_b}
\end{align}
となる.透過係数$T^J_{l_b S_b}$について,古典極限をとると粒子$y$と$I_B$のポテンシャル障壁の高さ$V_B$を用いることで,
\begin{align}\label{eq:decay_width_equation}
  \Gamma^J_{\qty{C;J;E_C}\rightarrow \qty{b;I_y I_B;\mathcal{E}_b}} = \frac{1}{2\pi\rho(E^*_C;J)} \sum_{I_B} 
  \sum_{S_b = \abs{I_y - I_B}}^{I_y + I_B} \sum_{l_b = \abs{J - S_b}}^{J+S_b} \int_{V_B}^{E^*_C - B_b} \dd{\mathcal{E}_b }\rho_B(E^*_C -B_b - \mathcal{E}_b; I_B)
\end{align}
となる.

ポテンシャル障壁の見積りは,クーロン障壁$V_C$を
\begin{align}
  V_C = Z_y Z_B \frac{e^2}{R_C}, \quad R_C = 0.5 + 1.36\qty(A_y^{1/3} + A_B^{1/3})\quad \mathrm{fm}
\end{align}
として見積り,
\begin{align}
  V_b = V_C + \frac{\hbar^2}{2\mu_{\beta}}\frac{l_b(l_b+1)}{R_C^2}
\end{align}
とすることで計算した.

\section{準位密度の見積り}
式~(\ref{eq:decay_width_equation})
を見ればわかるように崩壊幅を計算する際には複合核$C$および,蒸発残留核$B$の準位密度が必要である.
準位密度の計算には半経験的な式を用いる\cite{ILJINOV1992517}.

\begin{align}
  \rho(U) = \frac{\sqrt{\pi} a^{-1/4}}{12} (U - \Delta)^{-5/4}  \exp\qty[2\sqrt{a(U-\Delta)}]
\end{align}
ここで,$a$は準位密度パラメータであり,ペアリングエネルギー
\begin{align}
  \Delta = \chi \frac{12}{\sqrt{A}}
\end{align}
である.ここで,$\chi$は偶偶核で2,偶奇核で1,奇奇核で0となる数である.
スピン依存性も考慮に入れると,
\begin{align}
  \rho(U, I) = \frac{2I + 1}{2\sqrt{2\pi}\sigma^3} \exp\qty[-\frac{(I+1/2)^2}{2\sigma^2}] \rho(U)
\end{align}
と書かれる.ここで,$\sigma$はスピンカットオフパラメータであり,原子核の質量$M$と半径$R = 1.2 A^{1/3}$を用いて,
\begin{align}
  \sigma^2 = \sqrt{\frac{U-\Delta}{a}} \frac{0.4MR^2}{\hbar^2}
\end{align}
と書かれる.

準位密度パラメータ$a$については,実験データにフィットするやり方か
半経験的な式を用いるやり方の二つがある.
本研究では,半経験的な式を用いるやり方で行う.
殻補正を考慮に入れた原子核の質量公式から,
\begin{align}
  a = \tilde{a}\qty[ 1 + \delta W_g f(U-\Delta)/(U-\Delta)]
\end{align}
ここで,
\begin{align}
  \tilde{a} = \alpha A + \beta A^{2/3} b_s
\end{align}
はフェルミガス模型による準位密度パラメータの式である.
$\beta$は定数であり,$b_s$は同じ体積をもつ球の表面積を基準にしたときの
原子核の表面積の比を表す量である.

準位密度パラメータにおけるエネルギー依存性を持たせる項
$f$は殻模型計算により見積もられ,
\begin{align}
  f(U) = 1 - \exp(-\gamma U)
\end{align}
と書かれる.

$\delta$Wは,殻補正による効果を取り入れた項である.
この式は,経験的な見積りである
\begin{align}
  \delta W_g = M_{\text{exp}} - M_{\text{LD}}
\end{align}
という式で見積もる方法がある.
本研究では,文献\cite{ILJINOV1992517}で挙げられているように,
MyersとSwiateckiによる経験的な殻補正項を用いる.
また,各パラメータ
$\alpha,\beta,\gamma$は,\cite{ILJINOV1992517}においてフィットされたデータ
である,
\begin{align*}
  \alpha = 0.114, \quad \beta = 0.098, \quad \gamma = 0.051
\end{align*}
を用いる.

これらのパラメータを用いて,具体的に\isotope[24]{Mg}
と\isotope[25]{Mg}の準位密度を計算すると,図\ref{fig:level_density_24mg_and_25mg}のようになる.
\begin{figure}[t]
  \centering
  \includegraphics[width=0.9\textwidth]{figure/chap_decay/decay_width_Mg25_and_Mg24.png}
  \caption{
    \isotope[24]{Mg}と\isotope[25]{Mg}の準位密度と実験データ\cite{ILJINOV1992517}の比較.
  }
  \label{fig:level_density_24mg_and_25mg}
\end{figure}

次に,式~(\ref{eq:decay_width_equation})を用いて,崩壊幅の見積りをしよう.
まず,\isotope[24]{Mg}の崩壊幅を見積もると図\ref{fig:decay_width_24mg_j_0}のようになる.
\begin{figure}[tb]
  \centering
  \includegraphics[width=0.8\textwidth]{figure/chap_decay/decay_width_24mg_l0.png}
  \caption{式~(\ref{eq:decay_width_equation})を用いて,$J=0$における\isotope[24]{Mg}
  の崩壊幅の見積り.}
  \label{fig:decay_width_24mg_j_0}
\end{figure}
ほしいエネルギー領域$E_{\text{c.m.}}<5$MeVにおいて,計算できていないことがわかる.
これは,第2章で述べたように\isotope[12]{C}+\isotope[12]{C}核融合反応においては,
$Q$値が小さく,終状態における準位密度の統計的な見積りが正しくできないことを反映している.
そこで,
式~(\ref{eq:decay_width_equation})を再びみると,崩壊幅は,
\begin{align}
  \Gamma = \frac{1}{2\pi \rho}\times \qty(\mathrm{Number}\quad \mathrm{of} \quad \mathrm{states})
\end{align}
のように見ることができるので,
\isotope[23]{Mg},\isotope[23]{Na},\isotope[20]{Ne}のエネルギースペクトル
をNNDC\cite{NNDC_NuDat3}から参照し,終状態の数を具体的に数えることにより崩壊幅を見積もる.
すると,$J=0$における崩壊幅は,図\ref{fig:decay_width_24mg_j_0_real}のようになる.
\begin{figure}[tb]
  \centering
  \includegraphics[width=0.8\textwidth]{figure/chap_decay/decay_width_24mg_l0_direct_count.png}
  \caption{状態数を数えることにより求めた\isotope[24]{Mg}の崩壊幅.}
  \label{fig:decay_width_24mg_j_0_real}
\end{figure}

\isotope[25]{Mg}の崩壊幅については,十分統計的な見積りが成り立つ範囲での反応になっているため,
式~(\ref{eq:decay_width_equation})を用いて見積もることで,図\ref{fig:decay_width_25mg_j_1/2_real}を得る.
\begin{figure}
  \centering
  \includegraphics[width=0.8\textwidth]{figure/chap_decay/decay_width_Mg25.png}
  \caption{
    式~(\ref{eq:decay_width_equation})を用いて,$J=1/2$における\isotope[25]{Mg}
  の崩壊幅の見積り.
  }
  \label{fig:decay_width_25mg_j_1/2_real}
\end{figure}
これらの結果を比較すると,同じ衝突エネルギーに対して崩壊幅は
一桁程度違うことがわかる.