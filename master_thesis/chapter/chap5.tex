\chapter{原子核間ポテンシャル}\label{chap:nucleus_potential}

本研究では,原子核間ポテンシャルとしてM3Y + repulsive double folding potentialを用いる.
そこで,本章ではその説明を行う.
\section{M3Y + repulsive double folding potential}
一つ目のポテンシャルとして,Esbensenら\cite{Esbensen2011}がC+C核融合反応の解析に用いた,M3Y + repulsive Double Folding potentialを用いる.
このポテンシャルは,低エネルギー重イオン核融合反応におけるhindranceを説明するために導入されたポテンシャル\cite{PhysRevC.75.034606}である.
hindranceとは$\isotope[60]{Ni} + \isotope[89]{Y}$の核融合反応において最初に観測された\cite{PhysRevLett.89.052701}現象であり,
標準的な結合チャネル計算よりも断面積が急激にさがる現象のことである.

\begin{figure}[tbp]
    \centering
    \begin{tikzpicture}[>=Stealth, thick]

        % 座標定義 (ご提示の数値を元に設定)
        \coordinate (O1) at (1,2);
        \coordinate (O2) at (6,3);
        \coordinate (n1) at (1.5, 1.4);
        \coordinate (n2) at (6.3, 2.4);

        % 原子核
        \draw (O1) circle (1.3cm);
        \draw (O2) circle (1.5cm);

        % 重心の点
        \fill (O1) circle (1.5pt);
        \fill (O2) circle (1.5pt);

        % 重心間ベクトル R
        \draw[->] (O1) -- (O2) node[midway, above] {$\bm{r}$};

        % 内部座標ベクトル r1, r2
        \draw[->, dashed] (O1) -- (n1) node[midway, left] {$\bm{r}_1$};
        \draw[->, dashed] (O2) -- (n2) node[midway, right] {$\bm{r}_2$};

        % 核子間相対ベクトル s
        \draw[->] (n1) -- (n2) node[midway, below] {$\bm{s}$};

    \end{tikzpicture}
    \caption{Double Folding Potentialの図.\\
    $\bm{s} = \bm{r} + \bm{r}_2 -\bm{r}_1$とした.}
    \label{fig:double_folding}
\end{figure}
M3Y double folding potentialは,
\begin{align}
  U_n (\bm{r}) = \int \dd{\bm{r}_1} \dd{\bm{r}_2} \rho_1(\bm{r}_1) \rho_2(\bm{r}_2) v_\mathrm{M3Y} (\bm{r} + \bm{r}_2 - \bm{r}_1)
\end{align}
のように定義される.図~\ref{fig:double_folding}にdouble foldingポテンシャルの概念図
を書いた.位置$\bm{r}_1$と$\bm{r}_2$にある,異なる原子核間にある核子同士
の相互作用から原子核間ポテンシャルを求める方法である.
ここで,$v_\mathrm{M3Y}(\bm{r})$は,Reidポテンシャルから導かれる有効核子間相互作用であり,
\begin{align}\label{eq:m3y_interaction}
  v_{\mathrm{M3Y}}(\bm{r}) = A \frac{e^{-\beta_1 r}}{\beta_1 r} + B \frac{e^{-\beta_2 r}}{\beta_2 r} + C \delta(s)
\end{align}
と書かれる.ここで,
$A = 7999$ MeV,$B = -2134$ MeV,$C = -276$ MeV fm$^{-3}$,
$\beta_1 = 4$ fm $^{-1}$,$\beta_2 = 2.5$ fm $^{-1}$である\cite{SATCHLER1979183}.
本来は
\begin{align*}
  C = (1 - 0.005\frac{E}{A}) \mathrm{MeV}\,\mathrm{fm}^{-3}
\end{align*}
と書かれるが,第二項は微小であるため無視した.

また,式~\eqref{eq:m3y_interaction}の第三項であるデルタ関数型の相互作用は,
異なる原子核の核子間における反対称化の影響を表す交換項である.

これに,デルタ関数型の核子間相互作用
\begin{align}
  v_{\mathrm{r}}(\bm{r}) = v_r \delta(\bm{r})
\end{align}
を加える.これは二つの原子核が重なることにより
高密度状態が生じる際の
非圧縮率を反映するために導入される項であり,
Double Foldingの形で書くと,
\begin{align}
  U_{\mathrm{r}}(\bm{r}) = \int \dd{\bm{r}_1} \dd{\bm{r}_2} \rho_{1\mathrm{r}}(\bm{r}_1) \rho_{2\mathrm{r}}(\bm{r}_2) v_\mathrm{r} (\bm{r} + \bm{r}_2 - \bm{r}_1)
\end{align}
となる.



本研究では,\isotope[12]{C}と\isotope[13]{C}を球形と近似して,
原子核の密度分布$\rho$はM3Y項および斥力項ともにフェルミ分布関数
\begin{align}
  \rho_i (\bm{r}) = \frac{\rho_{0i}}{1 + \exp(-\frac{r-R_{0i}}{a_i})}
\end{align}
を用いる$(i = 1,2)$.
M3Y項におけるパラメータ$R_{0i}$および$a_i$の決め方は,
\isotope[12]{C}のpoint proton荷電半径を再現するように決める.
ただし,$a_i$は\isotope[12]{C}と\isotope[13]{C}とで同じに取る.
\begin{figure}[thpb]
  \centering
  \includegraphics[width=0.8\textwidth]{figure/chap5/charge_distribution.png}
  \caption{\cite{Esbensen2011}より引用.\isotope[12]{C}のpoint proton密度分布の実験データ\cite{DEVRIES1987495}と
  $a = 0.44$fm,$R_0 = 2.155$fmとしたフェルミ分布関数のプロット図.}
  \label{fig:charge_distribution}
\end{figure}

$a = 0.44$ fmというパラメータは図~\ref{fig:charge_distribution}にあるように,
\isotope[12]{C}のpoint proton荷電分布のテール部分をよく再現する.半径パラメータについては,
$R = 2.155$ fmという値を用いた.
フェルミ分布関数において,平均二乗半径は,
\begin{align}
  \ev{r^2} = \frac{3}{5}\qty[R^2 + \frac{7}{3}(\pi a)^2]
\end{align}
と計算することができる.
これらを用いて計算できる荷電半径は,文献\cite{ANGELI2004185}から得られるpoint proton荷電半径をよく再現する.
この半径パラメータは理想的にはマター分布の半径と一致するはずなので,
\isotope[12]{C}の計算においては,このパラメータを用いてポテンシャルを計算する.

\isotope[13]{C}の計算について,
マター分布の半径は\isotope[12]{C}の半径よりもバレンス中性子1個の影響により,
大きくなるはずである.もっとも簡単な見積りとして,Woods-Saxonポテンシャルにおける
一粒子軌道の平均二乗半径において$\ev{r_n^2} = 8.655$ fm$^2$という値が,$p_{1/2}$軌道
において得られる.\isotope[12]{C}のpoint proton荷電半径,$\ev{r^2}_{12} = 5.462(9)$ fm$^2$
という値を用いて,\isotope[13]{C}のマター分布の平均二乗半径を
\begin{align}
  13\ev{r^2}_{13} = 12\qty(\ev{r^2}_{12} + \frac{\ev{r_n^2}}{13})
\end{align}
という式で見積もると,
マター分布の平均二乗半径は,2.378 fmとなる.
これを用いて,$a = 0.44$ fmから,半径パラメータは
$R = 2.228$ fmとなる.

斥力項のパラメータ$v_{\mathrm{r}}$の合わせ方は,
原子核の非圧縮率$K = 234$MeVが再現されるように選ぶ.
具体的には,
\begin{align}
  V(r) = U_{\mathrm{M3Y}}(r) + U_{\mathrm{r}}(r)
\end{align}
とする.
このとき,状態方程式について,飽和密度近傍で展開すると,
\begin{align}
  \varepsilon(\rho, \delta) = \varepsilon(\rho_0, \delta) + \frac{K}{18\rho_0^2} (\rho - \rho_0)^2
\end{align}
となる.また,$V(r)$において,$r=0$でのエネルギーの増分は
\begin{align}
  V(0) \approx 2 A_p \qty[\varepsilon(2\rho_0, \delta) - \varepsilon(\rho_0, \delta)]
\end{align}
となる.ここで,$A_p$は二つの原子核のうち少ない方の質量数である.
よって,
\begin{align}
  V(0) = \frac{K A_p}{9}
\end{align}
となるように,$v_{\mathrm{r}}$を決める.

原子核の密度分布は半径パラメータはM3Yと同じものを用いるが,diffusenessパラメータ
$a_r$はM3Y項と斥力項では別のものを用いる.
文献\cite{Esbensen2011}では,\isotope[13]{C}の半径パラメータは2.228 fmではなく,
$R = 2.28$ fmとした方が実験データをよく再現することが指摘されている.
そこで,本研究でも\isotope[13]{C}の半径パラメータは$R = 2.28$ fmの方を採用する.
\isotope[12]{C}+\isotope[12]{C}における原子核間ポテンシャルを図~\ref{fig:potential_12c_12c}
にプロットする.repulsion項がある場合は,M3Y項だけの場合に比べてポテンシャルのポケットが浅く,
ポテンシャルの障壁が厚くなっていることがわかる.
また,\isotope[12]{C}+\isotope[13]{C}との比較を図~\ref{fig:potential_comp_1212_1213}にプロットする.
ポテンシャルの概形はほとんど変わらず,障壁の位置が\isotope[12]{C}+\isotope[13]{C}
のほうがより外側へとなっていることがわかる.
\begin{figure}[t]
  \centering
  \includegraphics[width=0.8\textwidth]{figure/chap5/potential_12c_12c.png}
  \caption{\isotope[12]{C}+\isotope[12]{C}ポテンシャルの例.$a_r = 0.31$とした.
  赤線がM3Y+repulsionポテンシャルであり,青破線がM3Yポテンシャルである.}
  \label{fig:potential_12c_12c}
\end{figure}
\begin{figure}[t]
  \centering
  \includegraphics[width=0.8\textwidth]{figure/chap5/potential_comp.png}
  \caption{
    $a_r = 0.31$におけるポテンシャルの比較.青が\isotope[12]{C}+\isotope[12]{C}であり,
    赤が\isotope[12]{C}+\isotope[13]{C}である.
  }
  \label{fig:potential_comp_1212_1213}
\end{figure}

% \section{Woods-Saxon Potential}

% M3Y+repulsionポテンシャルとの比較として,
% 原子核間ポテンシャルにWoods-Saxonポテンシャルを用いて
% 計算する.
% その際,パラメータの選択として,
% Aky\"{u}z-Wintherパラメータを用いる.
% 具体的な表式を以下で与える.
% \begin{align}
%   V_N(r) = - \frac{V_0}{1 + \exp\qty[(r-R_0)/a]}
% \end{align}
% 各パラメータとして,
% \begin{align}
%   &V_0 = 16\pi \gamma \bar{R} a, \\
%   &R_0 = R_P + R_T + 0.29 \\
%   &R_i = 1.233 A^{1/3} - 0.98 A^{-1/3} \quad (i = P, T)\\
%   &\bar{R} = R_P R_T /(R_P + R_T) \\
%   &\gamma = \gamma_0\qty[1 - 1.8\qty(\frac{N_P - Z_P}{A_P})\qty(\frac{N_T - Z_T}{A_T})]
% \end{align}
% である.ここで,$a=0.63$fm,$\gamma_0 = 0.95$MeV fm$^{-2}$である.
% また,クーロン力は
% \begin{align}
%   V_C(r) =
%    \frac{Z_P Z_T}{r}
% \end{align}
% で与える.
