\newpage
\chapter{結果}

ここでは,原子核ポテンシャルごとに計算結果を載せてその内容について議論する.
\section{M3Y + Rep ポテンシャル}
M3Y + repulsionポテンシャルにおいては,Repulsion termのdifusenessパラメータ$a_r$と,
カップリング行列のパラメータ$v_0$が実験にフィットすべきパラメータである.

最初に
\isotope[12]{C}+\isotope[12]{C}と\isotope[12]{C}+\isotope[13]{C}の両方の系において
おなじ$v_0$を用いて計算を行う.
式~(\ref{eq:fus_cs_12_12})および~(\ref{eq:fus_cs_12_13})を用いて,断面積を求める.
そこで得られた断面積をmodified $S$-factorと呼ばれる量へと変換する.
\begin{align}\label{eq:modified_s_factor}
  S^*(E) = E \sigma(E) \exp(2\pi \eta + 0.46 E)
\end{align}
ここで,$\eta = 36/137 \sqrt{\mu c^2/(2 E_{\text{c.m.}})}$はゾンマーフェルトパラメータである.
式~(\ref{eq:modified_s_factor})の第一項は,運動学的な要素である,
断面積の$1/E$依存性を取り除くための項であり,
$\exp(2\pi \eta)$は,クーロン透過率の逆数であり,
$\exp(0.46E)$は原子核の大きさによる補正項である.
このことにより,$S$-factorは共鳴の寄与をあらわにあらわすことができる.

$a_r = 0.31$および,$v_0 = 0.1$としたときの$S$-factorの比較をしよう.
\begin{figure}[t]
  \centering
  \includegraphics[width=0.9\textwidth]{figure/chap7/sfac_v0_0.1_comp.png}
  \caption{
    $v_0 = 0.1$,$a_r=0.31$としたときの$S$-factorの比較.赤線が\isotope[12]{C}+
    \isotope[13]{C}核融合反応であり,青線が\isotope[12]{C}+\isotope[12]{C}核融合反応である.
  }
  \label{fig:v0_0.1_and_ar_031}
\end{figure}
図\ref{fig:v0_0.1_and_ar_031}を見ればわかるように,
青色の\isotope[12]{C}+\isotope[12]{C}核融合反応では,共鳴構造が多数現れている一方,
赤色の\isotope[12]{C}+\isotope[13]{C}核融合反応では,そのような構造はなく滑らかなエネルギー依存性を
見せていることがわかる.
これは,同じ枠組みで計算した複合核の性質,つまり準位密度と崩壊幅との関係を反映したものである.
しかしながら,\isotope[12]{C}+\isotope[12]{C}の共鳴ピークが\isotope[12]{C}+\isotope[13]{C}の値に
一致するような振る舞いは見せない.

つぎに,各$l$ごとに振る舞いの違いを見ていこう.
%横に2枚の画像(jpg/png)表示
\begin{figure}[t]
  \centering
  \begin{subfigure}[t]{0.48\linewidth}
    \centering
    \includegraphics[width=0.9\textwidth]{figure/chap7/sfac_v0_0.1_12c12c_eachl.png}
    \caption{\isotope[12]{C}+\isotope[12]{C}}
    \label{fig:v001_ar031_12c_12c_eachl}
  \end{subfigure}\hfill
  \begin{subfigure}[t]{0.48\linewidth}
    \centering
    \includegraphics[width=0.9\textwidth]{figure/chap7/sfac_v0_0.1_12c13c_eachl.png}
    \caption{\isotope[12]{C}+\isotope[13]{C}}
    \label{fig:v001_ar031_12c_13c_eachl}
  \end{subfigure}
  \caption{$v_0 = 0.1,a_r = 0.31$としたときの,$S$-factorの各$l$における比較.左図が\isotope[12]{C}+\isotope[12]{C}
  核融合反応であり,右図が\isotope[12]{C}+\isotope[13]{C}核融合反応である.}
  \label{fig:v001_ar031_c_c_eachl}
\end{figure}
図\ref{fig:v001_ar031_12c_12c_eachl}において,$E_{\mathrm{c.m.}} \approx 4.8$MeVにおいて,
大きなピークがみられることがわかる.これは,$l=0$の原子核間ポテンシャルにおける,位相差から求めることができる
ポテンシャル共鳴のエネルギーの値と一致しており,このエネルギーの値においてクーロン障壁のトンネリング確率が
大きくなるため,$S$-factorにおいてもおおきな値を持つことがわかる.

次に,$a_r$は固定したまま,$v_0$を動かしていく.
%横に2枚の画像(jpg/png)表示
\begin{figure}[tb]
\begin{center}
\begin{tabular}{c}
\begin{minipage}{0.5\hsize}
\begin{center}
\includegraphics[width=0.9\textwidth]{figure/chap7/sfac_v0_0.3_comp.png}
\end{center}
\caption{
    $v_0 = 0.3$,$a_r = 0.31$としたときの,$S$-factorの図.
  }
\label{fig:v003_ar031}
\end{minipage}
\begin{minipage}{0.5\hsize}
\begin{center}
\includegraphics[width=0.9\textwidth]{figure/chap7/sfac_v0_0.6_comp.png}
\end{center}
\caption{
    $v_0 = 0.6$,$a_r = 0.31$としたときの,$S$-factorの図.
  }
\label{fig:v006_ar031}
\end{minipage}
\end{tabular}
\end{center}
\end{figure}

図\ref{fig:v003_ar031}は$v_0 = 0.3$とした$S$-factorであり,
図\ref{fig:v006_ar031}は$v_0 = 0.6$とした$S$-factorである.
$v_0$の値によらず,共鳴構造の発現には変化がないものの,
全体の構造は変化することがわかる.

次に,\isotope[12]{C}+\isotope[13]{C}
核融合反応の$S$-factorが,
\isotope[12]{C}+\isotope[12]{C}核融合反応の$S$-factor
の上限となり,共鳴のピークで一致するという振る舞いを再現しよう.
これを見るために,核融合チャネルへの遷移の効果がどのように効いてくるのかを確認する.
まず,全体のハミルトニアン
\begin{align}
  \bm{H}^{l,J^\pi} = 
  \begin{pmatrix}
    \bm{H}^l_\text{C+C} & \bm{V} \\
    \bm{V}^T & \bm{H}^{J^\pi}_\text{CN}
  \end{pmatrix} \tag{\ref{eq:total_hamiltonian}}
\end{align}
について,\isotope{C}+\isotope{C}チャネルに射影する.
このとき,有効ハミルトニアンとして,
\begin{align}
  \bm{H}_{\mathrm{C+C,eff}}^{l, J^\pi} = \bm{H}^l_\text{C+C} + \bm{V}^T \frac{1}{E-\bm{H}^{J^\pi}_\text{CN}} \bm{V}
\end{align}
が得られる.
このハミルトニアンの虚部は,
\begin{align}\label{eq:effctive_imaginary_part}
  \mathrm{Im} \qty(\bm{H}_{\mathrm{C+C,eff}}^{l, J^\pi}) = \mathrm{Im}\sum_\mu \frac{v_0^2 \delta_{i,i_e}}{E - \mathcal{E}_\mu + i\Gamma_\mu /2}
\end{align}
となる.吸収断面積が,
\begin{align}
  \sigma_{r} \propto \int \dd{r} \mathrm{Im} \qty(\bm{H}_{\mathrm{C+C,eff}}^{l, J^\pi}) \abs{u(r)}^2
\end{align}
と書かれるために,この虚部の値を\isotope[12]{C}+\isotope[12]{C}の共鳴エネルギー直上と,
\isotope[12]{C}+\isotope[13]{C}とで,一致させるように$v_0$の比を決める.

\isotope[12]{C}+\isotope[12]{C}の場合.
$E \sim E_C$を考える.ただし,$E_C$は複合核の固有エネルギーとする.
孤立共鳴に対応するため,式~(\ref{eq:effctive_imaginary_part})は
\begin{align}\label{eq:12_12_imaginary}
  \mathrm{Im} \qty(\bm{H}_{\mathrm{C+C,eff}}^{l, J^\pi}) = - i \frac{2v_0^2}{\Gamma_C} \delta_{i, i_e}
\end{align}
となる.

\isotope[12]{C}+\isotope[13]{C}の場合.
$E \sim E_C$を考える.ただし,$E_C$は複合核の固有エネルギーである.
準位間隔$D$と崩壊幅$\Gamma$を一定だと近似すると,
\begin{align*}
  \sum_\mu \frac{v_0^2}{-n D + i \Gamma/2} = -iv_0^2 \frac{\pi}{D}\coth \qty(\frac{\pi \Gamma}{2D})
\end{align*}
\begin{align}
  \mathrm{Im} \qty(\bm{H}_{\mathrm{C+C,eff}}^{l, J^\pi}) = - v_0 ^2 \frac{\pi}{D} \coth \qty(\frac{\pi \Gamma}{2D})
\end{align}
であるため,$D \ll \Gamma$のもとで,
\begin{align}
  \mathrm{Im} \qty(\bm{H}_{\mathrm{C+C,eff}}^{l, J^\pi}) = - v_0^2 \pi \rho(E_C)
\end{align}
である.
よって,有効ポテンシャルの虚部の比が1となるようにすると,
\begin{align}\label{eq:strength_ratio_equation}
  (v_0^{12+12})^2 / (v_0^{12+13})^2 = \frac{\Gamma^{24}_C}{2} \pi \rho^{25}(E_C)
\end{align}
と決めることができる.ただし,$\Gamma^{24}_C$は\isotope[24]{Mg}の$E \sim E_C$における崩壊幅であり,
$\rho^{25}(E_C)$は\isotope[25]{Mg}の$E \sim E_C$における準位密度である.

\begin{figure}[tb]
  \centering
  \includegraphics[width=0.9\textwidth]{figure/chap2/sfac_ref_compared.png}
  \caption{図\ref{fig:exp_data_comp}再掲}
\end{figure}

今までの計算結果から,\isotope[12]{C}+\isotope[12]{C}で最も寄与する部分波は,
$l = 0,2$であり,\isotope[12]{C}+\isotope[13]{C}で最も寄与する部分波は,
$l = 0,1,2$である.よって,これらの値における平均の値を比率の計算,式~(\ref{eq:strength_ratio_equation})
に用いる.すなわち,\isotope[24]{Mg}の$J = 0,2$における平均崩壊幅と,
\isotope[25]{Mg}の$J = 1/2, 3/2, 5/2$における平均準位間隔を用いて計算する.

殻模型による計算と,遷移状態理論による見積りから,
$\Gamma^{24} \sim 0.031$MeVであり,$\rho^{25} \sim 1/ 0.006$であるため,
これらの値をもとに計算する.

まず$v_0= 0.1,0.3,0.6$を
\isotope[12]{C}+\isotope[13]{C}核融合反応に用いて,
\isotope[12]{C}+\isotope[12]{C}核融合反応には式~(\ref{eq:strength_ratio_equation})
を用いて変換した値を用いる.具体的には,
$v_0 = 0.28,0.85, 1.7$を用いて計算し,同じ図にプロットする.
\begin{figure}[t]
  \centering
  \includegraphics[width= 0.9\textwidth]{figure/chap7/sfac_v0_0.28_comp.png}
  \caption{
    $v_0^{12+12} = 0.28$,$v_0^{12+13} = 0.1$,$a_r = 0.31$としたときの$S$-factorの比較.
  }
  \label{fig:v0_1212_0.28}
\end{figure}
%横に2枚の画像(jpg/png)表示
\begin{figure}[tb]
\begin{center}
\begin{tabular}{c}
\begin{minipage}{0.5\hsize}
\begin{center}
\includegraphics[width=8cm]{figure/chap7/sfac_v0_0.85_comp.png}
\end{center}
\caption{$v_0^{12+12} = 0.85$,$v_0^{12+13} = 0.3$,$a_r = 0.31$としたときの$S$-factorの比較}
\label{fig:v0_1212_0.85}
\end{minipage}
\begin{minipage}{0.5\hsize}
\begin{center}
\includegraphics[width=8cm]{figure/chap7/sfac_v0_1.7_comp.png}
\end{center}
\caption{$v_0^{12+12} = 1.7$,$v_0^{12+13} = 0.6$,$a_r = 0.31$としたときの$S$-factorの比較}
\label{fig:v0_1212_1.7}
\end{minipage}
\end{tabular}
\end{center}
\end{figure}

図\ref{fig:v0_1212_0.28},\ref{fig:v0_1212_0.85},\ref{fig:v0_1212_1.7}を見れば,
広いエネルギー範囲において\isotope[12]{
  C
}+\isotope[12]{C}核融合反応の$S$-factorの共鳴ピークが,\isotope[12]{C}+\isotope[13]{C}
核融合反応の$S$-factorと一致していることがわかる.

以下では,\isotope[12]{C}+\isotope[12]{C}反応における$v_0$と\isotope[12]{C}+\isotope[13]{C}
反応における$v_0$の比率を,式~(\ref{eq:strength_ratio_equation})を用いて決定する.

次に,パラメータ$a_r$を決める.このパラメータの値は\isotope[12]{C}+\isotope[13]{C}核融合反応の$S$-factor
の値を実験と合うように決める.
$a_r = 0.31$から$a_r = 0.36$まで,0.01刻みで変化させると,

\begin{figure}[tb]
  \centering
  \includegraphics[width=0.9\textwidth]{figure/chap7/m3y_comp_plt_v0_0.1_ar_changes.png}
  \caption{
    $v_0 = 0.1$を固定し,$a_r = 0.31$から$a_r = 0.36$まで0.01刻みで変化させたときの
    \isotope[12]{C}+\isotope[13]{C}核融合反応の$S$-factorの変化.
  }
\end{figure}
%横に2枚の画像(jpg/png)表示
\begin{figure}[tb]
\begin{center}
\begin{tabular}{c}
\begin{minipage}{0.5\hsize}
\begin{center}
\includegraphics[width=8cm]{figure/chap7/m3y_comp_plt_v0_0.3_ar_changes.png}
\end{center}
\caption{$v_0 = 0.3$を固定し,$a_r = 0.31$から$a_r = 0.36$まで0.01刻みで変化させたときの
    \isotope[12]{C}+\isotope[13]{C}核融合反応の$S$-factorの変化.}
\end{minipage}
\begin{minipage}{0.5\hsize}
\begin{center}
\includegraphics[width=8cm]{figure/chap7/m3y_comp_plt_v0_0.6_ar_changes.png}
\end{center}
\caption{$v_0 = 0.6$を固定し,$a_r = 0.31$から$a_r = 0.36$まで0.01刻みで変化させたときの
    \isotope[12]{C}+\isotope[13]{C}核融合反応の$S$-factorの変化.}
\end{minipage}
\end{tabular}
\end{center}
\end{figure}

実験データとともにプロットした図を図~\ref{fig:exp_comp_v001_ar_changes}にプロットする.

\begin{figure}[tb]
  \centering
  \includegraphics[width=0.8\textwidth]{figure/chap7/m3y_exp_comp_plt_v0_0.1_ar_changes.png}
  \caption{
    $v_0 = 0.3$を固定し,$a_r = 0.31$から$a_r = 0.36$まで0.01刻みで変化させたときの
    \isotope[12]{C}+\isotope[13]{C}核融合反応の$S$-factorの変化.実験データを赤点でプロットしてある.
  }
  \label{fig:exp_comp_v001_ar_changes}
\end{figure}

$v_0 = 0.3$あたりの概形が実験データの概形と一致するため,
次に$v_0$を変化させながら実験に合わせる.
\begin{figure}[tb]
  \centering
  \includegraphics[width=0.8\textwidth]{figure/chap7/m3y_exp_comp_plt_v0_0.2_ar_changes.png}
  \caption{
      $v_0 = 0.2$を固定し,$a_r = 0.31$から$a_r = 0.36$まで0.01刻みで変化させたときの
    \isotope[12]{C}+\isotope[13]{C}核融合反応の$S$-factorの変化.実験データを赤点でプロットしてある.
  }
  \label{fig:exp_comp_v002_ar_changes}
\end{figure}

グラフの概形から,図\ref{fig:exp_comp_v002_ar_changes}から,$v_0 = 0.2$,$a_r = 0.33$が実験データをよく再現するパラメータとなる.
そこで,式~(\ref{eq:strength_ratio_equation})から\isotope[12]{C}+\isotope[12]{C}系に
おける$v_0$の値を求めると,
$v_0 = 0.57$となるため,二つを同じ図にプロットすると,
\begin{figure}[tb]
  \centering
  \includegraphics[width = 0.8\textwidth]{figure/chap7/m3y_exp_comp_plt_v0_0.2_ar_3.png}
  \caption{
    \isotope[12]{C}+\isotope[12]{C}系では$v_0 = 0.57$,\isotope[12]{C}+\isotope[13]{C}系では,
    $v_0=0.2$を用いて計算した$S$-factor.青線が\isotope[12]{C}+\isotope[12]{C}核融合反応であり,赤線が
    \isotope[12]{C}+\isotope[13]{C}核融合反応に対応している.
  }
  \label{fig:m3y_rep_final_result}
\end{figure}
図\ref{fig:m3y_rep_final_result}のようになる.
各部分波における寄与も図~\ref{fig:m3y_rep_final_result_12_12_eachl}と図~\ref{fig:m3y_rep_final_result_12_13_eachl}に
プロットした.
%横に2枚の画像(jpg/png)表示
\begin{figure}[tb]
\begin{center}
\begin{tabular}{c}
\begin{minipage}{0.5\hsize}
\begin{center}
\includegraphics[width=8cm]{figure/chap7/m3y_exp_comp_plt_v0_0.2_ar_3_eachl.png}
\end{center}
\caption{図\ref{fig:m3y_rep_final_result}における\isotope[12]{C}+\isotope[13]{C}核融合反応の$S$-factorの各部分波による寄与.}
\label{fig:m3y_rep_final_result_12_12_eachl}
\end{minipage}
\begin{minipage}{0.5\hsize}
\begin{center}
\includegraphics[width=8cm]{figure/chap7/m3y_exp_comp_plt_v0_0.57_ar_3_eachl.png}
\end{center}
\caption{図\ref{fig:m3y_rep_final_result}における\isotope[12]{C}+\isotope[12]{C}核融合反応の$S$-factorの各部分波による寄与.}
\label{fig:m3y_rep_final_result_12_13_eachl}
\end{minipage}
\end{tabular}
\end{center}
\end{figure}

図~\ref{fig:m3y_rep_final_result}を見ると,\isotope[12]{C}+\isotope[12]{C}核融合反応の$S$-factorでは,
実験で得られたエネルギー領域においては同じ程度の大きさの共鳴構造を再現できていることがわかる.
それに加えて,実験で得られていないさらに低エネルギー側においては,
共鳴構造がさらに大きくなっていることがわかる.

% \section{Woods-Saxonポテンシャルの場合}
% 次に,原子核間ポテンシャルをWoods-Saxonポテンシャルとして計算した結果を表す.
% まず,$v_0$の値を図\ref{fig:m3y_rep_final_result}と同じ値を用いて計算した
% ものを図~\ref{fig:woods_pot_1st_edi}にプロットする.
% \begin{figure}[tb]
%   \centering
%   \includegraphics[width=0.8\textwidth]{figure/chap7/aw_exp_comp_25_24.png}
%   \caption{
%     Woods-Saxonポテンシャルを用いて計算した$S$-factorの比較.
%     ただし,$v_0$は図~\ref{fig:m3y_rep_final_result}と同じものを用いて計算した.
%   }
%   \label{fig:woods_pot_1st_edi}
% \end{figure}
