\section{結果}
ここでは,原子核ポテンシャルごとに計算結果を載せてその内容について議論する.
\subsection{M3Y + Rep ポテンシャル}
M3Y + repulsionポテンシャルにおいては,Repulsion termのdifusenessパラメータ$a_r$と,
カップリング行列のパラメータ$v_0$が実験にフィットすべきパラメータである.
最初に
\isotope[12]{C}+\isotope[12]{C}と\isotope[12]{C}+\isotope[13]{C}の両方の系において
おなじ$v_0$を用いて計算を行う.
式~(\ref{eq:fus_cs_12_12})および~(\ref{eq:fus_cs_12_13})を用いて,断面積を求める.
そこで得られた断面積をS-factorと呼ばれる量へと変換する.
\begin{align}
  S^*(E) = E \sigma(E) \exp(2\pi \eta + 0.46 E)
\end{align}
ここで,$\eta = 36/137 \sqrt{\mu c^2/(2 E_{\text{c.m.}})}$はゾンマーフェルトパラメータである.これは,断面積から自明なエネルギー依存性を取り除いたものになっており,
クーロン障壁以下のエネルギーにおける核融合反応においてよく用いられる.

