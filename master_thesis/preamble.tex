\usepackage{amsmath,amsfonts}
\usepackage{amssymb}
\usepackage{bm}
\usepackage{physics}
% 画像
\usepackage{graphics}
\usepackage{graphicx}
\usepackage{here} %画像の表示位置調整用
\usepackage{subcaption}
\usepackage{type1cm}
\usepackage{hyperref}
\usepackage{isotope}
\usepackage{tikz}
\usetikzlibrary{decorations.markings, arrows.meta}
\usetikzlibrary{snakes, calc}

\usepackage[style=phys,articletitle=false,biblabel=brackets,chaptertitle=false,pageranges=false]{biblatex}
\addbibresource{main.bib}
\usepackage{appendix}
%A4: 21.0 x 29.7cm

\numberwithin{figure}{chapter}
\numberwithin{equation}{chapter}
\numberwithin{table}{chapter}

\newtheorem{theorem}{定理}[chapter]   % sectionごとに番号リセット
\newtheorem{lemma}[theorem]{補題}      % theoremと同じカウンタを共有
\newtheorem{proposition}[theorem]{命題}
\newtheorem{corollary}[theorem]{系}


% \makeatletter
% \g@addto@macro\appendix{%
%   % セクション番号を A, B, C... にする(普通は自動だけど念のため)
%   \renewcommand{\thesection}{\Alph{section}}
%   % 式番号を A.1 の形式にする
%   \renewcommand{\theequation}{\thesection.\arabic{equation}}
% }
% \makeatother