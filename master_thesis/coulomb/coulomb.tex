\documentclass[a4paper,11pt]{ltjsarticle}


% 数式
\usepackage{amsmath,amsfonts}
\usepackage{amssymb}
\usepackage{bm}
\usepackage{physics}
% 画像
\usepackage{graphics}
\usepackage{graphicx}
\usepackage{here} %画像の表示位置調整用
\usepackage{type1cm}
\usepackage{hyperref}
\usepackage{isotope}
\usepackage[style=phys,articletitle=false,biblabel=brackets,chaptertitle=false,pageranges=false]{biblatex}
\addbibresource{coulomb.bib}
%A4: 21.0 x 29.7cm


\begin{document}

\title{test}
\author{長尾 昂青}
\date{\today}
\maketitle

\tableofcontents

\newpage
\section{クーロン散乱の量子論}

主に\cite{canto2013scattering}や\cite{BN00026961}を参照されたい.

クーロン散乱の波動関数は,シュレーディンガー方程式
\begin{align}\label{eq:coulomb_scattering_schrodinger}
  - \frac{\hbar^2}{2 \mu} \nabla^2 \phi_C(\bm{k}; \bm{r}) 
  + \frac{1}{4 \pi \varepsilon_0} \frac{q_P q_T}{r} \phi_C(\bm{k}; \bm{r}) = E\phi_C(\bm{k}; \bm{r})
\end{align}
を満たす.
これは,
\begin{align}\label{eq:coulomb_s_schro_rev}
  \qty[\nabla^2 + k^2- \frac{2\eta k}{r}] \phi_C(\bm{k}; \bm{r}) = 0
\end{align}
変形することができる.
ここで,$\eta$はゾンマーフェルトパラメータであり,
\begin{align}
  \eta = \frac{1}{4 \pi \varepsilon_0} \frac{q_P q_T}{\hbar v}
\end{align}
である.また,$q_T, q_P$はそれぞれ,標的核と入射核の電荷であり,$k = \sqrt{2\mu E}/\hbar$,$\mu$は換算質量である.

クーロン力は長距離力なので,波動関数は平面波には漸近しない.この方程式を解くために,
入射ビームの方向を$z$軸にとり,
\begin{align}\label{eq:coulomb_s_initial_ansatz}
  \phi_C(\bm{k};\bm{r}) = C e^{ikz}g(r-z)
\end{align}
と置く.クーロン力を0とする($\eta \rightarrow 0$)と,$ g \rightarrow 1$となる.
極座標$(r, \theta, \varphi)$を放物線座標$(\xi, \zeta, \varphi)$へと変換する.
ここで,
\begin{align}\label{eq:parabolic_coordinates}
  \xi = r-z = r(1-\cos \theta), \quad \zeta = r + z = r(1+ \cos \theta)
\end{align}
である.

ここで,$\xi$が定数の局面は,原点を共通の焦点としての$z$の方向に開いた放物線を$z$軸回りに
回転してできる回転放物面の集まりである.
また,$\zeta$が定数の局面も同様に,$z$の負の方向に開いた回転放物面の集まりである.
これは,式\ref{eq:parabolic_coordinates}の形から明らかである.

波動関数を式\ref{eq:coulomb_s_initial_ansatz}のように置いたが,これは散乱波を含む波動関数の
漸近的な振る舞いが
\begin{align}\label{eq:asymptotic_form}
  \psi(\bm{k};\bm{r}) \rightarrow e^{i \bm{k} \cdot \bm{r}} + f(\theta) \frac{e^{ikr}}{r}
\end{align}
であり,$e^{-ikr}$の形を含んでいない.このことから$e^{ikz}$という位相因子を取り出したときに,$g(r+z)$ではなくて$g(r-z)$となることが予想されるのである.


放物線座標でのラプラシアンを求める.これには,極座標のラプラシアンから出発するのがよいだろう.
\begin{align}\label{eq:laplacian_polar}
  \nabla^2 = \frac{1}{r^2} \pdv{r}\qty(r^2\pdv{r}) + \frac{1}{r^2 \sin \theta} 
  \pdv{\theta}\qty(\sin \theta \pdv{\theta}) + \frac{1}{r^2 \sin^2 \theta} \pdv[2]{\varphi}
\end{align}
式\ref{eq:parabolic_coordinates}を参照しながら,式\ref{eq:laplacian_polar}に出てくる各項を求める.
\begin{align*}
  r = \frac{\xi + \zeta}{2}, \quad r^2\sin^2\theta = \xi \zeta
\end{align*}
であり,偏微分項は,
\begin{align*}
  \pdv{r} = \pdv{\xi}{r} \pdv{\xi} + \pdv{\zeta}{r} \pdv{\zeta} = \qty(1- \cos \theta) \pdv{\xi} + \qty(1 + \cos \theta) \pdv{\zeta}
  = \frac{\xi}{r} \pdv{\xi} + \frac{\zeta}{r} \pdv{\zeta}
\end{align*}
\begin{align*}
  \pdv{\theta} = \pdv{\xi}{\theta} \pdv{\xi} + \pdv{\zeta}{\theta} \pdv{\zeta} = r \sin \theta \pdv{\xi} - r  \sin \theta \pdv{\zeta}
\end{align*}
よって,
\begin{align*}
  \pdv{r}\qty(r^2\pdv{r}) 
  &=  \pdv{r} \qty[r^2\qty( \frac{\xi}{r} \pdv{\xi} + \frac{\zeta}{r} \pdv{\zeta})] \\
  &=  r \pdv{r} \qty(\xi \pdv{\xi} + \zeta \pdv{\zeta}) + \qty(\xi \pdv{\xi} + \zeta \pdv{\zeta}) \\
  &= \qty(\xi \pdv{\xi} + \zeta \pdv{\zeta})\qty(\xi \pdv{\xi} + \zeta \pdv{\zeta}) + \qty(\xi \pdv{\xi} + \zeta \pdv{\zeta}) \\
  &= \qty(\xi^2 \pdv[2]{\xi} + \zeta^2 \pdv[2]{\zeta} + 2\xi\zeta \pdv{}{\xi}{\zeta}) + 2\qty(\xi \pdv{\xi} + \zeta \pdv{\zeta}) \\
\end{align*}
$\theta$の微分項は,
\begin{align*}
  \frac{1}{r^2 \sin \theta}\pdv{\theta}\qty(\sin \theta \pdv{\theta}) 
  &= \frac{1}{r^2 \sin \theta}\pdv{\theta} \qty(r \sin^2 \theta \pdv{\xi} - r  \sin^2 \theta \pdv{\zeta}) \\
  &= \frac{2\cos\theta}{r}\qty(\pdv{\xi} - \pdv{\zeta}) + \frac{\sin\theta}{r} \pdv{\theta}\qty( \pdv{\xi} - \pdv{\zeta}) \\
  &= \frac{2\cos\theta}{r}\qty(\pdv{\xi} - \pdv{\zeta}) + \frac{\sin\theta}{r}  \qty(r \sin \theta \pdv{\xi} - r  \sin \theta \pdv{\zeta} )\qty( \pdv{\xi} - \pdv{\zeta}) \\
  &= \frac{2\cos\theta}{r}\qty(\pdv{\xi} - \pdv{\zeta}) +  \qty(\sin^2 \theta \pdv{\xi} - \sin^2 \theta \pdv{\zeta} )\qty( \pdv{\xi} - \pdv{\zeta}) \\
  &= \frac{\zeta - \xi}{r^2}\qty(\pdv{\xi} - \pdv{\zeta}) +  \frac{1}{r^2}\qty( \xi\zeta \pdv{\xi} - \xi\zeta \pdv{\zeta} )\qty( \pdv{\xi} - \pdv{\zeta}) \\
  &= \frac{\zeta - \xi}{r^2}\qty(\pdv{\xi} - \pdv{\zeta}) +  \frac{\xi\zeta}{r^2}\qty(\pdv[2]{\xi} + \pdv[2]{\zeta} - 2 \pdv{}{\xi}{\zeta}) \\
\end{align*}
であるため,
\begin{align*}
  &\frac{1}{r^2}\pdv{r}\qty(r^2\pdv{r}) +  \frac{1}{r^2 \sin \theta}\pdv{\theta}\qty(\sin \theta \pdv{\theta}) \\
  &= \frac{\xi + \zeta}{r^2} \qty(\xi\pdv[2]{\xi} + \zeta\pdv[2]{\zeta}) + \frac{\xi + \zeta}{r^2}\qty(\pdv{\xi} + \pdv{\zeta}) \\
  &= \frac{2}{r}\qty(\xi \pdv[2]{\xi} + \pdv{\xi} + \zeta\pdv[2]{\zeta} + \pdv{\zeta}) \\
  &= \frac{4}{\xi + \zeta} \qty[\pdv{\xi}\qty(\xi\pdv{\xi}) + \pdv{\zeta}\qty(\zeta\pdv{\zeta})]
\end{align*}
を得る.
よって,式\ref{eq:laplacian_polar}は,放物線座標では,
\begin{align}\label{eq:laplacian_parabolic}
  \nabla^2 = \frac{4}{\xi + \zeta} \qty[\pdv{\xi}\qty(\xi\pdv{\xi}) + \pdv{\zeta}\qty(\zeta\pdv{\zeta})] + \frac{1}{\xi\zeta} \pdv[2]{\varphi}
\end{align}
となる.

式\ref{eq:laplacian_parabolic}を用い,式\ref{eq:coulomb_s_schro_rev}に式\ref{eq:coulomb_s_initial_ansatz}を代入する.
ここで,
\begin{align*}
  &\pdv{\xi} e^{ikz} = \pdv{\xi} e^{ik(\zeta-\xi)/2} = - \frac{ik}{2} e^{ikz}\\
  &\pdv{\zeta} e^{ikz} =   \pdv{\zeta} e^{ik(\zeta-\xi)/2} =  \frac{ik}{2} e^{ikz}
\end{align*}
であることと,$g(r-z) = g(\xi)$が$\zeta$に依らないことから,
\begin{align*}
  \pdv{\xi}\qty(\xi \pdv{\xi} e^{ikz}g(\xi)) 
  &= \pdv{\xi}\qty[\xi\qty(\frac{-ik}{2}e^{ikz}g(\xi) + e^{ikz}g'(\xi))] \\
  &=e^{ikz}\qty[\frac{-ik}{2} g(\xi) + g'(\xi) + \xi\qty(-\frac{k^2}{4} g(\xi) - ikg'(\xi) + g''(\xi))] \\
  \pdv{\zeta}\qty(\zeta \pdv{\zeta} e^{ikz} g(\xi)) &= \qty( \frac{ik}{2}  - \zeta \frac{k^2}{4})g(\xi)e^{ikz}
\end{align*}
となる.

よって,それぞれ足し合わせることで,
\begin{align*}
  \qty[\pdv{\xi}\qty(\xi\pdv{\xi}) + \pdv{\zeta}\qty(\zeta\pdv{\zeta})]e^{ikz} g(\xi)
  = \qty[ - \frac{k^2}{4}(\xi+\zeta)g(\xi) + \qty(1 - ik\xi )g'(\xi) + \xi g''(\xi)] e^{ikz}
\end{align*}
となるため,式\ref{eq:coulomb_s_schro_rev}に\ref{eq:coulomb_s_initial_ansatz}を代入すると,
\begin{align}\label{eq:scat_sol_dif_eq}
&\qty[-k^2 g(\xi) + \frac{4}{\xi + \zeta} \qty(1-ik\xi )g'(\xi) + \frac{4\xi}{\xi+ \zeta}g''(\xi) + k^2g(\xi) - \frac{4\eta k}{\xi + \zeta} g(\xi)] e^{ikz} = 0 
\end{align}
であり,変形して,
\begin{align}\label{eq:coulomb_sc_dif_eq}
  \xi g''(\xi) + (1 - ik\xi) g'(\xi) - \eta k g(\xi) = 0
\end{align}
が,$g(\xi)$の従う微分方程式である.
この微分方程式の解は合流型超幾何関数であり,
\begin{align}\label{eq:cofluent_hyp_geom_dif}
  z \dv[2]{F}{z} + (b - z) \dv{F}{z} - aF = 0
\end{align}
の解を,$F(a,b,z)$と書く.

よって,式\ref{eq:coulomb_sc_dif_eq}において,$s = ik\xi$と変数変換をし,$f(s) = g(\xi)$と置き換えることで,
\begin{align*}
  &\frac{s}{ik} (ik)^2 f''(s) + (1 - s) (ik) f'(s) - \eta k f(s) = 0 \\
  & s f''(s) + (1-s) f'(s) - (- i \eta) f(s)  = 0
\end{align*}
となる.よって\ref{eq:coulomb_s_schro_rev}の解は,
\begin{align}\label{eq:coulomb_sc_solution}
  \phi_C(\bm{k};\bm{r}) = C e^{ikz} F(-i\eta, 1, ik(r-z))
\end{align}
である.
\subsection{合流型超幾何関数}
\cite{BN01957611},\cite{igi_kawai1994}を参照されたい.

合流型超幾何関数は,式\ref{eq:cofluent_hyp_geom_dif}の解であり,
ガンマ関数を用いることで,
\begin{align}\label{eq:expansion_cofluent_hyp}
  F(a,b,z) = \sum_{n=0}^{\infty} \frac{\Gamma(a+n)\Gamma(b)}{\Gamma(a)\Gamma(b+n)} \frac{x^n}{n!}
\end{align}
と表される.
\newpage
% \bibliographystyle{unsrt}
% \bibliography{main}
\printbibliography% biblatex
\end{document}