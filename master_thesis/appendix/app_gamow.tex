\appendix
\chapter{ガモフピークの位置および幅の導出}
\label{app:gamow}

\section{ガモフピークの位置}

式\eqref{eq:reaction_rate_s}より,$S(E)$ を緩やかに変化するとみなすと,
反応率の被積分関数は
\begin{equation}
I(E)\propto S(E)\exp\!\left[-\Phi(E)\right],
\qquad
\Phi(E)\equiv \frac{E}{k_BT}+2\pi\eta(E)
\label{eq:I_Phi_app}
\end{equation}
と書ける.したがってガモフピークの位置 $E_0$ は $\Phi(E)$ を最小化することで求められる.

ゾンマーフェルトパラメータ
\(
\eta(E)=Z_1Z_2\alpha\sqrt{\mu c^2/(2E)}
\)
を用いると
\begin{equation}
2\pi\eta(E)=\frac{b}{\sqrt{E}},
\qquad
b\equiv 2\pi Z_1Z_2\alpha\sqrt{\frac{\mu c^2}{2}}
\label{eq:b_def_app}
\end{equation}
より
\begin{equation}
\Phi(E)=\frac{E}{k_BT}+\frac{b}{\sqrt{E}}
\label{eq:Phi_app}
\end{equation}
である.よって,
\begin{equation}
\frac{d\Phi}{dE}
= \frac{1}{k_BT}-\frac{b}{2}E^{-3/2} = 0
\label{eq:dPhi_app}
\end{equation}
を解くと
\begin{equation}
E_0=\left(\frac{b\,k_BT}{2}\right)^{2/3}
\label{eq:E0_app}
\end{equation}
を得る.

\section{ガモフピークの幅 $\Delta$}

次に $E_0$ 近傍で $\Phi(E)$ を2次まで展開すると
\begin{equation}
\Phi(E)\simeq \Phi(E_0)+\frac{1}{2}\Phi''(E_0)(E-E_0)^2
\label{eq:Taylor_app}
\end{equation}
となる.式\eqref{eq:Phi_app}より
\begin{equation}
\Phi''(E)=\frac{3b}{4}E^{-5/2}
\label{eq:Phi2_app}
\end{equation}
である.一方,式\eqref{eq:dPhi_app}から,
\begin{align}
  b = \frac{2}{k_B T} E_0^{3/2}
\end{align}
であるため,
\begin{equation}
\Phi''(E_0)=\frac{3b}{4}E_0^{-5/2}
=\frac{3}{2k_BT}\frac{1}{E_0}
\label{eq:Phi2_E0_app}
\end{equation}
を得る.

ここで,
\begin{align}
  \exp\qty[\Phi(E)] \propto \exp\qty[-\frac{1}{2}\Phi''(E_0)(E-E_0)^2]
\end{align}
から,ガウシアンの幅
\(
  \sigma_{E_0}
\)
は,
\begin{equation}
\sigma_{E_0}^2= \frac{1}{\Phi''(E_0)} = \frac{2}{3}E_0 k_BT
\label{eq:sigmaE_app}
\end{equation}
となる.ガモフピークの幅を
\begin{equation}
\Delta \equiv 2\sigma_E
\label{eq:Delta_def_app}
\end{equation}
と定義すると,
\begin{equation}
\Delta = 2\sqrt{\frac{2}{3}E_0 k_BT}
=\left(\frac{8 E_0 k_BT}{3}\right)^{1/2}
\label{eq:Delta_app}
\end{equation}
を得る.
