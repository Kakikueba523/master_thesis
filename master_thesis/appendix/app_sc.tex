\chapter{散乱波について}\label{app:sa}
式~\eqref{eq:scatt_asympt_form}では解の形を仮定したが,ここでは理論的な定式化を行う.
\section{リップマン・シュウィンガー方程式}
波数ベクトル$\bm{k}$,エネルギー$E_{\bm{k}} := \hbar^2 k^2 / 2\mu$の自由粒子のシュレーディンガー方程式は,
\begin{align}\label{eq:free_p_shcrodinger}
  (E_k - H_0) \ket{\phi(\bm{k})} = 0
\end{align}
であり,$r>\bar{R}$で消えるポテンシャル$V$をもつシュレーディンガー方程式は,
\begin{align}\label{eq:int_p_schrodinger}
  (E_k - H)\ket{\psi(\bm{k})} = 0
\end{align}
と書ける.ここで,$H_0 = K$は運動エネルギー演算子であり,$H = H_0 + V$である.

$H = H^\dagger, H_0 = H_0^\dagger$であることから,$\ket{\phi(\bm{k})}$と$\ket{\psi(\bm{k})}$は,以下の関係で正規化される.
\begin{align}\label{eq:normalization_condition}
  \braket{\phi(\bm{k}')}{\phi(\bm{k})} &= \delta (\bm{k}- \bm{k}') \\
  \braket{\psi(\bm{k}')}{\psi(\bm{k})} &= \delta (\bm{k}-\bm{k}') \\
  \braket{\psi_m}{\psi(\bm{k})} &= 0 \\
  \braket{\psi_m}{\psi_n} &= \delta_{mn}
\end{align}
$m$,$n$はハミルトニアン$H$の負エネルギー固有状態を表している.
$\ket{\psi(\bm{k})}$,$\ket{\phi(\bm{k})}$
における完全系の式は,
\begin{gather}\label{eq:complete_set}
  \int \ket{\phi(\bm{k})} \dd[3]{\bm{k}} \bra{\phi(\bm{k})} = \bm{1} \\
  \int \ket{\psi(\bm{k})} \dd[3]{\bm{k}} \bra{\psi(\bm{k})} + \sum_n \ket{\psi_n} \bra{\psi_n} = \bm{1} 
\end{gather}
である.

freeとfullのグリーン関数はそれぞれ,
\begin{align}\label{eq:def_green_fnc}
  G_0(E) = \frac{1}{E - H_0} \\
  G(E) = \frac{1}{E- H}
\end{align}
と定義される.
$G$と$G_0$の間の関係を求めるために,
\begin{align*}
  A^{-1} = B^{-1} + B^{-1}(B-A)A^{-1}
\end{align*}
という恒等式を用い,$A = E-H = E - (H_0 + V)$,$B = E - H_0$
と置き換えることで,
\begin{align}\label{eq:G_equal_G_0_equation}
  G(E) = G_0(E) + G_0(E)V G(E)
\end{align}
となり,$A = E - H_0$,$B = E-(H_0 +V)$と置き換えることで,
\begin{align}\label{eq:G_0_equal_G_equation}
  G_0(E) = G(E) + G(E)(-V)G_0(E) \notag \\
  G(E) = G_0(E) + G(E) V G_0(E)  
\end{align}
となる.

$\ket{\psi(\bm{k})}$についてみてみよう.
いま,式\eqref{eq:int_p_schrodinger}において,
\begin{align}\label{eq:int_p_shcro_rev}
  (E_k-H_0) \ket{\psi(\bm{k})} = V \ket{\psi(\bm{k})}
\end{align}
と書き換えて,
$\ket{\psi(\bm{k})} = \ket{\phi(\bm{k})} + \ket{\psi^\mathrm{sc} (\bm{k})}$と分解し,式\eqref{eq:int_p_shcro_rev}
に代入し式\eqref{eq:free_p_shcrodinger}を用いると,
\begin{align}\label{eq:int_p_shcro_rev_2}
  (E_k - H_0) \ket{\psi^\mathrm{sc}(\bm{k})} = V \ket{\psi(\bm{k})}
\end{align}
を得る.式\eqref{eq:int_p_shcro_rev_2}に左から$G_0(E_k)$を掛けて,
$\ket{\psi^\mathrm{sc}(\bm{k})} = \ket{\psi (\bm{k})} - \ket{\phi(\bm{k})}$
を代入して,
\begin{align}\label{eq:lippman_schwinger_eq_1}
  \ket{\psi(\bm{k})} = \ket{\phi(\bm{k})} + G_0(E_k) V \ket{\psi(\bm{k})}
\end{align}
を得る.
一方で,式\eqref{eq:int_p_shcro_rev_2}の右辺に,$\ket{\psi(\bm{k})} = \ket{\phi(\bm{k})} + \ket{\psi^\mathrm{sc} (\bm{k})}$
という関係式を用いることで,
\begin{align}\label{eq:inti_p_shcro_rev_3}
  (E_k - H_0 - V) \ket{\psi^\mathrm{sc}(\bm{k})} = V \ket{\phi(\bm{k})}
\end{align}
となり,左から$G(E_k)$を掛けて,
\begin{align*}
  \ket{\psi^\mathrm{sc}(\bm{k})} = G(E_k) V \ket{\phi(\bm{k})}
\end{align*}
を得る.よって,$\ket{\psi(\bm{k})} = \ket{\phi(\bm{k})} + \ket{\psi^\mathrm{sc} (\bm{k})}$
を用いると,
\begin{align}\label{eq:lippman_scwinger_eq_2}
  \ket{\psi(\bm{k})} = \ket{\phi(\bm{k})} + G(E_k) V \ket{\phi(\bm{k})} = (1 + G(E_k) V) \ket{\phi(\bm{k})}
\end{align}
となる.式\eqref{eq:lippman_schwinger_eq_1},\eqref{eq:lippman_scwinger_eq_2}がリップマン・シュウィンガー方程式である.

\section{自由粒子のグリーン関数}
$G_0(E_k)$の座標表示を得る.
まず,自由粒子のグリーン関数について,
完全系で展開すると,
\begin{align}
  G_0(E_k) 
  &= \int \dd[3]{\bm{q}} \frac{\ket{\phi(\bm{q})}\bra{\phi(\bm{q})}}{E_k - E_q} \notag \\
  &= - \qty(\frac{2\mu}{\hbar^2}) \int \dd[3]{\bm{q}} \frac{\ket{\phi(\bm{q})}\bra{\phi(\bm{q})}}{q^2 - k^2} \label{eq:g_0_decomposition}
\end{align}
となる.
式~\eqref{eq:g_0_decomposition}から,~$\mel{\bm{r}}{G_0(E_k)}{\bm{r}'} = G_0(E_k; \bm{r}, \bm{r}')$は,
\begin{align}\label{eq:mel_g0}
  G_0(E_k;\bm{r}, \bm{r}') = - \qty(\frac{2\mu}{\hbar^2}) \int \dd[3]{\bm{q}} \frac{\phi(\bm{q};\bm{r})\phi^*(\bm{q};\bm{r}')}{q^2 - k^2}
\end{align}
となる.ここで,
\begin{align}
  \braket{\bm{r}}{\bm{q}} = \phi(\bm{q};\bm{r}) = \frac{1}{(2\pi)^{3/2}}e^{i \bm{q}\cdot \bm{r}}
\end{align}
を用いた.
\begin{align}
  \bm{R} = \bm{r} - \bm{r}'
\end{align}
を用い,
\begin{align}
  \bm{R} \cdot \bm{q} = Rq \cos\theta_{\bm{q}}
\end{align}
とすると,
式~\eqref{eq:mel_g0}は
\begin{align}
  G_0(E_k; \bm{r}, \bm{r}') = -\qty(\frac{2\mu}{\hbar^2})\frac{1}{(2\pi)^2} \int_{0}^{\infty} \dd{q} \frac{q^2}{q^2 - k^2} \int_{-1}^{1} \dd{\qty(\cos\theta_{q})} e^{iqR\cos\theta_{q}}
\end{align}
となる.グリーン関数における$\bm{r},\bm{r}'$依存性が,その間の距離~$\abs{R} = \abs{\bm{r} -\bm{r}'}$のみに依るのは,$H_0$の並進および回転対称性に起因する.

ここで,
\(
  \cos\theta_{q}
\)
についての積分を実行すると,
\begin{align}\label{eq:mel_g0_2}
  G_0(E_k;\bm{r}, \bm{r}') = \qty(\frac{2\mu}{\hbar^2}) \frac{i}{(2\pi)^2 R} \qty(I_1 + I_2)
\end{align}
となる.ここで,
\begin{align}
  I_1 = \int_{0}^{\infty} \dd{q} \frac{qe^{iqR}}{q^2-k^2}, \qquad I_2 = - \int_{0}^{\infty} \dd{q} \frac{qe^{-iqR}}{q^2-k^2}
\end{align}
とした.
$I_2$について,$q \rightarrow -q$と積分変数を変換すると,
\begin{align}
  I_2 = - \int_{0}^{-\infty} \dd{q} \frac{qe^{iqR}}{q^2 - k^2}
\end{align}
となるため,
\begin{align}
  I_1 + I_2 = \int_{-\infty}^{\infty}\dd{q} \frac{qe^{iqR}}{q^2 - k^2}
\end{align}
となる.また,
\begin{align}
  \frac{q}{q^2-k^2} = \frac{1}{2}\qty(\frac{1}{q-k} + \frac{1}{q+k})
\end{align}
を用いることで,式~\eqref{eq:mel_g0_2}は,
\begin{align}\label{eq:mel_g0_3}
  G_0(E_k;\bm{r},\bm{r}') = \qty(\frac{2\mu}{\hbar^2}) \frac{i}{2(2\pi)^2 R} \int_{-\infty}^{\infty}\dd{q} \qty[\frac{1}{q-k} + \frac{1}{q+k}]e^{iqR}  
\end{align}
となる.式~\eqref{eq:mel_g0_3}は実軸上の極 $q = \pm k$ のために定義できないが,
この極を
\begin{align}
  \pm k \rightarrow \pm ( k + i\epsilon') , \qquad \pm k \rightarrow \pm ( k - i\epsilon')
\end{align}
と動かすことにより定義される
\(
  G_0^{(\pm)}(E_k;\bm{r}, \bm{r}')
\)
は定義できる.これらは,
\begin{align}
  G_0(E_k) \rightarrow G_0^{\pm}(E_k) = \lim_{\epsilon \rightarrow 0}\qty[\frac{1}{E_k - H_0 \pm i \epsilon}]
\end{align}
として対応付けられる.
ここで,
\begin{align*}
  \epsilon = \frac{\hbar^2 k}{\mu}\epsilon'
\end{align*}
は微小量である.
% また,$G_0^{(\pm)}(E_k;\bm{r},\bm{r}')$は,
% \begin{align}
%   \qty[E_k + \frac{\hbar^2}{2\mu}\nabla_{r}^2]G_0^{(\pm)}(E_k;\bm{r},\bm{r}') = \delta(\bm{r} - \bm{r}')
% \end{align}
% の解であることもわかる.
% また,明らかに
% \begin{align}
%   \qty[G_0^{(\pm)}(E_k)]^\dagger = G_0^{(\mp)}(E_k)
% \end{align}
% が成り立つ.

また,
\begin{align}
  \lim_{\epsilon \rightarrow 0} \frac{1}{x \pm i\epsilon} = \mathcal{P} \frac{1}{x} \mp i\pi \delta(x)
\end{align}
という関係式を用いると,
\begin{align}
  G_0^{(\pm)}(E_k) = \mathcal{P}\frac{1}{E_k - H_0}\mp i\pi \delta(E_k - H_0)
\end{align}
となる.$\mathcal{P}$はコーシーの主値である.
第一項がオフシェル,第二項がオンシェル部分である.

極をずらし定義した$G_0^{(\pm)}$について,
\begin{align}\label{eq:mel_g0_4}
  G_0^{(\pm)}(E_k;\bm{r},\bm{r}') =\qty(\frac{2\mu}{\hbar^2}) \frac{i}{2(2\pi)^2 R} I^{(\pm)}
\end{align}
である.ここで,
\begin{align}\label{eq:g0_integrant}
  I^{(\pm)} = \int_{-\infty}^{\infty}\dd{q} \qty[\frac{1}{q - (k\pm i\epsilon')} + \frac{1}{q + (k\pm i \epsilon')}]e^{iqR}
\end{align}
とした.この積分は,図~\ref{fig:contour_k_plane_b}の
経路をとる複素積分として評価される.
\begin{figure}[t]
  \centering
  % \usepackage{tikz}
  % \usetikzlibrary{decorations.markings}

  \begin{tikzpicture}[>=Latex, scale=1.05]
    \def\R{4.0} % 半円の半径

    % ---- axes ----
    \draw[->, thick] (-\R-0.8,0) -- (\R+1.2,0) node[below right] {$\mathrm{Re}\{q\}$};
    \draw[->, thick] (0,-1.4) -- (0,\R+0.9) node[left] {$\mathrm{Im}\{q\}$};

    % ---- contour Gamma (upper semicircle) ----
    \draw[thick,
      postaction={decorate},
      decoration={markings,
        mark=at position 0.18 with {\arrow{Latex}},
        mark=at position 0.48 with {\arrow{Latex}},
        mark=at position 0.78 with {\arrow{Latex}}
      }
    ] (\R,0) arc (0:180:\R);

    % ---- real-axis part with arrows to the right ----
    \draw[thick,
      postaction={decorate},
      decoration={markings,
        mark=at position 0.20 with {\arrow{Latex}},
        mark=at position 0.50 with {\arrow{Latex}},
        mark=at position 0.80 with {\arrow{Latex}}
      }
    ] (-\R,0) -- (\R,0);

    \node at (2.1,3.1) {$\Gamma$};

    % ---- dashed line D ----
    \draw[dashed, thick] (0,0) -- ({\R*cos(35)},{\R*sin(35)});
    \node at (1.55,1.55) {$D$};

    % ---- symbols (open circles + filled dots) ----
    \coordinate (xL) at (-2.4,0);
    \coordinate (xR) at ( 2.4,0);

    \draw[thick] (xL) circle (0.13);
    \fill (xL)+(0,-0.60) circle (0.15);

    \draw[thick] (xR) circle (0.13);
    \fill (xR)+(0,+0.60) circle (0.15);

    % ---- panel label ----
    % \node at (-2.9,2.6) {\Large (b)};
  \end{tikzpicture}

  \caption{複素$q$平面上の積分路.例として,$\pm k \rightarrow \pm(k + i \epsilon) $
  とずらした極を黒丸で書いた.}
  \label{fig:contour_k_plane_b}
\end{figure}
図\ref{fig:contour_k_plane_b}で$D \rightarrow \infty$の極限をとると,
式~\eqref{eq:g0_integrant}の指数関数部分が,
\begin{align*}
  e^{iDR(\cos\theta + i\sin\theta)} = e^{IDR\cos\theta} \times e^{-DR\sin\theta}
\end{align*}
となる.ここで,$q = D(\cos\theta + i \sin \theta)$とした.
$R$が0でなければ,経路$\Gamma$において,$\sin\theta > 0$であることから,
$D \rightarrow \infty$の極限において,
\begin{align*}
  e^{iqR} \rightarrow 0 
\end{align*}
となる.よって,留数定理により
\begin{align}
  I^{(\pm)} = 2\pi i \times \Res[q = \pm(k\pm i\epsilon)] = 2\pi i e^{\pm i kR}
\end{align}
を得る.
よって,式~\eqref{eq:mel_g0_4}は
\begin{align}\label{eq:mel_g0_final}
  G_0^{(\pm)}(E_k; \bm{r}, \bm{r}') = -\qty(\frac{2\mu}{\hbar^2}) \frac{1}{4\pi} \frac{e^{\pm i k \abs{\bm{r}-\bm{r}'}}}{\abs{\bm{r}-\bm{r}'}}
\end{align}
となる.
\section{散乱振幅}
リップマン・シュウィンガー方程式から,座標表示をとると,
\begin{align}
  \psi^{(\pm)}(\bm{k};\bm{r}) = \phi(\bm{k};\bm{r}) + \int \dd[3]{\bm{r}'} G_0^{(\pm)}(E_k;\bm{r},\bm{r}') V(\bm{r}') \psi^{(\pm)}(\bm{k};\bm{r}')
\end{align}
となり,式~\eqref{eq:mel_g0_final}を用いて
\begin{align}\label{eq:sc_g0_mel}
  \psi^{(\pm)}(\bm{k};\bm{r}) = \phi(\bm{k};\bm{r}) - \frac{\mu}{2\pi\hbar^2}\int \dd[3]{\bm{r}'} \frac{e^{\pm ik\abs{\bm{r}-\bm{r}'}}}{\abs{\bm{r} - \bm{r}'}} V(\bm{r}') \psi^{(\pm)}(\bm{k};\bm{r}')
\end{align}
となる.
ここで,$\abs{\bm{r}} \rightarrow \infty$の極限をとる.
\begin{align}
  \frac{1}{\abs{\bm{r}' - \bm{r}}} \rightarrow \frac{1}{r}
\end{align}
であり,
\begin{align*}
  k\abs{\bm{r}-\bm{r}'}  = k\qty(r^2 + r'^2 -2\bm{r}\cdot \bm{r}')^{1/2} \approx kr -k \hat{\bm{r}} \cdot \bm{r}'
\end{align*}
と近似すると,
\begin{align*}
  e^{\pm i k \abs{\bm{r}-\bm{r}'}} \approx e^{\pm ikr} \times e^{ \mp i k \hat{\bm{r}}\cdot \bm{r}'}
\end{align*}
となるため,式~\eqref{eq:sc_g0_mel}は,
\begin{align}\label{eq:sc_asym_g0}
  \psi^{(\pm)}(\bm{k};\bm{r}) \rightarrow \phi(\bm{k};\bm{r}) + \frac{e^{\pm ikr}}{r}  \qty[- \frac{\mu}{2\pi \hbar^2 } \int \dd[3]{\bm{r}'} e^{\mp i \bm{q} \cdot \bm{r}'} V(\bm{r}') \psi^{(\pm)}(\bm{k};\bm{r}')]
\end{align}
となる.ここで,~$\bm{q} = k \hat{\bm{r}}$を用いた.
$\bm{q}$は終状態の波数ベクトルである.また,式~\eqref{eq:sc_asym_g0}を書き換えると,
\begin{align}\label{eq:sc_sym_mel_g0}
  \psi^{(\pm)}(\bm{k}; \bm{r}) \rightarrow \frac{1}{(2\pi)^{3/2}} \qty{ e^{i\bm{k}\cdot \bm{r}} + \frac{e^{ikr}}{r} \qty[-2\pi^2 \qty(\frac{2\mu }{\hbar^2}) \mel{\phi(\pm\bm{q})}{V}{\psi^{(\pm)}(\bm{k})}]}
\end{align}
となる.

\(
  \psi^{(+)}
\)
についての式を見ると,漸近形が平面波と外向きの球面波の重ね合わせで表現されており,
第\ref{chap:scattering_theory}章で仮定した解の形を再現していることが分かる.
ここで,式~\eqref{eq:scatt_asympt_form}と比較すると,散乱振幅は
\begin{align}
  f(\theta) = -2\pi^2 \qty(\frac{2\mu}{\hbar^2})\mel{\phi(\bm{q})}{V}{\psi^{(+)}(\bm{k})}
\end{align}
と書かれることが分かる.
$\theta$は始状態の波数ベクトル$\bm{k}$と終状態の波数ベクトル$\bm{q}$のなす角である.