\chapter{微分方程式の解}
\section{球ベッセル微分方程式}\label{apx:bessel_eq}
球ベッセル微分方程式は,
\begin{align}\label{eq:spherical_bessel_differential_equation}
  \qty[\dv[2]{}{\rho} + \frac{2}{\rho}\dv{\rho} + 1 - \frac{l(l+1)}{\rho^2}]f_l(\rho) = 0
\end{align}
であらわされる.
まず,
\begin{align}\label{eq:f_l_by_chi}
  f_l (\rho) = (-\rho)^l \chi_l(\rho)
\end{align}
と置き換える.これを,式~(\ref{eq:spherical_bessel_differential_equation})に代入し,~$\chi_l$についての式に
書き直すと,
\begin{align*}
  (-\rho)^l \qty[\frac{l(l-1)}{\rho^2} + \frac{2l}{\rho}  \dv{\rho} + \dv[2]{}{\rho}]\chi_l(\rho) 
  + \frac{2}{\rho}(-\rho)^l\qty[\frac{l}{\rho} + \dv{\rho}]\chi_l(\rho) + \qty(1 - \frac{l(l+1)}{\rho^2})\chi_l(\rho) = 0 
\end{align*}
\begin{align}\label{eq:spherical_besse_eq_for_chi}
  \qty[\dv[2]{}{\rho} + \frac{2(l + 1)}{\rho}\dv{\rho} + 1 ]\chi_l(\rho) = 0
\end{align}
となる.この両辺に,~$\frac{1}{\rho}\dv{\rho}$を作用させると,
\begin{align}\label{eq:diff_eq_chi}
  \qty[\frac{1}{\rho}\dv[2]{}{\rho} + \frac{2(l+1)}{\rho^2}\dv{\rho} - \frac{2(l+1)}{\rho^3} + \frac{1}{\rho}]\dv{\chi_l}{\rho} = 0
\end{align}
となる.ここで,微分可能な関数$g(\rho)$について,
\begin{align*}
  \dv[2]{}{\rho}\frac{1}{\rho} g(\rho) &= \frac{1}{\rho} \dv[2]{g(\rho)}{\rho} -  2\frac{1}{\rho^2} \dv{g(\rho)}{\rho} + \frac{2}{\rho^3} g(\rho) \\
  \dv{\rho} \frac{1}{\rho}g(\rho) &= \frac{1}{\rho} \dv{g(\rho)}{\rho} - \frac{1}{\rho^2} g(\rho)
\end{align*}
という関係が成り立つことから,
\begin{align*}
  \frac{1}{\rho}\dv[2]{g(\rho)}{\rho} &= \dv[2]{}{\rho} \frac{1}{\rho} g(\rho) + \frac{2}{\rho} \dv{\rho} \frac{1}{\rho} g(\rho) \\
  \frac{1}{\rho} \dv{g(\rho)}{\rho} &= \dv{\rho} \frac{1}{\rho} g(\rho) + \frac{1}{\rho^2} g(\rho)
\end{align*}
が得られる.

よって,式~(\ref{eq:diff_eq_chi})に代入することで,
\begin{align*}
  \qty[\dv[2]{}{\rho} + \frac{2}{\rho} \dv{\rho} + \frac{2(l+1)}{\rho} \qty(\dv{\rho} + \frac{1}{\rho^2}) - \frac{2(l+1)}{\rho^2} + 1] \frac{1}{\rho} \dv{\chi_l}{\rho} = 0
\end{align*}
となり,まとめると,
\begin{align}\label{eq:diff_eq_chi_henkei}
  \qty[\dv[2]{}{\rho} + \frac{2(l+2)}{\rho} \dv{\rho} + 1] \frac{1}{\rho} \dv{\chi_l}{\rho} = 0
\end{align}
となる.これは式~(\ref{eq:diff_eq_chi})で,$l$を$l+1$で置き換えたものとなっているため,
\begin{align}\label{eq:reccursion_chi}
  \chi_{l+1} = \frac{1}{\rho} \dv{\chi_l}{\rho} 
\end{align}
であることがわかる.
よって,式~(\ref{eq:reccursion_chi})を繰り返し適用することで,
\begin{align}\label{eq:reccurion_chi_ans}
  \chi_l = \qty(\frac{1}{\rho}\dv{\rho})^l \chi_0
\end{align}
となることがわかる.
よって,式~(\ref{eq:f_l_by_chi})に代入することで,
式~(\ref{eq:spherical_bessel_differential_equation})の解が,
\begin{align}\label{eq:sol_of_spherical_bessel_in_deriv}
  f_l(\rho) = (-\rho)^l \qty(\frac{1}{\rho}\dv{\rho})^l \chi_0(\rho) = (-\rho)^l \qty(\frac{1}{\rho}\dv{\rho})^l f_0(\rho)
\end{align}
となることがわかる.$l = 0$の解は,
$f_l(\rho) = F_l(\rho)/\rho$の置き換えの下で,式~(\ref{eq:spherical_bessel_differential_equation})を
$F_l(\rho)$についての式に書き直すと,
\begin{align}\label{eq:diff_eq_l0_F0}
  \qty[\dv[2]{}{\rho} + 1] F_0(\rho) = 0
\end{align}
となるため,$F_0(\rho)$は,$\sin\rho$および$\cos\rho$の線形結合か,$e^{i\rho}$と$e^{-i\rho}$
の線形結合かで表される.
特に,$F_0(\rho) = \sin \rho$つまり,$f_0(\rho) = \sin\rho/\rho$
としたものである,
\begin{align}\label{eq:def_of_spherical_bessel}
  j_l(\rho) = (-\rho)^l \qty(\frac{1}{\rho}\dv{\rho})^l \qty(\frac{\sin\rho}{\rho})
\end{align}
は,球ベッセル関数と呼ばれ,$F_0(\rho) = \cos \rho$つまり,$f_0(\rho) = \cos\rho/\rho$とした,
\begin{align}\label{eq:def_of_nuemann_func}
  n_l(\rho) = (-\rho)^l \qty(\frac{1}{\rho}\dv{\rho})^l\qty(\frac{\cos\rho}{\rho})
\end{align}
は,球ノイマン関数と呼ばれる\footnote{式~\ref{eq:def_of_nuemann_func}について,符号の異なる定義もあることに注意.}.

% また,式~(\ref{eq:f_l_by_chi})と~(\ref{eq:reccursion_chi})から,
% \begin{align}
%   f_{l+1} 
%   &= (-\rho)^{l+1} \frac{1}{\rho} \dv{\rho}\qty(\frac{f_{l}}{(-\rho)^{l}}) \notag \\
%   &= \frac{l}{\rho} f_{l} - \dv{\rho} f_{l}
% \end{align}
% という関係式が得られる.また,式~(\ref{eq:sol_of_spherical_bessel_in_deriv})から,
% \begin{align}
%   f_l = (-\rho)^l
% \end{align}

\section{クーロンポテンシャル}\label{app:coulomb_diff_eq}

放物線座標でのラプラシアンを求める.これには,極座標のラプラシアンから出発するのがよいだろう.
放物線座標は,
\begin{align}\label{eq:parabolic_coordinates}
  \xi = r-z = r(1-\cos \theta), \quad \zeta = r + z = r(1+ \cos \theta)
\end{align}
で定義される,$\xi, \zeta$を用いて,球座標$(r, \theta, \varphi)$から$(\xi, \zeta, \varphi)$へと変換される座標である.

\begin{align}\label{eq:laplacian_polar}
  \nabla^2 = \frac{1}{r^2} \pdv{r}\qty(r^2\pdv{r}) + \frac{1}{r^2 \sin \theta} 
  \pdv{\theta}\qty(\sin \theta \pdv{\theta}) + \frac{1}{r^2 \sin^2 \theta} \pdv[2]{\varphi}
\end{align}
式~(\ref{eq:parabolic_coordinates})を参照しながら,式~(\ref{eq:laplacian_polar})に出てくる各項を求める.
\begin{align*}
  r = \frac{\xi + \zeta}{2}, \quad r^2\sin^2\theta = \xi \zeta
\end{align*}
であり,偏微分項は,
\begin{align*}
  \pdv{r} = \pdv{\xi}{r} \pdv{\xi} + \pdv{\zeta}{r} \pdv{\zeta} = \qty(1- \cos \theta) \pdv{\xi} + \qty(1 + \cos \theta) \pdv{\zeta}
  = \frac{\xi}{r} \pdv{\xi} + \frac{\zeta}{r} \pdv{\zeta}
\end{align*}
\begin{align*}
  \pdv{\theta} = \pdv{\xi}{\theta} \pdv{\xi} + \pdv{\zeta}{\theta} \pdv{\zeta} = r \sin \theta \pdv{\xi} - r  \sin \theta \pdv{\zeta}
\end{align*}
よって,
\begin{align*}
  \pdv{r}\qty(r^2\pdv{r}) 
  &=  \pdv{r} \qty[r^2\qty( \frac{\xi}{r} \pdv{\xi} + \frac{\zeta}{r} \pdv{\zeta})] \\
  &=  r \pdv{r} \qty(\xi \pdv{\xi} + \zeta \pdv{\zeta}) + \qty(\xi \pdv{\xi} + \zeta \pdv{\zeta}) \\
  &= \qty(\xi \pdv{\xi} + \zeta \pdv{\zeta})\qty(\xi \pdv{\xi} + \zeta \pdv{\zeta}) + \qty(\xi \pdv{\xi} + \zeta \pdv{\zeta}) \\
  &= \qty(\xi^2 \pdv[2]{\xi} + \zeta^2 \pdv[2]{\zeta} + 2\xi\zeta \pdv{}{\xi}{\zeta}) + 2\qty(\xi \pdv{\xi} + \zeta \pdv{\zeta}) \\
\end{align*}
$\theta$の微分項は,
\begin{align*}
  \frac{1}{r^2 \sin \theta}\pdv{\theta}\qty(\sin \theta \pdv{\theta}) 
  &= \frac{1}{r^2 \sin \theta}\pdv{\theta} \qty(r \sin^2 \theta \pdv{\xi} - r  \sin^2 \theta \pdv{\zeta}) \\
  &= \frac{2\cos\theta}{r}\qty(\pdv{\xi} - \pdv{\zeta}) + \frac{\sin\theta}{r} \pdv{\theta}\qty( \pdv{\xi} - \pdv{\zeta}) \\
  &= \frac{2\cos\theta}{r}\qty(\pdv{\xi} - \pdv{\zeta}) + \frac{\sin\theta}{r}  \qty(r \sin \theta \pdv{\xi} - r  \sin \theta \pdv{\zeta} )\qty( \pdv{\xi} - \pdv{\zeta}) \\
  &= \frac{2\cos\theta}{r}\qty(\pdv{\xi} - \pdv{\zeta}) +  \qty(\sin^2 \theta \pdv{\xi} - \sin^2 \theta \pdv{\zeta} )\qty( \pdv{\xi} - \pdv{\zeta}) \\
  &= \frac{\zeta - \xi}{r^2}\qty(\pdv{\xi} - \pdv{\zeta}) +  \frac{1}{r^2}\qty( \xi\zeta \pdv{\xi} - \xi\zeta \pdv{\zeta} )\qty( \pdv{\xi} - \pdv{\zeta}) \\
  &= \frac{\zeta - \xi}{r^2}\qty(\pdv{\xi} - \pdv{\zeta}) +  \frac{\xi\zeta}{r^2}\qty(\pdv[2]{\xi} + \pdv[2]{\zeta} - 2 \pdv{}{\xi}{\zeta}) \\
\end{align*}
であるため,
\begin{align*}
  &\frac{1}{r^2}\pdv{r}\qty(r^2\pdv{r}) +  \frac{1}{r^2 \sin \theta}\pdv{\theta}\qty(\sin \theta \pdv{\theta}) \\
  &= \frac{\xi + \zeta}{r^2} \qty(\xi\pdv[2]{\xi} + \zeta\pdv[2]{\zeta}) + \frac{\xi + \zeta}{r^2}\qty(\pdv{\xi} + \pdv{\zeta}) \\
  &= \frac{2}{r}\qty(\xi \pdv[2]{\xi} + \pdv{\xi} + \zeta\pdv[2]{\zeta} + \pdv{\zeta}) \\
  &= \frac{4}{\xi + \zeta} \qty[\pdv{\xi}\qty(\xi\pdv{\xi}) + \pdv{\zeta}\qty(\zeta\pdv{\zeta})]
\end{align*}
を得る.
よって,式~(\ref{eq:laplacian_polar})は,放物線座標では,
\begin{align}\label{eq:laplacian_parabolic}
  \nabla^2 = \frac{4}{\xi + \zeta} \qty[\pdv{\xi}\qty(\xi\pdv{\xi}) + \pdv{\zeta}\qty(\zeta\pdv{\zeta})] + \frac{1}{\xi\zeta} \pdv[2]{\varphi}
\end{align}
となる.

では,これを用いて,
\begin{align}\label{eq:coulomb_s_schro_henkei}
  \qty[\nabla^2 + k^2- \frac{2\eta k}{r}] \phi_C(\bm{k}; \bm{r}) = 0
\end{align}
を解こう.
本文中と同様に,
入射ビームの方向を$z$軸にとり,
\begin{align}\label{eq:coulomb_s_initial_katei}
  \phi_C(\bm{k};\bm{r}) = C e^{ikz}g(r-z)
\end{align}
と解の形を仮定する.

式~(\ref{eq:laplacian_parabolic})を用い,式~(\ref{eq:coulomb_s_schro_henkei})に式~(\ref{eq:coulomb_s_initial_katei})を代入する.
ここで,
\begin{align*}
  &\pdv{\xi} e^{ikz} = \pdv{\xi} e^{ik(\zeta-\xi)/2} = - \frac{ik}{2} e^{ikz}\\
  &\pdv{\zeta} e^{ikz} =   \pdv{\zeta} e^{ik(\zeta-\xi)/2} =  \frac{ik}{2} e^{ikz}
\end{align*}
であることと,$g(r-z) = g(\xi)$が$\zeta$に依らないことから,
\begin{align*}
  \pdv{\xi}\qty(\xi \pdv{\xi} e^{ikz}g(\xi)) 
  &= \pdv{\xi}\qty[\xi\qty(\frac{-ik}{2}e^{ikz}g(\xi) + e^{ikz}g'(\xi))] \\
  &=e^{ikz}\qty[\frac{-ik}{2} g(\xi) + g'(\xi) + \xi\qty(-\frac{k^2}{4} g(\xi) - ikg'(\xi) + g''(\xi))] \\
  \pdv{\zeta}\qty(\zeta \pdv{\zeta} e^{ikz} g(\xi)) &= \qty( \frac{ik}{2}  - \zeta \frac{k^2}{4})g(\xi)e^{ikz}
\end{align*}
となる.

よって,それぞれ足し合わせることで,
\begin{align*}
  \qty[\pdv{\xi}\qty(\xi\pdv{\xi}) + \pdv{\zeta}\qty(\zeta\pdv{\zeta})]e^{ikz} g(\xi)
  = \qty[ - \frac{k^2}{4}(\xi+\zeta)g(\xi) + \qty(1 - ik\xi )g'(\xi) + \xi g''(\xi)] e^{ikz}
\end{align*}
となるため,式~(\ref{eq:coulomb_s_schro_rev})に(\ref{eq:coulomb_s_initial_ansatz})を代入すると,
\begin{align}\label{eq:scat_sol_dif_eq}
&\qty[-k^2 g(\xi) + \frac{4}{\xi + \zeta} \qty(1-ik\xi )g'(\xi) + \frac{4\xi}{\xi+ \zeta}g''(\xi) + k^2g(\xi) - \frac{4\eta k}{\xi + \zeta} g(\xi)] e^{ikz} = 0 
\end{align}
であり,変形して,
\begin{align}
  \xi g''(\xi) + (1 - ik\xi) g'(\xi) - \eta k g(\xi) = 0
\end{align}
が,$g(\xi)$の従う微分方程式である.
上式において,$s = ik\xi$と変数変換をし,$f(s) = g(\xi)$と置き換えることで,
\begin{align*}
  s f''(s) + (1-s) f'(s) - (- i \eta) f(s)  = 0
\end{align*}
なる,合流型超幾何微分方程式を得る.

以下では,$b$は正の整数とする.
合流型超幾何関数は,式~(\ref{eq:cofluent_hyp_geom_dif})の解であり,
ガンマ関数を用いることで,
\begin{align}\label{eq:expansion_cofluent_hyp}
  F(a,b,z) = \sum_{n=0}^{\infty} \frac{\Gamma(a+n)\Gamma(b)}{\Gamma(a)\Gamma(b+n)} \frac{x^n}{n!}
\end{align}
と表される.

実際に,式~(\ref{eq:cofluent_hyp_geom_dif})から,級数解を求めてみる.
  \begin{align}\tag{\ref{eq:cofluent_hyp_geom_dif}}
    z \dv[2]{F}{z} + (b - z) \dv{F}{z} - aF = 0
  \end{align}
  の解は,
  \begin{align}\tag{\ref{eq:expansion_cofluent_hyp}}
    F(a,b,z) = \sum_{n=0}^{\infty} \frac{\Gamma(a+n)\Gamma(b)}{\Gamma(a)\Gamma(b+n)} \frac{z^n}{n!}
  \end{align}
  と書かれる.
  式~(\ref{eq:cofluent_hyp_geom_dif})について,$F$を
  \begin{align}\label{eq:cofluent_initial_guess}
    F(z) = \sum_{n = 0}^{\infty} c_n \frac{z^n}{n!}
  \end{align}
  と置く.式~(\ref{eq:cofluent_initial_guess})を式~(\ref{eq:cofluent_hyp_geom_dif})に代入して,漸化式を求める.
  まず,各項について求めると,
  \begin{align*}
    z \dv[2]{F}{z} &= \sum_{n = 2}^{\infty} c_n \frac{z^{n-1}}{(n-2)!} = \sum_{n=0}^{\infty} n c_{n+1} \frac{z^n}{n!} \\
    (b - z) \dv{F}{z} &= \sum_{n=1}^{\infty} c_n (b-z) \frac{z^{n-1}}{(n-1)!} = \sum_{n=0}^{\infty} (b c_{n+1} - n c_n) \frac{z^n}{n!} \\
    a F(z) &= \sum_{n=0}^{\infty} a c_n \frac{z^n}{n!} .
  \end{align*}
  よって,
  \begin{align}\label{eq:reccurance_cofluent_coeff}
    \sum_{n=0}^{\infty} \qty[\qty(n+b)c_{n+1} - (n + a) c_n] \frac{z^n}{n!} = 0
  \end{align}
  となる.式~(\ref{eq:reccurance_cofluent_coeff})が任意の$z$で成り立つことから,
  \begin{align}\label{eq:coeff_ratio}
    \frac{c_{n+1}}{c_n} = \frac{n+a}{n+b}
  \end{align}
  となるため,
  \begin{align}\label{eq:coeff_det}
    c_n = c_0 \prod_{k=0}^{n-1} \qty(\frac{k+a}{k+b}) = c_0 \frac{\Gamma(n + a)}{\Gamma(a)} \frac{\Gamma(b)}{\Gamma(n + b)}
  \end{align}
  を得る.
  ここで,$\Gamma(n+a) = (n-1+a) \Gamma(n-1 + a) = \cdots = (n-1 + a)(n-2 + a) \cdots (1 + a) a\Gamma(a)$を用いた.

  式,\ref{eq:coeff_det}を式~(\ref{eq:cofluent_initial_guess})に代入することで,
  \begin{align}\label{eq:solution_cofluent_diffeq}
    F(z) = c_0 \sum_{n=0}^{\infty}  \frac{\Gamma(a+n)\Gamma(b)}{\Gamma(a)\Gamma(b+n)} \frac{z^n}{n!}.
  \end{align}
  となり,式~(\ref{eq:solution_cofluent_diffeq})の係数を1としたものが,式~(\ref{eq:expansion_cofluent_hyp})となっている.

% ここで,式~(\ref{eq:cofluent_hyp_geom_dif})の解について,
% \begin{align}
%   F(z) = \int_C \dd{t} e^{tz} v(t)
% \end{align}
% の形を仮定し,代入してみると,$v(t)$についての微分方程式が得られる.
% \begin{align}
%   \dv{F}{z} = \int_C \dd{t} e^{tz} t v(t), \quad \dv[2]{F}{z} = \int_C \dd{t} e^{tz} t^2 v(t)
% \end{align}
% および,
% \begin{align}
%   z e^{tz} = \dv{t} e^{zt}
% \end{align}
% を用いることで,
% \begin{align}
%   \int_C \dd{t} \qty[\qty(t^2 - t)\dv{e^{zt}}{t} + (bt - a) e^{zt}]v(t) = 0
% \end{align}
% となる.ここで,第一項について部分積分をすると,
% \begin{align}
%   \int_C \dd{t} e^{zt}\qty{-(t^2 - t)\dv{v}{t} + \qty[(b-2)t -a + 1]v} + \qty[(t^2-t)ve^{zt}]_C = 0
% \end{align}
% となる.ここで,$\qty[\quad]_C$は経路$C$の始点と終点の差で定義される.
% $[\quad]_C = 0$とし,非積分関数$=0$から,
% \begin{align}
%   v(t) = A t^{a-1}(1-t)^{b-a-1}
% \end{align}
% となり,
% \begin{align}
%   F(z) = A \int_C \dd{t} t^{a-1} (1-t)^{b-a-1} e^{tz} 
% \end{align}
% を得る.ただし,積分経路$C$として,
% \begin{align}
%   \qty[t^a (1-t)^{b-a} e^{zt}]_C = 0
% \end{align}
% を課す.また,式~(\ref{eq:cofluent_hyp_geom_dif})において,$F = z^{1-b} F_1$と置き換えることで,
% $F_1$についての微分方程式
% \begin{align}
%   z \dv[2]{F_1}{z} + (2-b-z)\dv{F_1}{z} - (1-b+a)F_1 = 0
% \end{align}
% を得る.この方程式の解は,$F(1-b+a, 2-b, z)$であり,積分系は,同様にして,
% \begin{align}
%   F(z) = A_1\int_{C_1} \dd{t}e^{zt} t^{a-b}(1-t)^{-a}
% \end{align}
% となる.ただし経路$C_1$は,
% \begin{align}
%   \qty[t^{1-b+a}(1-t)^{1-a}e^{zt}]_{C_1} = 0
% \end{align}
% となるように取る.これらは$b=1$では同一の解となるが,それ以外では線形独立な解となっているため,
% 式~(\ref{eq:cofluent_hyp_geom_dif})の一般解は,
% \begin{align}
%   F = c_1 F(a, b, z) + c_2 z^{1-b} F(1-b+a, 2-b, z)
% \end{align}
% となる.
% ここで,規格化定数$A$を求めよう.
% 積分経路として,\ref{fig:dogbone_contour}をとる.
% ただし,$C_0$は実軸上において$\arg t - \arg(1-t) = 0$に取る.
% \begin{figure}[thbp]
%     \centering
%     \begin{tikzpicture}[scale=4]
%         % --- パラメータ設定 ---
%         \def\r{0.15}    % 両端の円の半径
%         \def\ang{30}    % 通路との接続角度 (degree)
        
%         % 補助計算: 直線と円の接点の座標
%         % 0の周りの接点: (r*cos(ang), r*sin(ang))
%         % 1の周りの接点: (1-r*cos(ang), r*sin(ang))
%         \pgfmathsetmacro{\px}{ \r*cos(\ang) }
%         \pgfmathsetmacro{\py}{ \r*sin(\ang) }
%         \pgfmathsetmacro{\qx}{ 1 - \r*cos(\ang) }

%         % 座標軸
%         \draw[->, >=stealth, gray!40] (-0.3, 0) -- (1.3, 0) node[right, black] {Re($t$)};
%         \draw[->, >=stealth, gray!40] (0, -0.3) -- (0, 0.3) node[above, black] {Im($t$)};

%         % 特異点 0 と 1
%         \fill (0,0) circle (0.5pt) node[below left] {$0$};
%         \fill (1,0) circle (0.5pt) node[below right] {$1$};

%         % 分枝切断線 (Branch Cut)
%         \draw[dashed, red!80] (0,0) -- (1,0);

%         % ドッグボーン輪郭の描画
%         \draw[
%             blue, thick,
%             decoration={
%                 markings,
%                 mark=at position 0.10 with {\arrow{Stealth}},
%                 mark=at position 0.35 with {\arrow{Stealth}},
%                 mark=at position 0.60 with {\arrow{Stealth}},
%                 mark=at position 0.85 with {\arrow{Stealth}}
%             },
%             postaction={decorate}
%         ]
%         % 1. 右端の円弧から開始 (1の右側を回る)
%         (1+\r, 0) arc (0 : {180-\ang} : \r) -- 
%         % 2. 上側の水平線 (1から0へ)
%         (\px, \py) arc (\ang : {360-\ang} : \r) -- 
%         % 3. 下側の水平線 (0から1へ)
%         (\qx, -\py) arc ({180+\ang} : 360 : \r) -- cycle;

%         % ラベル
%         \node[blue, above] at (0.5, \py) {$C_0$};
%     \end{tikzpicture}
%     \caption{$t=0$ と $t=1$ を囲むドッグボーン型(ダンベル型)積分路}
%     \label{fig:dogbone_contour}
% \end{figure}

% $F(0) = 1$であることから,
% \begin{align}
%   A^{-1} = \int_{C_0} \dd{t} t^{a-1}(1-t)^{b-1-a} = \int_{C_0} \dd{t} t^{b-2} \qty(\frac{t}{1-t})^{a-1}
% \end{align}
% となる.$t = 0$および,$t = 1$を囲む積分の寄与は0になる.
% また,$c$が正の整数であることから,$t^{b-2}$は1価関数であり,$\arg t - \arg(1-t) = 0$という仮定から,下を通る積分の値と
% 上を通る積分の値では~$e^{-2i\pi a}$だけ位相がずれる.
% よって,
% \begin{align}
%   A^{-1} = (1-e^{-2\pi i a})B(a, b-a)
% \end{align}
% となる.ここで,$B(a, b-a)$はベータ関数であり,ガンマ関数を用いて
% \begin{align}
%   B(a, b-a) = \frac{\Gamma(a)\Gamma(b-a)}{\Gamma(b)}
% \end{align}
% と表されるために,
% \begin{align}
%   A = \qty(1-e^{-2\pi i a})^{-1} \frac{\Gamma(b)}{\Gamma(a)\Gamma(b-a)}
% \end{align}
% となる.
% よって,合流型超幾何関数をラプラスの方法で積分表示すると,
% \begin{align}
%   F(a, b, z) = \frac{\Gamma(b)}{\qty(1-e^{-2\pi i a})\Gamma(a)\Gamma(b-a)} \int_{C_0} \dd{t}  t^{a-1} (1-t)^{b-a-1} e^{tz} 
% \end{align}
% となる.

% \begin{figure}[t]
%     \centering
%     \begin{tikzpicture}[scale=2.5]
%         % --- パラメータ設定 ---
%         \def\eps{0.2}      % 実軸からのズレ(通路の幅の半分)
%         \def\L{2.0}        % 虚軸方向への長さ

%         % 座標軸
%         \draw[->, >=stealth, gray!50] (-1.0, 0) -- (1.0, 0) node[right, black] {Re($z$)};
%         \draw[->, >=stealth, gray!50] (0, -\L-0.5) -- (0, 1.0) node[above, black] {Im($z$)};

%         % 原点の特異点
%         \fill (0,0) circle (1pt) node[above left] {$0$};
        
%         % 分枝切断線 (Branch Cut) - 負の虚軸
%         \draw[dashed, red!60, thick] (0,0) -- (0, -\L-0.2);

%         % Hankel経路の描画 (下から昇ってきて、原点を回り、下へ降りる)
%         \draw[
%             blue, thick,
%             decoration={
%                 markings,
%                 mark=at position 0.2 with {\arrow{Stealth}},
%                 mark=at position 0.5 with {\arrow{Stealth}},
%                 mark=at position 0.8 with {\arrow{Stealth}}
%             },
%             postaction={decorate}
%         ]
%         % 1. 右側:-i∞ 方向から原点へ向かう直線
%         (\eps, -\L) -- (\eps, 0) 
%         % 2. 原点周りの半円 (0度から180度まで反時計回り)
%         arc (0:180:\eps) 
%         % 3. 左側:原点から -i∞ 方向へ向かう直線
%         -- (-\eps, -\L);

%         % 経路のラベル
%         \node[blue, right] at (\eps, -\L/2) {$C_i$};
        
%         % 無限大の記号
%         \node[below] at (0, -\L) {$-i\infty$};

%     \end{tikzpicture}
%     \caption{虚軸の負の方向に伸びるHankel型積分経路.$i = 1$で$t=0$,$i = 2$で$t= 1$を囲む経路とする.}
%     \label{fig:hankel_negative_im}
% \end{figure}
% 経路$C_0$を$C_1$と$C_2$の和で書き換える.そうすると,二つの積分の寄与の和としてあらわされる.具体的には,
% \begin{align}
%   W_i ( a, b, z) = \frac{\Gamma(b)}{(1-e^{-2\pi i a})\Gamma(a)\Gamma(b-a)}\int_{C_i} \dd{t} e^{zt} t^{a-1} ( 1-t)^{b-a-1}
% \end{align}
% を用いる.ただし,$i=1,2$である.
% このとき,
% \begin{align}
%   F = W_1 + W_2
% \end{align}
% である.$W_1$について,
% \begin{align}
%   \int_{C_1} \dd{t} e^{zt}t^{a-1}(1-t)^{b-a-1}
% \end{align}
% という積分に関して,$zt \rightarrow -t$とすると,
% \begin{align}
%   \int_{C_1} \frac{\dd{t}}{-z}e^{-t} \qty(\frac{-t}{z})^{a-1}\qty(1+\frac{t}{z})^{b-a-1} \notag \\
%   = (-z)^{-a} \int_{C_1}\dd{t} e^{-t}t^{a-1} (1+\frac{t}{z})^{b-a-1}
% \end{align}
% である.よって,$\abs{t/z} \ll 1$のとき,
% \begin{align}
%   \qty(1 + \frac{t}{z})^{b-a-1} = \sum_{n=0}^{\infty} \frac{\Gamma(1+a-b+n)}{\Gamma(1+a-b)}\frac{(-z)^{-n}}{n!} t^n
% \end{align}
% と展開できるため,項別にみると,
% \begin{align}
%   \frac{\Gamma(1+a-b+n)}{\Gamma(1+a-b)}\frac{(-z)^{-n}}{n!} \int_{C_1} \dd{t} t^{n+a-1} e^{-t}
% \end{align}
% となる.この積分は,branchの定義に気を付けて実行すると,
% \begin{align}
%   W_1(a,b,z) \sim \frac{\Gamma(b)}{\Gamma(b-a)}(-z)^{-a} \sum_{n} \frac{\Gamma(n+a)}{\Gamma(a)} \frac{\Gamma(1+a-b+n)}{\Gamma(1+a-b)} \frac{(-z)^{-n}}{n!}
% \end{align}
% となる.また,
% \begin{align}
%   W_2(a, b, z) = e^{z} W_1(b-a,b,-z)
% \end{align}
% という関係式を用いると,
% \begin{align}
%   W_2(a, b, z) \sim  \frac{\Gamma(b)}{\Gamma(a)} e^{z} z^{b-a} \sum_n \frac{\Gamma(1-a+n)}{\Gamma(1-a)}\frac{\Gamma(b-a+n)}{\Gamma(b-a)} \frac{z^{-n}}{n!}
% \end{align}
% となる.